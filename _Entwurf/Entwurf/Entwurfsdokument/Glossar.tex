\chapter{Glossar}


% F?r Abk?rzungen: s. LaTeX: \footnote, Zeigt am Ende der Seite z.B. f?r XYZ?, ? XYZ ist ... an.


\begin{longtable}{p{0.35\textwidth} p{0.5\textwidth}}



Bounding-Frustum & Ein Frustum, das z.B. aus der View- und Projection-Matrix berechnet werden kann und dabei hilft zu bestimmen, ob ein 3D-Modell in dem f"ur den Spieler sichtbaren Bereich des 3D-Raums liegt. \footnote{Genauere Definition: \url{http://msdn.microsoft.com/de-de/library/microsoft.xna.framework.boundingfrustum(v=xnagamestudio.40).aspx}} \\



Einstellungsdatei & Eine Datei im INI-Format, die die Einstellungen des Spiels speichert. Die zentrale Einstellungsdatei von Knot$^3$ liegt unter Windows in einem Unterordner des Dokumente-Ordners des aktuellen Windows-Benutzers, beispielsweise in \textit{C:\textbackslash Users\textbackslash Name\textbackslash Documents\textbackslash Knot3\textbackslash config.ini}. Wird Knot$^3$ unter Linux in Verbindung mit der offenen XNA-Implementierung Monogame gespielt, wird eine Datei aus dem Unterverzeichnis ".knot3" des Homeverzeichnisses ausgelesen, etwa \textit{/home/name/.knot3/config.ini}. \\



Inputhandler & Ein Inputhandler ist eine M"oglichkeit um Benutzereingaben zu verarbeiten. Der Imputhandler stellt beispielsweise Informationen "uber die Position der Maus bereit.  \\



Rendereffekt & Ein Effekt, der auf 3D-Modelle oder Sprites angewandt wird und durch Shader realisiert wird. \\



Rendertarget & Ein Rendertarget ist ein Pixel-Buffer. \\



Spielkomponentenliste & Die Sammlung von GameComponent-Elementen, die im Besitz des Spiels sind. \newline Quelle: \url{http://msdn.microsoft.com/de-de/library/microsoft.xna.framework.game.components(v=xnagamestudio.40).aspx} \\



Sprite & Ein Sprite (engl. unter anderem f"ur ein Geistwesen, Kobold) ist ein Grafikobjekt, das von der Grafikhardware "uber das Hintergrundbild bzw. den restlichen Inhalt der Bildschirmanzeige eingeblendet wird. Die Positionierung wird dabei komplett von der Grafikhardware erledigt. \newline Quelle: \url{http://de.wikipedia.org/wiki/Sprite_(Computergrafik)} \\



Spritestapel & Erm"oglicht das Zeichnen einer Gruppe von Sprites mithilfe derselben Einstellungen. \newline Quelle: \url{http://msdn.microsoft.com/de-de/library/microsoft.xna.framework.graphics.spritebatch(v=xnagamestudio.40).aspx}\\



Widget & Ein Widget ist eine Komponente eines grafischen Fenstersystems. Das Widget besteht zum einen aus dem Fenster, einem sichtbaren Bereich, der Maus- und/oder Tastaturereignisse empf"angt, und zum anderen aus dem nicht sichtbaren Objekt, das den Zustand der Komponente speichert und "uber bestimmte Zeichenoperationen den sichtbaren Bereich ver"andern kann. \newline Quelle: \url{http://de.wikipedia.org/wiki/Widget} \\



XNA-Content-Pipeline & Die XNA Game Studio-Inhalts-Pipeline ist eine Gruppe von Prozessen, die beim Build eines Spiels mit Grafikassets angewendet werden. Der Prozess beginnt mit einem Grafikasset, das urspr"unglich als Datei vorliegt. Dieses Grafikasset wird dann in Daten transformiert, die "uber die XNA Framework-Klassenbibliothek abgerufen und in einem XNA Game Studio-Spiel verwendet werden k"onnen. \newline Quelle: \url{http://msdn.microsoft.com/de-de/library/bb447745.aspx#fbid=cm5w_zXCQc2} \\



\end{longtable}