% !TeX encoding = UTF-8
%
% Glossar für Projekt-spezifische Fachausdrücke.
%


%%%
%%% Wörter und deren Definition im Projekt-Sprachgebrauch.
%%%


%%
%% Nutzer.
%%
\newglossaryentry{fa:Nutzer}{
%
	name = Nutzer,
	description = {(Beschreibung) ...},
	type = FA
%
}
\newglossaryentry{fa:Spieler}{
%
	name = Spieler,
	description = {So wird der Nutzer des Programmes genannt},
	type = FA
%
}

\newglossaryentry{fa:Spielername}{
%
	name = Spielername,
	description = {Der Name der in den Bestenliste für den aktuellen Spieler eingetragen werden},
	type = FA
%
}

%
% Menüs
%
\newglossaryentry{fa:Hauptmenu}{
%
	name = {Hauptmen{\"u}},
	description = {Dieses Menü ist der erste Bildschirm mit dem der Spieler interagieren kann. Hier kann er Einstellungen zum Spiel vornehmen (z.B. Grafik und Ton).},
	type = FA
%
}
\newglossaryentry{fa:Pausemenu}{
%
	name = {Pause-Men{\"u}},
	description = {Sonderform vom Hauptmenü in dem Einstellungen zum laufenden Spiel getätigt werden können (z.B. Speichern, Laden, Grafikeinstellungen, Rückkehr zum Hauptmenü (beenden des aktuellen Spiels) und Verlassen Spiels)},
	type = FA
%
}
\newglossaryentry{fa:Tutorial}{
%
	name = Tutorial,
	description = {In diesen speziellen Challenge-Leveln wird dem Spieler die Bedienung des Spieles erklärt.},
	type = FA
%
}
%
% Spielmodi
%
\newglossaryentry{fa:Knoten}{
%
	name = Knoten,
	description = {Im Spiel arbeitet der Spieler an einem dreidimensionalen (Gitter-)Knoten, dabei beginnt er mit einer Ausgangsform (im Zweidimensionalen z.B. ein Quadrat). Wie am Beispiel des Quadrats zu sehen ist, besteht ein Knoten aus einem geschlossenen Gebilde.},
	type = FA
%
}
\newglossaryentry{fa:Kante}{
%
	name = Kante,
	description = {Die Verbindung zwischen zwei Rasterpunkten des Knotens.},
	type = FA
%
}
\newglossaryentry{fa:Creative}{
%
	name = Creative,
	description = {Der Creative(-Mode) ist der erste Spielmodus. Im Creative(-Mode) baut der Spieler ausgehend von einer Grundform einen beliebigen (Gitter-)Knoten. Das Spiel gibt dem Spieler einige Hilfsfunktionen zur Bewertung der Komplexität seines gebauten Knotens.},
	type = FA
%
}

\newglossaryentry{fa:Challenge}{
%
	name = Challenge,
	description = {Der Spieler bekommt die Aufgabe einen vorgegebenen Knoten nachzubauen.},
	type = FA
%
}
\newglossaryentry{fa:Modifikation}{
%
	name = Modifikation,
	description = {Beschreibt eine beliebige Änderung am Knoten. Umfasst damit Transformationen, Einfärben, ... alles was den Knoten ändert.},
	type = FA
%
}
\newglossaryentry{fa:Transformation}{
%
	name = Transformation,
	description = {Verändern des Knoten durch Verschiebung der Kanten und Teilkanten.},
	type = FA
%
}
\newglossaryentry{fa:Datenaustauschformat}{
%
	name = Datenaustauschformat,
	description = {Das Speicherformat der Level wie in den Produktdaten beschrieben. Knoten-Speicherformat, das auch für \gls{ak:PSE}-Gruppe Queup verwendbar ist.},
	type = FA
%
}
\newglossaryentry{fa:Level}{
%
	name = Level,
	description = { In sich beendetes Spiel: Eine Challenge ist gleichzeitig ein Level. Ein Level hat einen Startknoten und einen Zielknoten. Transformiert der Spieler den Startknoten durch mehrere Schritte in den Zielknoten, so ist das Level beendet. Es gibt verschiedene Standard-Levels, welche von 1-10 mit steigender Schwierigkeit geordnet sind.},
	type = FA
%
}
\newglossaryentry{fa:Ausgangsknoten}{
%
	name = Ausgangsknoten,
	description = { Diesen Knoten muss der Spieler im Challenge-Mode transfomieren, sodass er dem Referenzknoten gleicht.},
	type = FA
%
}
\newglossaryentry{fa:Referenzknoten}{
%
	name = Referenzknoten,
	description = { Bildet die Referenz für die Transformation des Ausgangsknoten im Challengen-Mode.},
	type = FA
%
}
\newglossaryentry{fa:Rasterpunkt}{
%
	name = Rasterpunkt,
	description = { Gitterpunktepunkte im Raum die in einem vorgegeben Abstand zu einander sind.},
	type = FA
%
}
\newglossaryentry{fa:Undo}{
%
	name = Undo,
	description = { Mit der Undo-Funktion kann eine vorherige Transformation zurückgenommen werden.},
	type = FA
%
}
\newglossaryentry{fa:Redo}{
%
	name = Redo,
	description = { Mit der Redo-Funktion kann eine vorherige Undo-Aktion wiederhergestellt werden.},
	type = FA
%
}
\newglossaryentry{fa:Shadereffekte}{
%
	name = Shadereffekte,
	description = { Mit der Undo-Funktion kann eine vorherige Transformation zurückgenommen werden.},
	type = FA
%
}
\newglossaryentry{fa:Rendereffekte}{
%
	name = Rendereffekte,
	description = { Mit der Undo-Funktion kann eine vorherige Transformation zurückgenommen werden.},
	type = FA
%
}
\newglossaryentry{fa:Bestenliste}{
%
	name = Bestenliste,
	description = { Mit der Undo-Funktion kann eine vorherige Transformation zurückgenommen werden.},
	type = FA
%
}
\newglossaryentry{fa:Abschlussbildschirm}{
%
	name = Abschlussbildschirm,
	description = { Ist der eingeblendete Bildschirm nach dem erfolgreichen Abschluss eines Levels im Challenge-Mode. Hier wird Platzierung des Spielers in der Bestenliste angezeigt (anhand der Spielzeit) und der Spieler kann das Level bewerten.},
	type = FA
%
}
\newglossaryentry{fa:virtuellerKnoten}{
%
	name = virtueller Knoten,
	description = { Wenn ein Spieler in Knot³ einen Zug ausführt, werden ihm durch eine vorläufige Skizzierung der Knoten-Transformationen (je nach Interaktion) die möglichen Resultate des Zugs gezeigt. },
	type = FA
%
}
\newglossaryentry{fa:Easteregg}{
%
	name = Easteregg,
	description = {Versteckte Funktionen und Spielinhalte.},
	type = FA
%
}
\newglossaryentry{fa:Knot3D}{
%
	name = {Knot},
	description = {Steht sowohl für das Spielkonzept als auch für den Namen des zu entwickelnden Spiels.},
	type = FA
%
}
\newglossaryentry{fa:Komplexitaet}{
%
	name = {Komplexit{\"a}t},
	description = {Methode, um die Knotenkomplexität und somit die Schwierigkeit eines Knotens zu bewerten.},
	type = FA
%
}
\newglossaryentry{fa:Zug}{
%
	name = {Zug},
	description = {Ein (Spiel-)Zug ist die Interaktion des Spielers mit dem 3D-Modell des Knotens, um selbigen zu  transformieren. Zug meint i. A. einen gültigen Zug und ist die Kurzversion für Knoten-Transformation.},
	type = FA
%
}
\newglossaryentry{fa:gZug}{
%
	name = {g{\"u}ltiger Zug},
	description = {Eine Transoformation, deren Ergebnis wieder ein Knoten ist.},
	type = FA
%
}
\newglossaryentry{fa:uZug}{
%
	name = {unm{\"o}glicher Zug},
	description = {Züge, welchen den Knoten zerstören würden, wären sie erlaubt. Siehe \ref{NU:GO}: Gültige und ungültige Züge.},
	type = FA
%
}
\newglossaryentry{fa:Kamera}{
%
	name = {Kamera},
	description = {Die Ansicht des Spielers auf den Knoten.},
	type = FA
%
}
\newglossaryentry{fa:Zielsystem}{
%
	name = {Zielsystem},
	description = {Systeme für die das Spiel entwikelt wurde und auf denen es ohne Probleme laufen sollte. Die Zielsysteme sind Windows 7 und Windows 8.1.},
	type = FA
%
}
\newglossaryentry{fa:EinstM}{
%
	name = {Einstellungsmen{\"u}},
	description = {In diesem Menü sind Einstellungen von Grafik und Ton möglich. Es ist über das Haupt- oder Pausemenü erreichbar.},
	type = FA
%
}
\newglossaryentry{fa:Textur}{
%
	name = {Textur},
	description = {Flächige Verbindung zwischen Kanten.},
	type = FA
%
}
\newglossaryentry{fa:Credits}{
%
	name = {Credits},
	description = {Die Nennung aller Mitwirkenden an der Entwicklung von Knot$^3$ und Informationen zum Spiel. In der grafischen Oberfläche, dort wo der Schriftzug Knot$^3$ zu sehen ist, genügt ein Klick auf denselbigen und die Credits werden angezeigt.},
	type = FA
%
}






% % % % % % % % % % % % noch einbinden
% Stimmt der Eintrag bei Rendereffekt




\newglossaryentry{fa:Rendermodus}{
%
	name = {Rendermodus},
	description = {Art und weise, wie der Knoten dargestellt wird.},
	type = FA
%
}

\newglossaryentry{fa:SpielAbbr}{
%
	name = {Spielabbruch},
	description = {Wenn der Spieler ein Spiel vorzeitig beendet. Aufruf des Pausenmenü, gefolgt von einem Klick auf "`quit"' führt zu einem Spielabbruch.},
	type = FA
%
}
\newglossaryentry{fa:MockUp}{
%
	name = {Mock-Up},
	description = {Prototypische Skizze einer grafischen Benutzeroberfläche.},
	type = FA
%
}
