% !TeX encoding = UTF-8
%
% Projekt- und Produkt-Umfang:
%


\chapter{Umfang}
\label{UF}~\\


%
% Klare Beschreibung der Ziele, welche zu erreichen sind.
%
\section{Ziele}
\label{UF:Ziele}

Das Spiel versetzt einen einzelnen Spieler in die Lage Knoten im dreidimensionalen Raum zu erstellen und zu modifizieren. Zudem wird dem Spieler erlaubt sich in verschiedenen Herausforderungen mit anderen Spielern zu messen.\\


\section{Kriterien}
% 
%
\subsection*{\underline{Pflicht:}}

\vspace{1em}

\begin{ids}{\gls{PK}}

	\id[10] Spielmodus 1: \gls{fa:Creative}.
		
	\id[20] Spielmodus 2: \gls{fa:Challenge}.
		
	\id[30] Knoten{"u}berg{"a}nge m{"u}ssen eindeutig erkennbar sein.
		
	\id[40] Darstellung mit passenden 3D-Modellen an {"U}berg{"a}ngen.
		
	\id[50] Selektion und \gls{fa:Modifikation} von Kantenz{"u}gen.
		
	\id[60] {Ü}bergehen unm{"o}glicher Zust{"a}nde, wenn m{"o}glich.
		
	\id[70] {\gls{fa:Bestenliste}} der besten Zeiten für ein Level.
		
	\id[80] Einfaches \gls{fa:Datenaustauschformat} f{"u}r die Levels, welches lokal austauschbar ist.
		
	\id[80] Mindestens zehn eindeutige \gls{fa:Level} mit steigendem \\Schwierigkeitsgrad.
		
	\id[90] Die Steuerung ist intuitiv.
		
	\id[100] Ein sinnvolles \gls{fa:Undo} unterstützt den Spieler.
		
	\id[110] Gute automatische \gls{fa:Kamera}f{"u}hrung.
		
	\id[120] Standard Sprache ist Englisch.	
		
	\id[130] Windows als Plattform muss unterst{"u}tzt werden.

\end{ids}


\subsection*{\underline{Optional:}}

\vspace{1em}

\begin{ids}{\gls{OK}}
	
	\id[10] Begleitender Sound erg{"a}nzt das Spielerlebnis.
	
	\id[20] Während des Spielens läuft Hintergrundmusik.
	
	\id[30] Eine ver{"a}nderbare Tastaturbelegung.
	
	\id[40] Einf{"a}rbung von Kanten nach Spieler Pr{"a}ferenz.
	
	\id[50] Zus{"a}tzliche Lokalisierung in Deutsch.
	
	\id[60] {\gls{fa:Redo}} welches vorangegangene Undo r{"u}ckg{"a}ngig macht.
	
	\id[70] Optionale Fl{"a}chenerstellung zwischen benachbarten \\Kanten.
	
	\id[90] Spielerbewertungen f{"u}r Knoten.
	
	\id[100] Durchschnittszeit des Bestehens einer Challenge.
	
	\id[110] {\gls{fa:Easteregg}}s k{"o}nnen gefunden werden.
	
	\id[120] Unterst{"u}tzende Tutorials die den Einstieg erleichtern.
		
	\id[130] Der Einsatz eines oder mehrerer \gls{fa:Shadereffekte}.
	
	\id[140] Der Einsatz von besonderen \gls{fa:Rendereffekte}n.
	
	\id[150] Online-Austausch der Leveldaten.
	
	\id[160] 3D-Drucker kompatible Ausgabe der Leveldaten.
	
	\id[170] Unterstützung für Linux-Plattformen.

\end{ids}


%
% Produktgrenzen starten auf einer eigenen Seite.
%
\clearpage


\section{Grenzen}
\label{UF:Grenzen}

\begin{ids}{\gls{AK}}

	\id[10] Das Spiel ist keine 3D-Modellierungssoftware.
	
	\id[20] Versionen f{"u}r mobile Ger{"a}te sind nicht geplant.
	
	\id[20] Außer Maus und Tastatur ist keine Unterst{"u}tzung durch weiter Eingabeger{"a}te,wie z.B.  ber{"u}hrungsempfindliche Bildschirme, geplant.
	
	\id[40]F{"u}rs Spielen wird keine Internetverbindung ben{"o}tigt. 
	
	\id[50] Ein Spiel beansprucht je nach Schwierigkeit einiges an Zeit und ist deswegen nicht zum Spielen f{"u}r Zwischendurch geeignet.
	
	\id[60] Das Spiel ist f{"u}r einen Spieler konzipiert.
	
\end{ids}


