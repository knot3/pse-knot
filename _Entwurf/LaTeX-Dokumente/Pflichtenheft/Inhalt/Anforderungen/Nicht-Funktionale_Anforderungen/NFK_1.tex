% !TeX encoding = UTF-8
%
% Nicht-Funktionale Anforderungen.
%

%\subsection{(Name der Nicht-Funktionalen Kategorie) ...}

%
% Hier in {} ein Bezeichner-Kürzel (K in /...K_.../) eingeben:
%
\renewcommand{\K}{}


%
% !
%
\paragraph*{\underline{Pflicht:}}~\\

\begin{ids}{\gls{PNFA\K}} 

 	\id[10] Transformierung des Knotens muss durch die Maus möglich sein.
 	
 	\id[20] Die Kamera muss mit Hilfe der Maus und der Tastatur navigierbar sein (Drehen, Zoomen und Bewegen).
 	
 	\id[30] Das Spiel sollte unter Standard-Grafikeinstellungen immer mindestens eine Bildwiederholungsrate von 30 Bildern pro Sekunde haben.
 	
 	\id[40] Grafische Gestaltung der Knoten soll die Übersicht des Spielers nicht einschränken oder verschlechtern.
 	
 	\id[50] {"U}bersichtliche Menüführung, u. A. durch den Einsatz von Alternativen zur Navigation über aufklappbare Listen.
 	
  	\id[60] Intuitive Spielsteuerung, welche schnell erlernbar ist.
  	
  	\id[70] Starten und anschließendes Beenden muss in weniger als 45 Sekunden möglich sein.
  	
  	\id[80] Speichern darf den Dialog mit dem Spieler nicht wesentlich verzögern.
  	
  	\id[90] Als Standard-Sprache für die grafische Oberfläche ist Englisch voreingestellt.
  	
  	\id[100] Strukturierte Übersicht über alle importierten Levels.
	
\end{ids}

~\\


%
% ?
%
\paragraph*{\underline{Optional:}}~\\


\begin{ids}{\gls{ONFA\K}}

 	\id[110] Erweiterbarkeit durch Einbindung von Internationalisierungen.
 	
 	\id[120] Einstellen kontrastreicher Farben für Menschen mit "`eingeschränkten Sehfähigkeiten"'.
 	
  	\id[130] Betrügereien bei den Highscores sollen automatisch erkannt/ersichtlich werden.
  	
  	\id[140] Im Tutorial werden dem Spieler die Funktionen des Spiels und die Bedienung verständlich Schritt für Schritt erklärt.  
	
\end{ids}

~\\
