% !TeX encoding = UTF-8
%
% Funktionale Anforderungen.
%


\subsection{Darstellung}

%
% Hier in {} ein Bezeichner-Kürzel (z.B.: K, -> PFAK_...) eingeben:
%
\renewcommand{\K}{}
%
% HINWEIS: Bezeichner im Glossar definieren (oder Referenz auf Glossar entfernen).
%

~\\
Alle wichtigen Informationen werden dem \gls{fa:Spieler} visuell oder akustisch dargestellt. Die Atmosphäre wird durch die musikalische Untermalung verbessert.

~\\

%
% !
%
\paragraph*{\underline{Pflicht:}}~\\

\begin{ids}{\gls{PFA\K}}

	\id[ 280] {\gls{fa:Knoten}} bestehen aus Kanten, welche durch schmale längliche Zylinder dargestellt werden.
	
	\id[ 290] Die Kanten eines \gls{fa:Knoten}s werden im dreidimensionalen Raum an \gls{fa:Rasterpunkt} ausgerichtet.
	
	\id[ 300] Bei Kreuzungen im \gls{fa:Knoten} weichen die Kanten sich gegenseitig aus, sodass der Kantenverlauf eindeutig bleibt.
	
	\id[ 310] Während des Transformieren werden die neu entstehenden Kanten transparent an der nächsten gültigen Position angezeigt. Sobald der Vorgang beendet ist wird die Kante an dieser Position ohne Transparenz dargestellt.
	
	\id[ 320] Der \gls{fa:Spieler} kann sich eine Übersicht zu allen Knoten, welche er im \gls{fa:Creative}-Mode erstellt hat anzeigen lassen, um daraus einen zur weiteren Bearbeitung auszuwählen.
	
	\id[ 330] Nach der Auswahl des \gls{fa:Challenge}-Mode kann der \gls{fa:Spieler} in einer Übersicht nach verschiedenen Kriterien ein Level auswählen.
	
	\id[ 340] Nach dem Start eines \gls{fa:Level}s sieht der Spieler beide \gls{fa:Knoten} (\gls{fa:Ausgangsknoten} und \gls{fa:Referenzknoten}). Sobald er die erste Veränderung am \gls{fa:Ausgangsknoten} vornimmt startet die Zeitmessung.
	
	\id[ 350] Ausgewählte Kanten werden visuell hervorgehoben.
	
	\id[ 355] An ausgew{"a}hlten Kanten werden Pfeile parallel zu der Richtung der Koordinatenachsen angezeigt, in welche eine gültige \gls{fa:Transformation} m{"o}glich w{"a}re.
	
\end{ids}


%
% ?
%
\paragraph*{\underline{Optional:}}~\\


\begin{ids}{\gls{OFA\K}}

	\id[ 360] Die vom Spieler ausgew{"a}hlte Musik wird im Hintergrund wiederholt abgespielt.
	
	\id[ 370] Die Levelliste kann der Spieler sortieren lassen.
	
	\id[ 380] Die Levelliste kann der Spieler filtern lassen.
	
\end{ids}


