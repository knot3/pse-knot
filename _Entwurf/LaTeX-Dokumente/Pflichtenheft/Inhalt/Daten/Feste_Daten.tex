% !TeX encoding = UTF-8
%
% Feste-Daten:
%


\subsection{Feste Daten}
\label{PD:persDat}

Daten, welche über längere Zeit zu speichern sind.\\


%
% !
%
\paragraph*{\underline{Pflicht:}}~\\


\begin{ids}{\gls{PD}}

	\id[ 10] Der Name des Spielers.
	
	\id[ 20] Eine Bestenliste pro Challenge mit:
	
		\begin{itemize}
		
			\item Spielername
			\item Zeit
			\item Bewertung
			
		\end{itemize}
	
	\id[ 30] Das Standardsprachpaket (englisch).
	
	\id[ 40] Zehn (10) Challenge-Levels steigenden Schwierigkeitsgrades.
	
	\begin{itemize}
	
		\item Jedes Level hat einen Namen.
		\item Die Knoten sind im Standardformat gespeichert.
		
	\end{itemize}
	
	\id[ 50] Die Standard-Grafikeinstellungen werden beim ersten Spielstart gespeichert.
	
	\id[ 60] Für die Steuerung sind Standardeinstellungen (Tastaturbelegung) verfügbar.
		
	\id[ 80] Das ausführbare Programm, welches auf dem jeweiligen \gls{fa:Zielsystem} lauffähig ist.
	
	\id[ 90] Alle Daten des Spiels werden in einer Ordnerstruktur auf dem Zielsystem gespeichert.
		
	\id[110] Bilder, welche in den Menüs der grafischen Oberflächen Verwendung finden.
	
\end{ids}


%
% ?
%
\paragraph*{\underline{Optional:}}~\\


\begin{ids}{\gls{OD}}

	\id[ 70] Mitgeliefert werden einige Musikstücke, die während des Spielens im Hintergrund ablaufen.

	\id[100] Mit enthalten sind Musikdateien für Hintergrundtöne bei Spezialeffekten.

	\id[200] Spielerprofile.
	
	\id[210] Deutsches Sprachpaket.
	
	\id[220] 3D-Drucker Konvertierungsmodul, um das Standard-Speicherformat von Knoten in ein für 3D-Drucker geeignetes Format zu konvertieren.
	
	\id[230] Eine Spielstatistik zeigt zu jeder bestandenen Challenge und gebautem Knoten eine Übersicht mit Informationen über den Spielverlauf an.
	
	\id[250] Eine Knot$^3$-Homepage. Mit Bereichen für:
	
	\begin{itemize}
		
		\item Informationen zum Spiel
		\item Spiel-Download
		\item Level-Downloads
		\item Support-Adressen
		\item Credits
	
	\end{itemize}
	
\end{ids}


