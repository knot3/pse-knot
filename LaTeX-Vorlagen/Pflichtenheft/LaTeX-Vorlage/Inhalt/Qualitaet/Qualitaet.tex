% !TeX encoding = UTF-8
%
% Produkt-Qualitätssicherung:
%


\chapter{Qualitätssicherung}
\label{Qualitätssicherung}


%
% Tabelle in welcher Prioritätsanforderungen zu priosisieren sind.
%
\section{Prioritäten}

\begin{center}
	
	
	\setlength{\LTleft}{-20cm plus -1fill}
	\setlength{\LTright}{\LTleft}
	
	\begin{longtable}{|l|c|c|c|c|c|}
	
	
		\hline 
		
		Bewertung & $\bigstar$ & 
		$\bigstar\bigstar$ & 
		$\bigstar\bigstar\bigstar$ & 
		$\bigstar\bigstar\bigstar\bigstar$ &
		$\bigstar\bigstar\bigstar\bigstar\bigstar$ \\
		
		\hline
		\hline
		
		Benutzerfreundlichkeit &  &  &  &  &  \\
		
		\hline
		
		Zuverlässigkeit &  &  &  &  &  \\
				
		\hline

		Robustheit &  &  &  &  &  \\
		
		\hline

		Korrektheit &  &  &  &  &  \\
		
		\hline

		Portierbarkeit &  &  &  &  &  \\
		
		\hline

		Sicherheit &  &  &  &  &  \\
		
		\hline

		Erweiterbarkeit &  &  &  &  &  \\
		
		\hline

		Effizienz &  &  &  &  &  \\
		
		\hline

		Datenschutz &  &  &  &  &  \\
		
		\hline
		
		Barrierefreiheit &  &  &  &  &  \\
				
		\hline

		Wartbarkeit &  &  &  &  &  \\
		
		\hline

		Wiederverwendbarkeit &  &  &  &  &  \\
		
		\hline

		Kompatibilität &  &  &  &  &  \\
		
		\hline

		Modularität &  &  &  &  &  \\
		
		\hline

		Verifizierbarkeit &  &  &  &  &  \\
		
		\hline

		Verständlichkeit &  &  &  &  &  \\
		
		\hline

		Konfigurierbarkeit &  &  &  &  &  \\
		
		\hline
	
	\end{longtable}

\end{center}


\section{Testfälle}

\subsection*{Pflicht-Testfälle}

...
\\


\subsection*{Optionale Testfälle}

...
\\
