\chapter{Hinzugefügte Klassen}
Dieses Kapitel enthält eine Auflistung der Klassen, welche während der Implementierungsphase hinzugefügt wurden und sich erst dann als notwendig erwiesen haben.

\section{Core:}
\subsection{FloatOptionInfo}
Grund: Analog zu BooleanOptionInfo brauchten wir auch eine Möglichkeit, Gleitkommazahlen in der Einstellungsdatei abzuspeichern.
\subsection{KeyOptionInfo}
Grund: Analog zu FloatOptionInfo brauchten wir für die Konfigurierbarkeit der Tastengelegungen auch eine Möglichkeit, Tasten in der Einstellungsdatei abzuspeichern, repräsentiert durch das "Keys"-Enum von XNA.
\subsection{IGameScreen}
Grund: Um ein Mock-Objekt für GameScreen erstellen zu können, werden alle von außen verwendeten Eigenschaften und Methoden von GameScreen in einem Interface zusammengefasst und überall im Code wird IGameScreen verwendet. Dies erleichtert die Unit-Tests.
\subsection{IMouseClickEventListener, IMouseMoveEventListener und IMouseScrollEventListener}
Grund: Die in IMouseEventListener definierten Methoden wurden in drei Interfaces aufgeteilt, damit nicht jede Klasse,
die Teile der Funktionalität benötigt, alle Methoden überschreiben muss und die dann teilweise leer sind.

\section{GameObjects:}
\subsection{EdgeColoring}
Grund: War im Entwurf nicht spezifiziert, da die Kolorierung als optional angesehen wurde. Ein einfacher Inputhandler analog zu EdgeRectangles, der auf Tastendruck einen ColorPickDialog für die selektierten Kanten öffnet.
\subsection{EdgeRectangles}
Grund: War im Entwurf nicht spezifiziert, da die Berechteckung als optional angesehen wurde. Ein einfacher Inputhandler analog zu EdgeColoring, der auf Tastendruck den selektierten Kanten eine gemeinsame neue Flächen-ID zuweist.
\subsection{ModelColoring}
Grund: Alle GameModels halten jetzt nicht mehr nur ein Color-Objekt, das ihre Farbe beinhaltet, sondern eine Instanz von ModelColoring. Dies ist eine abstrakte Klasse, die die Farbe von Modellen kapselt. Konkrete Implementierungen dieser Klasse sind SingleColor für eine einheitliche Farbe und GradientColor für einen Farbverlauf zwischen zwei Farben. Derzeit können wir aufgrund der Eingeschränktheit der XNA-API noch keinen Farbverlauf auf Modellen darstellen, aber sobald unser diesnezüglicher Erkenntnisfindungsprozess abgeschlossen ist, haben wir dann bereits die benötigte Infrastruktur im Code, um Modelle mit verschiedenen Farbeffekten zu versehen. Zusätzlich wird in ModelColoring auch die Hervorhebungsfarbe und dessen Intensität gekapselt und es gibt verschiedene Properties, um entweder die einzelnen Farbparameter abzufragen oder für Shadereffekte, die keine Farbverläufe unterstützen, eine Mischfarbe zu berechnen.
\subsection{SkyCube}
Grund: Ein IGameObject, das den weltraumartigen Hintergrund in unserem aktuellen Design darstellt.
\subsection{TexturedRectangle}
Grund: Ein texturiertes, frei im Raum positionierbares Rechteck, das aus zwei Dreiecken zusammengesetzt ist und low level gezeichnet wird.
\subsection{TexturedRectangleInfo}
Grund: Das Metadaten-Objekt zu TexturedRectangle.

\section{RenderEffects:}
\subsection{OpaqueEffekt}
Grund: Ein zusätzlicher RenderEffekt, welcher im Entwurf nicht berücksichtigt wurde.
\subsection{RenderEffectLibrary}
Grund: Sammelt alle aktuell im Spiel verwendbaren RenderEffekte und bietet Schnittstellen an, um den aktuell vom Spieler im Optionsmenü ausgewählten Rendereffekt zu instanziieren.
\subsection{IRenderEffectStack}
Grund: Ein Interface, das benötigt wird, damit wir ein gemeinsames Interface für den echten RenderEffectStack und dessen Mock-Objekt bei den Unit-Tests haben.

\section{KnotData:}
\subsection{DirectionHelper}
Grund: Enthält Methoden rund um Direction.
\subsection{RectangleMap}
Grund: Analog zu NodeMap stellt RectangleMap eine Zuordnung zwischen den Kanten und den Flächen zwischen ihnen dar. Die der RectangleMap übergebenen Kanten enthalten je keine, eine oder mehrere Flächen-IDs, aus denen dann in dieser Klasse die Positionen und Ausmaße der Flächen berechnet werden. 

\section{Widgets:}
\subsection{Border}
Grund: Zeichnet einen einfarbigen Rahmen mit einer definierbaren Breite um ein Widget.
\subsection{Bounds}
Grund: Eine verbesserte Neuimplementierung der Rectangle-Klasse aus XNA, die auf ScreenPoint-Objekten basiert und erweiterte Funktionalität bietet, darunter die prozentuale Skalierung auf die aktuelle Bildschirmgröße.
\subsection{ChallengePauseDialog}
Grund: Vorher war nur ein Pause-Dialog vorgesehen. Es hat sich herausgestellt, dass wir für die beiden Modi unterschiedliche Funktionalität benötigen. Im Fall des Challenge-Modus muss der Pause-Dialog die Zeit anhalten.
\subsection{ColorPickDialog}
Grund: Ein Dialog, der ein ColorPickItem enthält und zum Colorpicken aufgerufen wird.
\subsection{Container}
Grund: Container wurde in Menu umbenannt und VerticalMenu in Menu, weil dies die angedachte Funktionalität besser beschreibt.
\subsection{CreativePauseDialog}
Grund: Vorher war nur ein Pause-Dialog vorgesehen. Es hat sich herausgestellt, dass wir für die beiden Modi unterschiedliche Funktionalität benötigen.
\subsection{ErrorDialog}
Grund: Ein Dialog der Fehlermeldungen anzeigt.
\subsection{Lines}
Grund: Ein Widget, das die Linien zeichnet, die im HfG-Design vorgegeben waren und so aussehen wie im Mockup.
\subsection{MenuEntry}
Grund: Umbenannter MenuButton.
\subsection{ScreenPoint}
Grund: Eine verbesserte Neuimplementierung sowohl der Klassen Vector2 als auch Point aus dem XNA-Framework, die sich implizit in beide konvertieren lässt und an deren Stelle in der XNA-API verwenden lässt. Die X- und Y-Korrdinaten von ScreenPoint sind als Delegate implementiert, was eine verschachtelte Abhängigkeit zwischen mehreren ScreenPoints ermöglicht. Jeder ScreenPoint hält eine Referenz auf den IGameScreen, in dem er verwendet wird, und skaliert beim Casten in Vector2 oder Point auf den aktuell im GameScreen verwendeten Viewport. Diese Implementierung ist von der erweiterten Monogame-API inspiriert und behebt sowohl die Schwäche, dass zwischen zwei Points des XNA-Frameworks keine Rechenoperationen ausgeführt werden können als auch das Problem, dass beide keine Delegate-Werte unterstützen und damit keine Abhängigkeiten und komplexeren automatischen Berechnungen durchgeführt werden können.
\subsection{State}
Grund: Umbenannter ItemState, da jetzt alle Widgets (nicht nur MenuItems) Selected oder Hovered sein können.
\subsection{TextItem}
Grund: Wir haben festgestellt, dass wir keine Möglichkeit hatten reine Textkomponenten für Dialoge oder Menüs zu nutzen. Wir hätten die Klasse des Buttons verwenden können, aber dies wäre nicht die korrekte Verwendung der Klasse im Sinne des Entwurfs. Aus diesem Grund haben wir eine Klasse für reine Textdarstellung eingefügt.

\section{Audio:}
Im Entwurf wurde das Thema Audio nicht berücksichtigt und aus diesem Grund wurden folgende Klassen hinzugefügt.
\subsection{AudioManager}
Grund: Verwaltet alle Audio-Dateien die im Spiel verwendet werden. 
\subsection{IAudioFile}
Grund:
\subsection{IPlaylist}
Grund:
\subsection{LoopPlaylist}
Grund:
\subsection{OggVorbisFile}
Grund:
\subsection{Sound}
Grund:
\subsection{SoundEffectFile}
Grund:

\section{Utilities:}
\subsection{BoundingCylinder}
Grund:
\subsection{ColorHelper}
Grund:
\subsection{DictionaryHelper}
Grund:
\subsection{EnumHelper}
Grund:
\subsection{FileIndex}
Grund:
\subsection{FileUtility}
Grund:
\subsection{FrustumHelper}
Grund:
\subsection{HfGDesign}
Grund:
\subsection{IniFile}
Grund:
\subsection{InputHelper}
Grund:
\subsection{ModelHelper}
Grund:
\subsection{MonoHelperMG}
Grund:
\subsection{MonoHelperXNA}
Grund:
\subsection{RayExtensions}
Grund:
\subsection{SavegameLoader}
Grund:
\subsection{ShaderHelper}
Grund:
\subsection{TextHelper}
Grund:
\subsection{TextureHelper}
Grund:
\subsection{VectorHelper}
Grund:



