\chapter{Hinzugefügte Klassen}
Dieses Kapitel enthält eine Auflistung der Klassen, welche während der Implementierungsphase hinzugefügt wurden und sich erst dann als notwendig erwiesen haben.

\section{Core:}
\subsection{FloatOptionInfo}
Grund: Analog zu BooleanOptionInfo brauchten wir auch eine Möglichkeit, Gleitkommazahlen in der Einstellungsdatei abzuspeichern.
\subsection{KeyOptionInfo}
Grund: Analog zu FloatOptionInfo brauchten wir für die Konfigurierbarkeit der Tastengelegungen auch eine Möglichkeit, Tasten in der Einstellungsdatei abzuspeichern, repräsentiert durch das "Keys"-Enum von XNA.
\subsection{IGameScreen}
Grund: Um ein Mock-Objekt für GameScreen erstellen zu können, werden alle von außen verwendeten Eigenschaften und Methoden von GameScreen in einem Interface zusammengefasst und überall im Code wird IGameScreen verwendet. Dies erleichtert die Unit-Tests.
\subsection{IMouseClickEventListener, IMouseMoveEventListener und IMouseScrollEventListener}
Grund:

\section{GameObjects:}
\subsection{EdgeColoring}
Grund:
\subsection{EdgeRectangles}
Grund:
\subsection{GameObjectInfo}
Grund:
\subsection{ModelColoring}
Grund:
\subsection{SkyCube}
Grund:
\subsection{TexturedRectangle}
Grund:
\subsection{TexturedRectangleInfo}
Grund:

\section{RenderEffects:}
\subsection{OpaqueEffekt}
Grund:
\subsection{RenderEffectLibrary}
Grund:
\subsection{IRenderEffectStack}
Grund:
\subsection{ResizeEffect}
Grund:

\section{KnotData:}
\subsection{DirectionHelper}
Grund:
\subsection{RectangleMap}
Grund:

\section{Widgets:}
Grund:
\subsection{Border}
Grund:
\subsection{Bounds}
Grund:
\subsection{ChallengePauseDialog}
Grund: Vorher war nur ein Pause-Dialog vorgesehen. Es hat sich herausgestellt, dass wir für die beiden Modi unterschiedliche Funktionalität benötigen. Im Fall des Challenge-Modus muss der Pause-Dialog die Zeit anhalten.
\subsection{ColorPickDialog}
Grund:
\subsection{Container}
Grund:
\subsection{CreativePauseDialog}
Grund: Vorher war nur ein Pause-Dialog vorgesehen. Es hat sich herausgestellt, dass wir für die beiden Modi unterschiedliche Funktionalität benötigen.
\subsection{ErrorDialog}
Grund:
\subsection{Lines}
Grund:
\subsection{MenuEntry}
Grund:
\subsection{Screenpoint}
Grund:
\subsection{State}
Grund:
\subsection{TextItem}
Grund: Wir haben festgestellt, dass wir keine Möglichkeit hatten reine Textkomponenten für Dialoge oder Menüs zu nutzen. Wir hätten die Klasse des Buttons verwenden können, aber dies wäre nicht die korrekte Verwendung der Klasse im Sinne des Entwurfs. Aus diesem Grund haben wir eine Klasse für reine Textdarstellung eingefügt.

\section{Audio:}
\subsection{AudioManager}
Grund:
\subsection{IAudioFile}
Grund:
\subsection{IPlaylist}
Grund:
\subsection{LoopPlaylist}
Grund:
\subsection{OggVorbisFile}
Grund:
\subsection{Sound}
Grund:
\subsection{SoundEffectFile}
Grund:

\section{Utilities:}
\subsection{BoundingCylinder}
Grund:
\subsection{ColorHelper}
Grund:
\subsection{DictionaryHelper}
Grund:
\subsection{EnumHelper}
Grund:
\subsection{FileIndex}
Grund:
\subsection{FileUtility}
Grund:
\subsection{FrustumHelper}
Grund:
\subsection{HfGDesign}
Grund:
\subsection{IniFile}
Grund:
\subsection{InputHelper}
Grund:
\subsection{ModelHelper}
Grund:
\subsection{MonoHelperMG}
Grund:
\subsection{MonoHelperXNA}
Grund:
\subsection{RayExtensions}
Grund:
\subsection{SavegameLoader}
Grund:
\subsection{ShaderHelper}
Grund:
\subsection{TextHelper}
Grund:
\subsection{TextureHelper}
Grund:
\subsection{VectorHelper}
Grund:



