\chapter{Bericht}
Während den Ferien und zu Beginn der Implementierung wurde von Tobias alles, was er schon wärend der anderen Phasen für den Prototypen entwickelt hat auf unseren Entwurf angepasst und implementiert.
Dadurch hatten wir früh eine umfangreiche Basis auf die wir aufbauen konnten, dies umfasste die grundlegende Klassenstruktur, die grundlegende Datenstruktur, die meisten Screens und Widgets, die 3D-Komponenten und -Objekte und den Standard-Rendereffekt.
Die Gruppenarbeit begann dann Mitte Januar.
Als nächstes wurde die Datenstruktur vervollständigt und noch fehlende Widgets wie z.B. die SliderItems hinzugefügt.
Dabei stießen wir bei der Datenstruktur auf keine unerwarteten Probleme, nur auf die Erwarteten.
Das waren die Gültigkeitsprüfung eines Zuges und der Vergleich zweier Knoten.
Bei der Gültigkeitsprüfung des Zuges ergab sich, das es die beste Variante ist den Zug intern ohne Änderung an der Datenstruktur auszuführen und die entstandene Strktur auf Gültigkeit zu prüfen.
Beim Vergleich zweier Knoten hat der direkte Zugriff auf die Datenstruktur und die Art der Struktur den erwünschten erleichternden Effekt gehabt. Da jedes Element des doppelt verketteten Kreises als Startpunkt dienen konnte, wurde zum Vergleich einfach ein neuer, eindeutiger Startpunkt gewählt.
Bei den Widgets hingegen sind einige Varianten aufgetaucht die wir im Entwurf noch nicht bedacht hatten. Zum Beispiel gab es keine gute Variante reinen Text anzuzeigen, der nicht auf Eingaben regaieren sollte.
