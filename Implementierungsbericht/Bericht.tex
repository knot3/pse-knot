\chapter{Bericht}
Während den Ferien und zu Beginn der Implementierung wurde von Tobias viel der Arbeit, die er während der anderen Phasen in den Prototypen investiert hat, auf unseren Entwurf angepasst und implementiert.
\\
Dadurch hatten wir früh eine umfangreiche Basis, auf die wir aufbauen konnten, dies umfasste die grundlegende Klassenstruktur, die grundlegende Datenstruktur, die meisten Screens und Widgets, die 3-D-Komponenten und -Objekte und den Standard-Rendereffekt.
\\\\
Die Gruppenarbeit begann dann Mitte Januar.
Als Nächstes wurde die Datenstruktur vervollständigt und noch fehlende Widgets wie z.B. die SliderItems hinzugefügt.
Dabei stießen wir bei der Datenstruktur auf keine unerwarteten Probleme, nur auf die Erwarteten.
Das waren die Gültigkeitsprüfung eines Zuges und der Vergleich zweier Knoten.
\\\\
Bei der Gültigkeitsprüfung des Zuges ergab sich, das es die beste Variante ist den Zug intern ohne Änderung an der Datenstruktur auszuführen und die entstandene Struktur auf Gültigkeit zu prüfen.
\\
Beim Vergleich zweier Knoten hat der direkte Zugriff auf die Datenstruktur und die Art der Struktur den erwünschten erleichternden Effekt gehabt. Da jedes Element des doppelt verketteten Kreises als Startpunkt dienen konnte, wurde zum Vergleich einfach ein neuer, eindeutiger Startpunkt gewählt.
\\\\
Bei den Widgets hingegen sind einige Varianten aufgetaucht, die wir im Entwurf noch nicht bedacht hatten. Zum Beispiel gab es keine gute Variante reinen Text anzuzeigen, der nicht auf Eingaben reagieren sollte.
\\\\


Außerdem mussten wir feststellen, dass wir für das Pausenmenü im Creative und im Challenge Mode doch zwei unterschiedliche Widgets brauchen, da sie sich intern stärker unterscheiden als erwartet. Dazu kamen noch verschieden Helfer, die einem die Arbeit nur sehr erleichtert haben wie Border, das die Umrahmung von Widgets einfach ermöglicht.
Während dieser Zeit haben sich andere aus unserem Team mit der Auswahl der Musik beschäftigt.
Der schnelle Fortschritt, den wir durch Tobias Arbeiten in den anderen Phasen und in den Ferien hatten, führte in der Mitte der Implementierungsphase zu einer stark erhöhten Bereitschaft viel der Arbeitszeit mit konzeptionellen Ideen zur Implementierung und mit Recherchen zu Musik, Grafiken und 3D-Modellen zu verbringen.
Dies änderte sich erst als uns die Zeit längst eingeholt hatte.
Bei der Audio-API sind wir auf das Problem gestoßen, dass XNA nur den proprietären WMA-Codec oder reine WAV-Dateien unterstützt. MonoGame unterstützt allerdings auf Plattformen wie Linux aus lizenzrechtlichen Gründen keine WMA-Dateien, sondern nur WAV-Dateien. Da wir nicht vollständig darauf verzichten wollten, komprimierte Audiodateien verwenden zu können, haben wir eine modifizierte Version der Library OggSharp \footnote{https://github.com/tobiasschulz/oggsharp} eingebunden, um den freien Ogg-Vorbis-Codec verwenden zu können, der sich auch allein aus lizenzrechtlichen Gründen besser eignet als WMA.
Nun wurden auch endlich Langzeitbaustellen angegangen, wie zum Beispiel die Kantenübergänge.
Dies stellte sich als sehr aufwendig heraus, da keine Möglichkeit gefunden wurde die Ausrichtung der Übergänge Algorythmisch zu ermitteln, sondern jeder mögliche Übergang einzeln von Hand bearbeitet werden mussten.
Neben vielen kleinen Änderungen, die das Spiel vor allem optisch ansprechender machen sollte, wurde am Schluss noch mit den Spiel internen Mitteln die Challenges erstellt. Was gleichzeitig auch einen recht umfangreichen Funktionstest beinhaltete und zu einigen Äderungen führte, zum Beispiel war vorher nie aufgefallen, dass das Anklicken der Kanten um die Breite des Fensterrahmens versetzt war.
