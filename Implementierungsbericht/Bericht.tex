\chapter{Bericht}
Während den Ferien und zu beginn der Implementierung wurde von Tobias alles,
was er schon wärend der anderen Phasen für den Prototypen entwickelt hat
auf unseren Entwurf angepasst und Implementiert.
Dadurch hatten wir früh eine umfangreiche Basis auf die wir aufbauen konnten.
Das umfasste die grundlegende Datenstruktur sowie viele Screens und Widgets.
Als nächstes wurde die Datenstruktur verfolständigt und noch fehlende Widgets hinzugefügt.
Dabei stießen wir bei der Datenstruktur auf kein unerwarteten Probleme, nur auf die erwarteten.
Das waren die Gültigkeitsprüfung eines Zuges und der Vergleich zweier Knoten.
Bei der Gültigkeitsprüfung des Zuges ergab sich, das es die beste Variante ist den Zug intern ohne Änderung an der Datenstruktur auszuführen und die entstandene Strktur auf gültigkeit zu prüfen.
Beim Vergleich zweier Knoten hat der dierekte Zugriff auf die Datenstruktur und die Art der Struktur den erwünschten erleichternden Effekt gehabt. Dadurch das jedes Element des doppelt verketteten Kreises als Startpunkt dienen konnte, wurde zum vergleich einfach ein neuer, eindeutiger Startpunkt gewählt.
Bei den Widgets hingegen sind einige Varianten aufgetaucht die wir im Entwurf noch nicht bedacht hatten. Zum Beispiel gab es keine gute Variante reinen Text anzuzeigen.
