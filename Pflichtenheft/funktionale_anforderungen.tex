\chapter{Funktionale Anforderungen}



\begin{tabular}{|p{0.125\textwidth}|p{0.875\textwidth}|}
\hline 
\textbackslash F 110  \textbackslash & Speicherung einer Bestenliste für die Levels \\ 
\hline 
\textbackslash F 120  \textbackslash  & Import und Export des Austauschdatei-Formates \\ 
\hline 
\textbackslash F 130  \textbackslash  & Strukturierte Übersicht über alles importierten Levels \\ 
\hline
\textbackslash F 140  \textbackslash  & Wechsel zwischen verschiedenen Kameraeinstellungen (Geführte, zentrierte oder frei-bewegliche Kamera)  \\ 
\hline
\textbackslash F 150  \textbackslash  & Setzen von neuen Ankerpunkten an Kanten \\ 
\hline
\textbackslash F 160  \textbackslash  & Wechsel zwischen verschiedenen Kameraeinstellungen (Geführte, zentrierte oder frei-bewegliche Kamera)  \\ 
\hline
\textbackslash F 170  \textbackslash  & Durch Tastendruck (ESC?) ist das Pause-Menü erreichbar\\ 
\hline
\textbackslash F 180  \textbackslash  & Standard Grafikeinstellungen werden vom Programm vorgegeben oder bestimmt.\\ 
\hline
\textbackslash F 190  \textbackslash  & Im Tutorial werden über Textausgaben und grafische Visualisierungen dem Spieler schrittweise die einzelnen Bedienungsmöglichkeiten beigebracht  \\
\hline
\textbackslash F 200  \textbackslash  & Der Spieler kann Einstellungen zur Grafik und dem Ton im Menüpunkt Einstellungen des Hauptmenüs bzw. Pause-Menü vornehmen\\
\hline
\textbackslash F 210  \textbackslash  & Beim Starten des Frei-Bau-Modus (Sandkasten-Modus) wird dem Spieler ein einfacher Knoten (4 Kanten, 4 Ecken) zum Transfomieren bereitgestellt \\
\hline
\textbackslash F 220  \textbackslash  & Beim Verlassen des aktuellen Spiels (Frei-Bau-Modus) über das Pause-Menü wird nachgefragt ob der aktuelle Knoten gespeichert werden soll, ohne Speicher der Modus verlassen werden soll oder ob dieser Vorgang abgebrochen werden soll\\
\hline
\textbackslash F 230  \textbackslash  & Der Spieler kann im Frei-Bau-Modus (Sandkasten-Modus) aus zwei erstellten Knoten eine Level für den Challenge-Modus erstellen.\\
\hline
\textbackslash F 240  \textbackslash  & Nach der Auswahl des Challenge-Modus kann der Spieler über eine Übersicht ein Level auswählen anhand von Schwierigkeitsgrad, Bewertungen oder Bestzeit \\
\hline
\textbackslash F 250  \textbackslash  & Nach dem Start eines Levels sieht der Spieler beide Knoten (Ausgangsknoten und Referenzknoten) und kann die Ansicht beliebig verändern. Sobald er die erste Veränderung am Ausgangsknoten vornimmt startet die Zeitmessung\\
\hline
\textbackslash F 260  \textbackslash  & Spiel prüft den transformierten Ausgangsknoten auf Gleichheit mit dem Referenzknoten. Falls Gleichheit besteht wird die Zeit angehalten und der Abschlussbildschirm wird eingeblendet \\
\hline
\textbackslash F 270  \textbackslash  & Das Spiel speichert die Platzierung des Spielers in der Bestenliste und die Bewertung des Levels\\
\hline
\end{tabular} 
	
