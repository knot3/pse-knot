\chapter{Funktionale Anforderungen}

% TODO: Feingliederung der Funktionalen Anforderungen (Kategorien für z.B. f. Dinge die während eines Spiels passieren, Einstellungen, ...)

\renewcommand{\theenumi}{/F\_\arabic{enumi}0/}
\renewcommand{\labelenumi}{\theenumi}

\begin{enumerate}

\item Speicherung einer Bestenliste für die Levels
\item Import und Export des Austauschdatei-Formates 
\item Strukturierte Übersicht über alles importierten Levels 
\item Wechsel zwischen verschiedenen Kameraeinstellungen (Geführte, zentrierte oder frei-bewegliche Kamera)  
\item Setzen von neuen Ankerpunkten an Kanten 
\item Durch Tastendruck (ESC?) ist das Pause-Menü erreichbar 
\item Standard Grafikeinstellungen werden vom Programm vorgegeben oder bestimmt. 
\item Im Tutorial werden über Textausgaben und grafische Visualisierungen dem Spieler schrittweise die einzelnen Bedienungsmöglichkeiten beigebracht 
\item Der Spieler kann Einstellungen zur Grafik und dem Ton im Menüpunkt Einstellungen des Hauptmenüs bzw. Pause-Menü vornehmen
\item Beim Starten des Frei-Bau-Modus (Sandkasten-Modus) wird dem Spieler ein einfacher Knoten (4 Kanten, 4 Ecken, Basisquadrat?) zum Transfomieren bereitgestellt
\item Beim Verlassen des aktuellen Spiels (Frei-Bau-Modus) über das Pause-Menü wird nachgefragt ob der aktuelle Knoten gespeichert werden soll, ohne Speicher der Modus verlassen werden soll oder ob dieser Vorgang abgebrochen werden soll % TODO: verbessern.
\item Der Spieler kann im Frei-Bau-Modus (Sandkasten-Modus) aus zwei erstellten Knoten eine Level für den Challenge-Modus erstellen.
\item Nach der Auswahl des Challenge-Modus kann der Spieler über eine Übersicht ein Level auswählen anhand von Schwierigkeitsgrad, Bewertungen oder Bestzeit
\item Nach dem Start eines Levels sieht der Spieler beide Knoten (Ausgangsknoten und Referenzknoten) und kann die Ansicht beliebig verändern. Sobald er die erste Veränderung am Ausgangsknoten vornimmt startet die Zeitmessung
\item Spiel prüft den transformierten Ausgangsknoten auf Gleichheit mit dem Referenzknoten. Falls Gleichheit besteht wird die Zeit angehalten und der Abschlussbildschirm wird eingeblendet
\item Das Spiel speichert die Platzierung des Spielers in der Bestenliste und die Bewertung des Levels
\item Spieler können Namen/Nicknames eingeben, welche gespeichert werden.

\end{enumerate}

	
