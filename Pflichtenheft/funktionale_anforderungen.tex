\chapter{Funktionale Anforderungen}

% TODO: Feingliederung der Funktionalen Anforderungen (Kategorien für z.B. f. Dinge die während eines Spiels passieren, Einstellungen, ...)

\renewcommand{\theenumi}{/F\_\arabic{enumi}0/}
\renewcommand{\labelenumi}{\theenumi}



\section{Konfiguration}
Der Spieler kann verschiedene Eigenschaften des Programms einsehen und an seine Vorlieben anpassen.
\begin{enumerate}
 
\item Der Spieler kann Einstellungen zur Grafik und dem Ton im Menüpunkt Einstellungen des Hauptmenüs bzw. Pause-Menü vornehmen.
\item Standard Grafikeinstellungen werden vom Programm vorgegeben.
\item In den Einstellungen kann der Spieler die Tastaturbelegung einsehen und  ändern. % opt.
\item Wechsel zwischen verschiedenen Kameraeinstellungen (Geführte oder frei-bewegliche Kamera).  % opt.
\item Durch Tastendruck ist das Pause-Menü während des laufenden Spiels erreichbar.
\item Die Farben zum Einfärben von Knoten kann der Spieler selbständig festlegen. Die Anzahl ist aber beschränkt. % opt.
\item Der Spieler kann die Sprache der grafischen Oberfläche des Spiels einsehen und ändern. % opt.
\item Der Spieler kann seinen Spielernamen ändern.

\end{enumerate}

\section{Spielfunktionen}
Der Spieler kann durch verschiedene Funktionen mit dem Spiel interagieren. Er kann zum Beispiel die Kamera drehen und den Knoten verformen. 

\begin{enumerate}[resume]


\item Beim Starten des Creative-Modus wird dem Spieler ein einfacher Knoten zum Transfomieren bereitgestellt.
\item Der Spieler kann im Creative-Modus aus zwei erstellten Knoten eine Level für den Challenge-Modus erstellen.
=======
\item Beim Starten des Creative-Modus wird dem Spieler ein einfacher Knoten  zum Transfomieren bereitgestellt.
\item Der Spieler kann im Creative-Modus aus zwei erstellten Knoten ein Level für den Challenge-Modus erstellen.

\item Die Kanten des Knotens können vom Spieler vollständig oder teilweise ausgewählt werden.
\item Ausgewählte Kanten kann der Spieler in die Richtung der Koordinatenachsen transformieren.
\item Das Programm überprüft, ob eine Transformation gültig ist, falls nicht wird diese nicht ausgeführt.
\item Wenn der Spieler auf den Undo-Button klickt wird seine letzte Transformation rückgängig gemacht (beliebig wiederholbar). 
\item Wenn der Spieler die Undo-Funktion genutzt hat, kann er seine letzten Undo-Aktionen durch Klicks auf den Redo-Button schrittweise rückgängig machen. Redo funktioniert nur so lange der Spieler keine Veränderung am Knoten vorgenommen hat. % opt.
\item Im Challenge-Modus prüft das Programm den transformierten Ausgangsknoten auf Gleichheit mit dem Referenzknoten. Falls Gleichheit besteht wird die Zeit angehalten und der Abschlussbildschirm wird eingeblendet.
\item Kanten können vom Spieler eingefärbt werden. % opt.
\item Der Spieler kann im Creative-Modus vier Kanten auswählen, zwischen denen eine Fläche erstellt wird, sofern diese Kanten ein Rechteck bilden. %opt.
\item Falls der Spieler nur drei Kanten für eine Fläche auswählt, wird  die fehlende Kante durch eine \textquotedblleft virtuelle Kante\textquotedblright~ersetzt.
\item Der Spieler kann das Spiel jederzeit beenden.
\item Nach erfolgreichem Beenden einer Challenge kann der Spieler die Challenge neu starten. % opt.
\item Von einem Knoten kann der Spieler ein Bild erzeugen und abspeichern. %opt
\item Beim Erzeugen eines Bildes kann der Spieler verschiedene Render-Modi auswählen.
\item Ein erstellter Knoten kann in ein Format für 3D-Drucker exportiert werden. %opt
\item Der Spieler kann Eastereggs finden. %opt

\end{enumerate}

\section{Darstellung}
Alle wichtigen Informationen werden dem Spieler visuell oder akustisch dargestellt. Die Atmosphäre wird durch die musikalische Untermalung verbessert.

\begin{enumerate}[resume]
\item Knoten bestehen aus Kanten, welche durch schmale längliche Zylnder dargestellt werden.
\item Die Kanten eines Knotens werden im dreidimensionalen Raum an Rasterpunkten ausgerichtet.
\item Bei Kreuzungen im Knoten weichen die Kanten sich gegenseitig aus, sodass der Kantenverlauf eindeutig bleibt.
\item Während des Transformieren wird die neu entstehenden Kanten transparent an der nächsten gültigen Position angezeigt. Sobald der Vorgang beendet ist wird die Kante an dieser Position ohne Transparenz dargestellt..
\item Der Spieler kann sich eine Übersicht zu allen Knoten, welche er im Creative-Modus erstellt hat anzeigen lassen, um daraus einen zur weiteren Bearbeitung auszuwählen.
\item Strukturierte Übersicht über alle importierten Levels. % nicht funktional
\item Nach der Auswahl des Challenge-Modus kann der Spieler in einer Übersicht nach verschiedenen Kriterien ein Level auswählen.
\item Nach dem Start eines Levels sieht der Spieler beide Knoten (Ausgangsknoten und Referenzknoten). Sobald er die erste Veränderung am Ausgangsknoten vornimmt startet die Zeitmessung.
\item Ausgewählte Kanten werden visuell hervorgehoben.
\item Die vom Spieler ausgew{"a}hlte Musik wird im Hintergrund wiederholt abgespielt. % opt.
\item Die Levelliste kann der Spieler sortieren lassen. % opt.
\item Die Levelliste kann der Spieler filtern lassen. % opt.

\end{enumerate}
\section{Datenverwaltung}
Grundlegende Inhalte des Spieles werden abgespeichert und verwaltet.
Diese Inhalte können auch zwischen verschiedenen Systemen ausgetauscht werden. 

\begin{enumerate}[resume]

\item Der Spieler kann den Knoten im Creative-Modus abspeichern.
\item Wenn ein Knoten abgespeichert wird, kann der Spieler ein Bild von seinem Knoten erstellen, welches als Vorschaubild verwendet wird. %opt
\item Speicherung einer Bestenliste für jedes Level.
\item Import und Export von Knoten und Challenges mit Hilfe eines Austauschdatei-Formates.
\item Beim Verlassen des Creative-Modus über das Pause-Menü  kann der Spieler auswählen ob er den aktuellen Knoten speichern möchte oder ohne Speichern den Modus verlassen will. % opt
\item Der Spieler kann einen Spielernamen eingeben, welcher gespeichert wird.
\item Das Spiel speichert die Platzierung des Spielers  für das Level in einer Bestenliste unter dessen Spielernamen.
\item Das Spiel speichert Spieler-Bewertungen des Levels. % opt.
\item Die Durchschnittszeit beim Bestehen einer Challenge wird automatisch mitgespeichert. %opt.
\item Das Importieren ungültiger Knoten ist nicht möglich.


\end{enumerate}

	
