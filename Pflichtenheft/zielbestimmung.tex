\chapter{Zielbestimmung}

Das Spiel versetzt einen einzelnen Spieler in die Lage Knoten im dreidimensionalen Raum zu erstellen und zu modifizieren. Zwischen den Kanten der Knoten besteht die Möglichkeit Flächen einzusetzen und diese zu texturieren. Zudem wird dem Spieler erlaubt sich in verschiedenen Herausforderungen mit anderen Spielern zu messen.\\


\section{Musskriterien}

\begin{itemize}

	\item Spielmodus 1 Freies Erstellen
	
	\item Spielmodus 2 Challenges
	
	\item Knotenübergänge müssen eindeutig erkennbar sein.
	
	\item Darstellung mit passenden 3D-Modellen an Übergängen.
	
	\item Selektion und Modifikation von Kantenzügen.
	
	\item Übergehen unmöglicher Zustände, wenn möglich.
	
	\item Highscores: Heuristik zur Komplexität / Eindeutigkeit.
	
	\item einfaches Datenaustauschformat für die Levels
	
	\item mindestens zehn eindeutige Level mit steigendem Schwierigkeitsgrad.
	
	\item intuitive Steuerung
	
	\item sinnvolles Undo
	
	\item gute automatische Kameraführung
	\item Standard Sprache ist Englisch
	

	\item  Einfaches Speicherformat das lokal Austauschbar ist
	
	\item Windows als Plattform muss unterstützt werden
	
\end{itemize}

\section{Kannkriterien}


\begin{itemize}
\item Begleitender Sound ergänzt das Spielerlebnis

\item Der Einsatz von Hintergrundmusik

\item Eine veränderbare Tastaturbelegung

\item Einfärbung von Kanten nach Spieler Präferenz

\item zusätzliche Lokalisierung in Deutsch

\item Redo welches vorangegangene Undo rückgängig macht

\item optionale Flächenerstellung zwischen benachbarten Kanten

\item nach Challenge Beendigung sofortiges Wiederholen möglich

\item Spielerbewertungen für Knoten

\item Durchschnittszeit des Bestehens einer Challenge

\item Eastereggs können gefunden werden

\item Unterstützende Tutorials die den Einstieg erleichtern
	
\item Der Einsatz eines oder mehrerer Shadereffekte

\item Der Einsatz von besonderen Rendereffekten 

\item Online-Austausch der Leveldaten

\item 3D-Drucker kompatible Ausgabe der Leveldaten

\item Linux als Plattform wird unterstützt
\end{itemize}

\section{Abgrenzungskriterien}


\begin{itemize}



\item Das Spiel ist kein 3D-Modelierungsprogramm

\item Das Spiel benötigt keine Internetverbindung zum Spielen
\item Das Spiel ist nicht für mobile Geräte oder die Bedienung durch Berührungsempfindliche Bildschirme gedacht
\item Das Spiel kann je nach Schwierigkeit einiges an Zeit beanspruchen und ist deswegen nicht für das Spielen zwischendurch geplant
\item Das Spiel ist für einen Spieler konzipiert

\end{itemize}