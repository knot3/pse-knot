\chapter{Glossar}

\begin{tabular}{|p{0.25\textwidth}|p{0.75\textwidth}|}
\hline 
\ Knot~$^3$ & Spielkonzept und Spiel-Name (engl. für Knoten)\\
\hline
\ Knoten & % !!! %
\\
\hline
\ Transformieren & Verändern des Knoten durch Verschiebung der Kanten und Teilkanten\\
\hline 
\ Tutorial & Vereinfachter Freibau-Modus (Sandkasten-Modus)  in dem das grundlegende Bedienkonzept erläutert wird. Es ist über das Hauptmenü erreichbar.\\
\hline 
\ Ankerpunkt & wird bei der Transformation des Knoten als neue Kante betrachtet. Wird benötigt wenn eine Kante nur teilweise transformiert werden soll (z.B. halbieren einer Kante)\\ 
\hline 
Hauptmenü & Dieses Menü ist der erste Bildschirm mit dem der Spieler interagieren kann. Hier kann er Einstellungen zum Spiel vornehmen (z.B. Grafik und Ton) oder ein neues Spiel in einem der beiden Modi starten. \\ 
\hline 
Pause-Menü & Sonderform vom Hauptmenü in dem Einstellungen zum laufenden Spiel getätigt werden können (z.B. Speichern, Laden, Grafikeinstellungen, Rückkehr zum Hauptmenü (beenden des aktuellen Spiels) und Verlassen Spiels) \\ 
\hline 
Einstellungsmenü & In diesem Menü sind Einstellungen zu Grafik und Ton möglich. Erreichbar über das Hauptmenü bzw. Pause-Menü \\ 
\hline 
Referenzknoten & Bildet die Referenz für die Transformation des Ausgangsknoten im Challengen-Modus\\ 
\hline 
Ausgangsknoten & Diesen Knoten muss der Spieler im Challenge-Modus transfomieren, sodass er dem Referenzknoten gleicht \\ 
\hline 
Abschlussbildschirm & Ist der eingeblendete Bildschirm nach dem erfolgreichen Abschluss eines Levels im Challenge-Modus. Hier wird Platzierung des Spielers in der Bestenliste angezeigt (anhand der Spielzeit) und der Spieler kann das Level bewerten.\\ 
\hline 
Austauschdatei-Format & %% !!!
\\
\hline
Undo & %% !!!
\\
\hline
Challenge & Spielmodus: Der Spieler bekommt die Aufgabe einen vorgegebenen
Knoten nachzubauen.\\
\hline

\end{tabular}
