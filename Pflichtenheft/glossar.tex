\chapter{Glossar}

\begin{longtable}{|p{0.25\textwidth}|p{0.75\textwidth}|}
\hline 
\ Knot~$^3$ & Spielkonzept und Spiel-Name (engl. für Knoten)\\
\hline
\ Knoten & Im Spiel arbeitet der Spieler an einem dreidimensionalen (Gitter-)Knoten, dabei beginnt er mit einer Ausgangsform (im Zweidimensionalen z.B. ein Quadrat). Wie am Beispiel des Quadrats zu sehen ist, besteht ein Knoten aus einem geschlossenen Gebilde.\\
\hline
\ Transformieren & Verändern des Knoten durch Verschiebung der Kanten und Teilkanten\\
\hline 
\ Tutorial & Vereinfachter Freibau-Modus (Sandkasten-Modus)  in dem das grundlegende Bedienkonzept erläutert wird. Es ist über das Hauptmenü erreichbar.\\
\hline 
Hauptmenü & Dieses Menü ist der erste Bildschirm mit dem der Spieler interagieren kann. Hier kann er Einstellungen zum Spiel vornehmen (z.B. Grafik und Ton) oder ein neues Spiel in einem der beiden Modi starten. \\ 
\hline 
Pause-Menü & Sonderform vom Hauptmenü in dem Einstellungen zum laufenden Spiel getätigt werden können (z.B. Speichern, Laden, Grafikeinstellungen, Rückkehr zum Hauptmenü (beenden des aktuellen Spiels) und Verlassen Spiels) \\ 
\hline 
Einstellungsmenü & In diesem Menü sind Einstellungen zu Grafik und Ton möglich. Erreichbar über das Hauptmenü bzw. Pause-Menü \\ 
\hline 
Referenzknoten & Bildet die Referenz für die Transformation des Ausgangsknoten im Challengen-Modus\\ 
\hline 
Ausgangsknoten & Diesen Knoten muss der Spieler im Challenge-Modus transfomieren, sodass er dem Referenzknoten gleicht \\ 
\hline 
Abschlussbildschirm & Ist der eingeblendete Bildschirm nach dem erfolgreichen Abschluss eines Levels im Challenge-Modus. Hier wird Platzierung des Spielers in der Bestenliste angezeigt (anhand der Spielzeit) und der Spieler kann das Level bewerten.\\ 
\hline 
Austauschdatei-Format & %% !!!
\\
\hline 
Easteregg & versteckte Funktionen und Spielinhalte %% !!!
\\
\hline 
Render-Modus &  %% !!!
\\
\hline
Undo & Mit der Undo-Funktion kann eine vorherige Transformation zurückgenommen werden.
\\
\hline
Redo & Mit der Redo-Funktion kann eine zurückgenommene Transformation wiederhergestellt werden.
\\
\hline
Challenge & Spielmodus: Der Spieler bekommt die Aufgabe einen vorgegebenen Knoten nachzubauen.\\
\hline
Creative(-Mode) & Der Creative(-Mode) ist der erste Spielmodus. Im Creative(-Mode) baut der Spieler ausgehend von einer Grundform einen beliebigen (Gitter-)Knoten. Das Spiel gibt dem Spieler einige Hilfsfunktionen zur Bewertung der Komplexität seines gebauten Knotens.\\
\hline
Bestenliste & Zu jeder Challenge gibt es eine Bestenliste. 
Die Liste ist nach den Zeiten der schnellsten Spieler geordnet: auf dem ersten Platz ist der Schnellste, auf dem letzten Platz ist der Langsamste.
\\
\hline
Textur & Flächige Verbindungen zwischen Kanten.\\
\hline
Credits & Die Nennung aller Mitwirkenden an der Entwicklung von Knot³. Im Spiel zeigt ein Klick auf Knot³ die Credits an.\\
\hline
Windows Zielsysteme & Systeme für die das Spiel Knot³ entwickelt ist und ohne Probleme laufen sollte. Windows 7 und Windows 8.1 sind Zielsysteme.\\
\hline
(Spiel-)Abbruch & Wenn der Spieler ein Spiel vorzeitig beendet. Ein Klick auf "Pause", gefolgt von einem Klick auf "Quit" führt zu einem (Spiel-)Abbruch.\\
\hline
Shadereffekte & \\
\hline
Knoten-Komplexitätsmaße & Funktionen, welche den Sieler im Creative(-Mode) unterstützen seinen Knoten zu bewerten.\\
\hline
Virtuelle Knoten & Wenn ein Spieler in Knot³ einen Zug ausführt, werden ihm durch eine vorläufige Skizzierung der Knoten-Transformationen (je nach Interaktion) die möglichen Resultate des Zugs gezeigt. \\
\hline
Spielernamen & Der Name des Spielers, wie er ihn in Knot³ eingestellt hat.\\
\hline
Level & In sich beendetes Spiel: Eine Challenge ist gleichzeitig ein Level. Ein Level hat einen Startknoten und einen Zielknoten. Transformiert der Spieler den Startknoten durch mehrere Schritte in den Zielknoten, so ist das Level beendet. Es gibt verschiedene Standard-Levels, welche von 1-10 mit steigender Schwierigkeit geordnet sind.\\
\hline
Zug & Ein (Spiel-)Zug ist die Interaktion des Spielers mit dem 3D-Modell des Knotens, um selbigen zu transformieren. Zug meint i. A. einen gültigen Zug und ist die Kurzversion für Knoten-Transformation.\\
\hline
Gültiger Zug & Eine Knoten-Transformation.\\
\hline
Ungültiger Zug & Züge, welche den Knoten zerstören könnten sind ungültig, nicht erlaubt und nicht durchführbar. Z.B. dürfen sich Kanten auf einer Geraden nicht berühren. Für weitere Informationen hierzu, siehe {\color{red}[TODO: VERWEIS]}\\
\hline
Kamera & Die Ansicht des Spielers während eines Spiels auf den Knoten.\\
\hline
Hauptmenü & Das erste nach dem Spielstart sichtbare Menü.\\
\hline
Modifikation & Beschreibt eine beliebige Änderung am Knoten. Umfasst damit Transformationen, Einfärben, ... alles was den Knoten ändert.\\
\hline
\end{longtable}
