\chapter{Produkteinsatz}

Das Spiel dient zu Unterhaltungszwecken.

\section{Anwendungsbereiche}


\begin{itemize}

	\item Unterhaltungssoftware im Heimanwendungsbereich. 
	
	\item Ein Werkzeug für Künstler zur Modellierung von 3D-Knoten, z.B. als Minikunstwerke, geeignet für den 3D-Druck.
	
	\item Ein Gedächtnis-/Knobelspiel zum Training der
	geistigen Fähigkeiten.
	
	\item Als Mini-Spiel in anderen Spielen. Angenommen die Entwickler eines anderen Spiels beschließen
	an irgendeiner  Stelle den Spieler mit einer verschlossenen Tür zu konfrontieren. Diese muss
	geöffnet werden. Die Entwickler nutzen die Zeit, um das Terrain hinter der Tür zu laden. Hier könnte
	Knot³ als "Hacktool" dem Spieler helfen die Tür zu öffnen und den Entwicklern Zeit geben..
	
	
\end{itemize}

\section{Zielgruppen}

Da das Spiel allein von räumlichem Vorstellungsvermögen abhängt kann prinzipiell jeder, der die Bedienung mit Maus und Tastatur versteht es spielen, vorausgesetzt er versteht Englisch.
\\
Das Spiel richtet sich jedoch besonders an Leute die Spaß an kreativem Erstellen von 3D-Knoten haben, bzw. ihr Räumliches Vorstellungsvermögen im Challenge-modus unter Beweis stellen wollen.





