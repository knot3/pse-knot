\chapter{Produkteinsatz}

Das Spiel soll Spieler mit Vorliebe für das Gestalten im dreidimensionalem Raum ansprechen sowie auch Spieler die gerne ihre Fähigkeiten im Vorstellen von Räumlichen Sachverhalten mit anderen vergleichen und messen wollen.

\section{Anwendungsbereiche}


\begin{itemize}

	\item Unterhaltungssoftware im Heimanwendungsbereich. 
	
	\item Ein Werkzeug für Künstler zur Modellierung von 3D-Knoten, z.B. als Minikunstwerke, geeignet für den 3D-Druck.
	
	\item Ein Gedächtnis-/Knobelspiel zum Training der
	geistigen Fähigkeiten.
	
	
	
\end{itemize}

\section{Zielgruppen}

Da das Spiel allein von räumlichem Vorstellungsvermögen abhängt kann prinzipiell jeder, der die Bedienung mit Maus und Tastatur versteht es spielen, vorausgesetzt er versteht Englisch.
\\
Das Spiel richtet sich jedoch besonders an Leute die Spaß an kreativem Erstellen von 3D-Knoten haben, bzw. ihr Räumliches Vorstellungsvermögen im Challenge-modus unter Beweis stellen wollen.





