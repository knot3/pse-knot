\documentclass{scrreprt}

\usepackage{lmodern} % Löst Problem mit nicht korrekt dargestellten Zeichen in SVG-Grafiken

\usepackage[utf8x]{inputenc}
\usepackage[ngerman]{babel}
\usepackage{microtype}
\usepackage{enumitem}
\usepackage{graphicx}


\usepackage{svg} % Fragen? Siehe GitHub-Wiki oder E-Mail an pse oder Issue. Das svg-Paket ist auf Win. nicht trivial.


\usepackage{hyperref} % Soll am Ende stehen. Für Links innerhalb und nach außen.


\begin{document}

\title{Pflichtenheft\\~\\~\\\Huge{KNOT$^3$}\\~\\\Large (\href{http://pp.info.uni-karlsruhe.de/lehre/WS201314/pse/}{Praxis der Softwareentwicklung am KIT}: Echtzeit-Computergrafik in der Spieleentwicklung \\am \href{http://cg.ibds.kit.edu/index.php}{Lehrstuhl für Computergrafik})}
\author{Tobias Schulz, Maximilian Reuter, Pascal Knodel, \\Gerd Augsburg, Christina Erler, Daniel Warzel} 
\date{\today}

\maketitle
% %TODO: Vorlage mit:
% Titel
% (KIT-/Knot³)Logo (wenn möglich)
% Felder für Veranstaltung, 
% PSE-Teilnehmer, 
% Betreuer, 
% Versionsnummer, 
% Erstes Datum und das der letzten Aktualisierung ...

\tableofcontents

% TODO: Inkonsistente Bezeichner /F 10/, /NF10/! Gemeinsames Format definieren, 
% evtl. d. LaTeX-Befehl überprüfen. Hinweis: Keine Nummerierung, bereits vergebene Bezeichner stehen fest/
% könnten bereits refernziert worden sein! Lediglich das Format einhalten!

\chapter{Zielbestimmung}

Das Spiel versetzt einen einzelnen Spieler in die Lage Knoten im dreidimensionalen Raum zu erstellen und zu modifizieren. Zwischen den Kanten der Knoten besteht die Möglichkeit Flächen einzusetzen und diese zu texturieren. Zudem wird dem Spieler erlaubt sich in verschiedenen Herausforderungen mit anderen Spielern zu messen.\\


\section{Musskriterien}

\begin{itemize}

	\item Spielmodus 1 Freies Erstellen
	
	\item Spielmodus 2 Challenges
	
	\item Knotenübergänge müssen eindeutig erkennbar sein.
	
	\item Darstellung mit passenden 3D-Modellen an Übergängen.
	
	\item Selektion und Modifikation von Kantenzügen.
	
	\item Übergehen unmöglicher Zustände, wenn möglich.
	
	\item Highscores: Heuristik zur Komplexität / Eindeutigkeit.
	
	\item einfaches Datenaustauschformat für die Levels
	
	\item mindestens zehn eindeutige Level mit steigendem Schwierigkeitsgrad.
	
	\item intuitive Steuerung
	
	\item sinnvolles Undo
	
	\item gute automatische Kameraführung
	\item Standard Sprache ist Englisch
	

	\item  Einfaches Speicherformat das lokal Austauschbar ist
	
\end{itemize}

\section{Kannkriterien}


\begin{itemize}
\item Sound soll unterstützen

\item Der Einsatz von Hintergrundmusik

\item Eine veränderbare Tastaturbelegung

\item Einfärbung von Kanten nach Spieler Präferenz

\item zusätzliche Lokalisierung in Deutsch

\item Redo welches vorangegangene Undo rückgängig macht

\item optionale Flächenerstellung zwischen benachbarten Kanten

\item nach Challenge Beendigung sofortiges Wiederholen möglich

\item Spielerbewertungen für Knoten

\item Durchschnittszeit des Bestehens einer Challenge

\item Eastereggs können gefunden werden

\item Unterstützende Tutorials die den Einstieg erleichtern
	
\item Der Einsatz eines oder mehrerer Shadereffekte

\item Der Einsatz von besonderen Rendereffekten 

\item Online-Austausch der Leveldaten

\item 3D-Drucker kompatible Ausgabe der Leveldaten
\end{itemize}

\section{Abgrenzungskriterien}


\begin{itemize}



\item Das Spiel ist kein 3D-Modelierungsprogramm

\item Das Spiel ist für einen Spieler konzipiert

\end{itemize} % .tex's weg :P!

\chapter{Produkteinsatz}

Das Spiel dient zu Unterhaltungszwecken.

\section{Anwendungsbereiche}


\begin{itemize}

	\item Unterhaltungssoftware im Heimanwendungsbereich. 
	
	\item Modellierung von 3D-Knoten,
	z.B. als Minikunstwerke, geeignet für den 3D-Druck.
	
	\item Ein Gedächtnis-/Knobelspiel zum Training der
	geistigen Fähigkeiten. 
	
\end{itemize}

\section{Zielgruppen}

Jugendliche und Erwachsene.






\input{produktumgebung.tex}

\chapter{Funktionale Anforderungen}

% TODO: Feingliederung der Funktionalen Anforderungen (Kategorien für z.B. f. Dinge die während eines Spiels passieren, Einstellungen, ...)

\begin{tabular}{|p{0.125\textwidth}|p{0.875\textwidth}|}
\hline 
\textbackslash F 110  \textbackslash & Speicherung einer Bestenliste für die Levels \\ 
\hline 
\textbackslash F 120  \textbackslash  & Import und Export des Austauschdatei-Formates \\ 
\hline 
\textbackslash F 130  \textbackslash  & Strukturierte Übersicht über alles importierten Levels \\ 
\hline
\textbackslash F 140  \textbackslash  & Wechsel zwischen verschiedenen Kameraeinstellungen (Geführte, zentrierte oder frei-bewegliche Kamera)  \\ 
\hline
\textbackslash F 150  \textbackslash  & Setzen von neuen Ankerpunkten an Kanten \\ 
\hline
\textbackslash F 170  \textbackslash  & Durch Tastendruck (ESC?) ist das Pause-Menü erreichbar\\ 
\hline
\textbackslash F 180  \textbackslash  & Standard Grafikeinstellungen werden vom Programm vorgegeben oder bestimmt.\\ 
\hline
\textbackslash F 190  \textbackslash  & Im Tutorial werden über Textausgaben und grafische Visualisierungen dem Spieler schrittweise die einzelnen Bedienungsmöglichkeiten beigebracht  \\
\hline
\textbackslash F 200  \textbackslash  & Der Spieler kann Einstellungen zur Grafik und dem Ton im Menüpunkt Einstellungen des Hauptmenüs bzw. Pause-Menü vornehmen\\
\hline
\textbackslash F 210  \textbackslash  & Beim Starten des Frei-Bau-Modus (Sandkasten-Modus) wird dem Spieler ein einfacher Knoten (4 Kanten, 4 Ecken, Basisquadrat?) zum Transfomieren bereitgestellt \\
\hline
\textbackslash F 220  \textbackslash  & Beim Verlassen des aktuellen Spiels (Frei-Bau-Modus) über das Pause-Menü wird nachgefragt ob der aktuelle Knoten gespeichert werden soll, ohne Speicher der Modus verlassen werden soll oder ob dieser Vorgang abgebrochen werden soll\\ % TODO: verbessern.
\hline
\textbackslash F 230  \textbackslash  & Der Spieler kann im Frei-Bau-Modus (Sandkasten-Modus) aus zwei erstellten Knoten eine Level für den Challenge-Modus erstellen.\\
\hline
\textbackslash F 240  \textbackslash  & Nach der Auswahl des Challenge-Modus kann der Spieler über eine Übersicht ein Level auswählen anhand von Schwierigkeitsgrad, Bewertungen oder Bestzeit \\
\hline
\textbackslash F 250  \textbackslash  & Nach dem Start eines Levels sieht der Spieler beide Knoten (Ausgangsknoten und Referenzknoten) und kann die Ansicht beliebig verändern. Sobald er die erste Veränderung am Ausgangsknoten vornimmt startet die Zeitmessung\\
\hline
\textbackslash F 260  \textbackslash  & Spiel prüft den transformierten Ausgangsknoten auf Gleichheit mit dem Referenzknoten. Falls Gleichheit besteht wird die Zeit angehalten und der Abschlussbildschirm wird eingeblendet \\
\hline
\textbackslash F 270  \textbackslash  & Das Spiel speichert die Platzierung des Spielers in der Bestenliste und die Bewertung des Levels\\
\hline
\textbackslash F 280  \textbackslash  & Spieler können Namen/Nicknames eingeben, welche gespeichert werden.\\
\hline
\end{tabular} 
	


\chapter{Produktdaten}

\renewcommand{\theenumi}{/PD\_\arabic{enumi}0/}
\renewcommand{\labelenumi}{\theenumi}

\begin{enumerate}

\item Jeder Nutzer hat die Möglichkeit ein persönliches Spielerprofil anzulegen. Diese Daten unterstützen ihn auch bei späteren Aktionen, wie z.B. durch eine Autovervollständigung bei einem neuen Bestenlisteneintrag. 

  \begin{itemize}
     \item Nickname
     \item Schwierigkeitsgrad
  \end{itemize}

\item Eine Spielestatistik bietet dem aktuellen Spieler eine Übersicht über gebaute Knoten und absolvierte Challenges.

  \begin{itemize}
     \item Spielzeit
     \item Errungenschaften
     \item Bestandene Challenges
     \item Übersicht der Creatives
  \end{itemize}

\item Standard-Spracheinstellungen sind verfügbar

  \begin{itemize}
     \item Deutsche Sprache
     \item Englische Sprache
  \end{itemize}
  
\item Neue Sprachpakete können von der Support-Webseite gezogen werden.
\item 10-Levels sind bei jedem Knot³-Spielpaket enthalten.

  \begin{itemize}
     \item Levelname
     \item Empfehlung (Anfänger oder Fortgeschrittene)
  \end{itemize}

\item Standard-Grafikeinstellungen werden einmalig beim ersten Start des Spiels ermittelt.
\item Vom Spieler angepasste Grafikeinstellungen sind auch beim nächsten Spielstart weiterhin aktiv.
\item Standard-Steuerungseinstellungen sind voreingestellt.
\item Spielstände des 2. Modus (Spielstandname, Spieler, Spielzeit, ...) können aus einer eigenen Übersicht ausgewählt und geladen werden.
\item Jeder Windows-Nutzer, für den eine lokale Kopie von Knot³ installiert wurde, kann seine Erfolge in einer lokale Bestenliste einsehen.
\item Bei der Entwicklung werden Visual Studio C-Sharp/XNA 4.0 Projekte gespeichert.
\item Bei der Entwicklung entstehen Dokumentationen der Quelltexte.
\item Grafiken welche bei der Benutzeroberfläche eingebunden werden.
\item Die Dokumentation des Spiels für den Spielers ist im Hauptmenü abrufbar.
\item Die Online-Bestenliste ist über das WWW aufrufbar.

  \begin{itemize}
     \item Spielername
     \item Datum
     \item Erreichte Punkte
     \item Spieldauer
  \end{itemize}
  
\item Eine Knot³-Homepage ist erste Anlaufstelle für Support, Downloads und die Bestenliste.
\item Es gibt eine Webseite von der die Online-Bestenliste abrufbar ist.
\item Es gibt eine Support-Webseite für Sprachpakete und neue Levels.
\item Die Adresse, an welche die Highscores zur Online-Veröffentlichung gesendet werden ist in den Einstellungen gespeichert.
\item Knoten bei Spielständen des 1. Modus werden in einem Standardformat gespeichert.

  \begin{itemize}
     \item Knoten
     \item Hinweis: Die Komplexität wird dynamisch berechnet und muss nicht gespeichert werden.
  \end{itemize}
  
\item Knoten bei Spielständen des 1. Modus werden in einem Format welches 3D-Drucker verstehen gespeichert.



\end{enumerate}




\chapter{Nichtfunktionale Anforderungen}

\renewcommand{\theenumi}{/NF\_\arabic{enumi}0/}
\renewcommand{\labelenumi}{\theenumi}

\begin{enumerate}

\item Transformierung des Knotens muss durch die Maus möglich sein.
\item Die Kamera muss mit Hilfe der Maus und der Tastatur navigierbar sein (Drehen, Zoomen und Bewegen)
\item Das Spiel sollte unter Standard-Grafikeinstellungen immer mindestens eine Bildwiederholungsrate von 30 Bildern pro Sekunde haben.
\item Grafische Gestaltung der Knoten soll die Übersicht des Spielers nicht einschränken oder verschlechtern  
\item Übersichtliche Menüführung, u. A. durch den Einsatz von Alternativen zur Navigation über aufklappbare Listen.
\item Intuitive Spielsteuerung, welche schnell erlernbar ist. 
\item Erweiterbarkeit durch Einbindung von Internationalisierungen. %opt.
\item Einstellen kontrastreicher Farben für Menschen mit "eingeschränkten Sehfähigkeiten". % opt.
\item Betrügereien bei den Highscores sollen automatisch erkannt/ersichtlich werden. % opt.
\item Starten und anschließendes Beenden muss in weniger als 45 Sekunden möglich sein. % !!! Besserer Wert? %
\item Speichern darf den Dialog mit dem Spieler nicht wesentlich verzögern. 
\item Als Standard-Sprache für die grafische Oberfläche ist Englisch voreingestellt.
\item Strukturierte Übersicht über alle importierten Levels.

\end{enumerate}


\chapter{Globale Testfälle}

\renewcommand{\theenumi}{/T\_\arabic{enumi}0/}
\renewcommand{\labelenumi}{\theenumi}

\begin{enumerate}
\item Die Grafikauflösung wird im Einstellungsmenü verändert.\\
	 	\textit{Erwartet:} Das Spiel verwendet die gewünschte Auflösung sofern diese vom System unterstützt wird. Falls nicht wird eine Fehlermeldung eingeblendet, die darauf hinweist, dass diese Einstellung nicht möglich ist. Die Auflösung wird in diesem Fall nicht geändert.
	 	
\item Die Lautstärke der Musik und Toneffekte wird im Einstellungsmenü angepasst. \\  
\textit{Erwartet:} Bei erhöhter Lautstärke wird die Musik oder die Toneffekte lauter abgespielt als bei niedrigeren Einstellungen. Die Soundeffekte oder Musik werden nicht abgespielt wenn die Lautstärke auf den Wert 0 gestellt wurde. Falls nur die Musik auf 0 gestellt wird wird nur die Musik nicht abgespielt, aber die Toneffekte werden mit ihrer Lautstärke weiterhin ausgegeben.
\item Beenden des Spiels über das Hauptmenü\\
\textit{Erwartet:} Das Spiel schließt sich vollständig, d.h. alle laufenden Prozesse des Spieles werden beendet und der Speicher wird freigeben.
\item Verlassen eines aktiven Spiels über das Pause-Menü.\\
\textit{Erwartet:} Nach dem klicken auf den  über das Pause-Menü erscheint das Hauptmenü.
\item Transformieren des Knotens sowohl im Challenge-Modus als auch im Creative-Modus.\\
\textit{Erwartet:} Falls die Transformation gültig ist wird die Kante entsprechend transformiert. Dies funktioniert sowohl im Challenge-Modus als auch im Creative-Modus.
\item Kamerapostion verändern (bewegen, drehen und zoomen) im Challengen-Modus als auch im Creative-Modus.\\
\textit{Erwartet:} Die Kameraposition verändert sich wie gewünscht in die vorgegebene Richtung. Dies funktioniert sowohl im Challenge-Modus als auch im Creative-Modus.
\item Erfolgreiches Beenden einer Challenge. \\
\textit{Erwartet:} Die Zeit wird gestoppt und der Abschlussbildschirm wird eingeblendet. Falls die Zeit für die Bestenliste ausgereicht hat wird diese direkt eingetragen. Nun hat man die Möglichkeit die Challenge zu wiederholen oder zum Hauptmenü zu wechseln.
\item Speicherung eines Knotens den man im Creative-Modus erstellt hat und späteres Laden.\\
\textit{Erwartet:} Ein Knoten wird in einer Datei im Austauschformat gespeichert. Wenn diese Datei geladen wird erhält man den vorher abgespeicherten Knoten zurück.
\item Installation des Spiels auf Windows Zielsystemen
\item Restlose Deinstallation des Spiels von Windows Zielsystemen.

\end{enumerate}

\begin{enumerate}[resume]
\item Rückgängig machen beliebig vieler Knoten-Transformationen. 
\item Wiederholen von genau eines Schrittes, der zuvor rückgängig gemacht wurde. 
\end{enumerate}


\chapter{Entwicklung}

\section{Software}

\begin{itemize}


\item GitHub-Client
\item LateX (unterstützt durch Incscape und perl)
\item Visual Studio 2013
\item C-Sharp
\item ...

\end{itemize}

\chapter{Systemmodelle}


\section{Interaktionsverlauf}

	\begin{longtable}{|p{0.25\textwidth}|p{0.75\textwidth}|}
    \hline
    main menu & Das ist die erste Ansicht, die der Nutzer bekommt. Von hier aus erreicht er alle Bereiche des Programms.\\
    \hline
    creative & Von hier aus startet der Nutzer ein neues Spiel, mit einem einfachem Standardknoten, lädt einen Speicherstand oder startet das erstellen von neuen Herausforderungen (challenges).\\
    \hline
    challenge & Eine Übersicht, der vorhandenen Herausforderungen. Der Nutzer kann nach verschieden Kriterien suchen und sortieren lassen und in einer Vorschau weitere Informationen betrachten.\\
    \hline
    settings & Einstellungen an Grafik, Ton und Steuerung. Außerdem kann die persönliche Farbpalette angepasst werden.\\
    \hline
    credits & Zeigt Infos über die Mitwirkenden an dem Programm und über das Programm selber.\\
    \hline
    
   \end{longtable}
   
   ~\\
    
	\begin{figure}[h]
		\centering
	 	\includesvg[width = 0.95\textwidth]{menu}
	 	\caption{Hauptmenüeinträge}
	\end{figure}
	
	\clearpage
	~\\
	
	Im Spiel kann der Nutzer auch die {\color{red} Einstellungen} erreichen. Die Menüeinträge in den unterschiedlichen Spielmodi können variieren.
		
	\begin{longtable}{|p{0.25\textwidth}|p{0.75\textwidth}|}
	
	\hline
	settings & Genau wie aus dem Hauptmenü.\\
	\hline
	save & Speichert den aktuellen Spielstand.\\
	\hline
	quit & Beendet das laufende Spiel. \\
	\hline
	render options & Bietet dem Nutzer verschiedene Möglichkeiten seinen Knoten zu rendern und zu exportieren.\\
	\hline
	
	\end{longtable}
	
	\begin{figure}[!ht]
		  \centering
		  \includesvg[width = \textwidth]{ingamemenu}
		  \caption{Menü f. Einstellungen während des Spiels.}
	\end{figure}
	
\clearpage


\section{Benutzerinteraktionsmodelle}

	\begin{figure}[ht]
		  \centering
		  \includesvg[width = 0.9\textwidth]{Auswahl}
	\end{figure}
	
\clearpage

\subsection{Spielzüge}

~\\
...

\subsubsection{Beispiele gültiger Züge}

~\\
...

\subsubsection{Beispiele ungültiger Züge}

	\begin{figure}[htb]
	  \centering
	  \includegraphics[width = \textwidth]{Systemmodelle/Ungueltiger_Zug.png}
	  \caption{Parallele Kantenvereinigung}
	  \label{fig:zug1}
	\end{figure}

Der Knoten auf der linken Seite \ref{fig:zug1} beschreibt eine gültige Spielsituation. Der Spieler wählt eine Kante (blaue Hervorhebung) aus, um einen weiteren Zug vorzunehmen.
Einem Spieler ist es nicht möglich, zwei parallele Kanten (hier: die Blaue und die Rote) zu einer Kante zu vereinen. Der Knoten soll immer aus einem geschlossenen Kreis von Kanten bestehen. Der Knoten auf der rechten Seite \ref{fig:zug1} ist daher eine ungültige Spielsituation.

\clearpage

	\begin{figure}[htb]
	  \centering
	  \includegraphics[width = \textwidth]{Systemmodelle/Ungueltiger_Zug2.png}
	  \caption{{\color{red}Änderung der Kantenzuordnung}}
	  \label{fig:zug2}
	\end{figure}
	
Auf der 

\clearpage

\section{Grafische Bedienuns-Oberflächen}

	\begin{figure}[ht]
	  \centering
	  \includegraphics[width = 0.95\textwidth]{Systemmodelle/01_Knot3-mainscreen.png}
	  \caption{Hauptmenü}
	\end{figure}

	\begin{figure}[ht]
	  \centering
	  \includegraphics[width = 0.95\textwidth]{Systemmodelle/04_Knot3-select-Challenge.png}
	  \caption{Menü für Herausforderungen, mit Ausschnitt der Bestenliste}
	\end{figure}
	
	\begin{figure}[ht]
	  \centering
	  \includegraphics[width = 0.95\textwidth]{Systemmodelle/05_Knot3-Colour-select.png}
	  \caption{Creative: Kantenfärben}
	\end{figure}
	
	\begin{figure}[ht]
	  \centering
	  \includegraphics[width = 0.95\textwidth]{Systemmodelle/08_Knot3-menu-graphics.png}
	  \caption{Menü f. Grafikeinstellungen}
	\end{figure}

	
\clearpage
	
\section{Anwendungsfälle}

~\\
...

	\begin{figure}[ht]
	  \centering
	  \includesvg[svgpath=Systemmodelle/, width = \textwidth]{inGame}
	  \caption{Interaktionen während eines Spiels (allgemein)}
	\end{figure}

%	\begin{figure}[ht]
%	  \centering
%	  \includesvg[width = \textwidth]{Anwendungsfall}
%	  %\caption{...}
%	\end{figure}



\chapter{Glossar}

\begin{longtable}{|p{0.25\textwidth}|p{0.75\textwidth}|}
\hline 
\ Knot~$^3$ & Spielkonzept und Spiel-Name (engl. für Knoten)\\
\hline
\ Knoten & Im Spiel arbeitet der Spieler an einem dreidimensionalen (Gitter-)Knoten, dabei beginnt er mit einer Ausgangsform (im Zweidimensionalen z.B. ein Quadrat). Wie am Beispiel des Quadrats zu sehen ist, besteht ein Knoten aus einem geschlossenen Gebilde.\\
\hline
\ Transformieren & Verändern des Knoten durch Verschiebung der Kanten und Teilkanten\\
\hline 
\ Tutorial & Vereinfachter Freibau-Modus (Sandkasten-Modus)  in dem das grundlegende Bedienkonzept erläutert wird. Es ist über das Hauptmenü erreichbar.\\
\hline 
Hauptmenü & Dieses Menü ist der erste Bildschirm mit dem der Spieler interagieren kann. Hier kann er Einstellungen zum Spiel vornehmen (z.B. Grafik und Ton) oder ein neues Spiel in einem der beiden Modi starten. \\ 
\hline 
Pause-Menü & Sonderform vom Hauptmenü in dem Einstellungen zum laufenden Spiel getätigt werden können (z.B. Speichern, Laden, Grafikeinstellungen, Rückkehr zum Hauptmenü (beenden des aktuellen Spiels) und Verlassen Spiels) \\ 
\hline 
Einstellungsmenü & In diesem Menü sind Einstellungen zu Grafik und Ton möglich. Erreichbar über das Hauptmenü bzw. Pause-Menü \\ 
\hline 
Referenzknoten & Bildet die Referenz für die Transformation des Ausgangsknoten im Challengen-Modus\\ 
\hline 
Ausgangsknoten & Diesen Knoten muss der Spieler im Challenge-Modus transfomieren, sodass er dem Referenzknoten gleicht \\ 
\hline 
Abschlussbildschirm & Ist der eingeblendete Bildschirm nach dem erfolgreichen Abschluss eines Levels im Challenge-Modus. Hier wird Platzierung des Spielers in der Bestenliste angezeigt (anhand der Spielzeit) und der Spieler kann das Level bewerten.\\ 
\hline 
Austauschdatei-Format & %% !!!
\\
\hline 
Easteregg & versteckte Funktionen und Spielinhalte %% !!!
\\
\hline 
Render-Modus &  %% !!!
\\
\hline
Undo & Mit der Undo-Funktion kann eine vorherige Transformation zurückgenommen werden.
\\
\hline
Redo & Mit der Redo-Funktion kann eine zurückgenommene Transformation wiederhergestellt werden.
\\
\hline
Challenge & Spielmodus: Der Spieler bekommt die Aufgabe einen vorgegebenen Knoten nachzubauen.\\
\hline
Creative(-Mode) & Der Creative(-Mode) ist der erste Spielmodus. Im Creative(-Mode) baut der Spieler ausgehend von einer Grundform einen beliebigen (Gitter-)Knoten. Das Spiel gibt dem Spieler einige Hilfsfunktionen zur Bewertung der Komplexität seines gebauten Knotens.\\
\hline
Bestenliste & Zu jeder Challenge gibt es eine Bestenliste. 
Die Liste ist nach den Zeiten der schnellsten Spieler geordnet: auf dem ersten Platz ist der Schnellste, auf dem letzten Platz ist der Langsamste.
\\
\hline
Textur & Flächige Verbindungen zwischen Kanten.\\
\hline
Credits & Die Nennung aller Mitwirkenden an der Entwicklung von Knot³. Im Spiel zeigt ein Klick auf Knot³ die Credits an.\\
\hline
Windows Zielsysteme & Systeme für die das Spiel Knot³ entwickelt ist und ohne Probleme laufen sollte. Windows 7 und Windows 8.1 sind Zielsysteme.\\
\hline
(Spiel-)Abbruch & Wenn der Spieler ein Spiel vorzeitig beendet. Ein Klick auf "Pause", gefolgt von einem Klick auf "Quit" führt zu einem (Spiel-)Abbruch.\\
\hline
Shadereffekte & \\
\hline
Knoten-Komplexitätsmaße & Funktionen, welche den Sieler im Creative(-Mode) unterstützen seinen Knoten zu bewerten.\\
\hline
Virtuelle Knoten & Wenn ein Spieler in Knot³ einen Zug ausführt, werden ihm durch eine vorläufige Skizzierung der Knoten-Transformationen (je nach Interaktion) die möglichen Resultate des Zugs gezeigt. \\
\hline
Spielernamen & Der Name des Spielers, wie er ihn in Knot³ eingestellt hat.\\
\hline
Level & In sich beendetes Spiel: Eine Challenge ist gleichzeitig ein Level. Ein Level hat einen Startknoten und einen Zielknoten. Transformiert der Spieler den Startknoten durch mehrere Schritte in den Zielknoten, so ist das Level beendet. Es gibt verschiedene Standard-Levels, welche von 1-10 mit steigender Schwierigkeit geordnet sind.\\
\hline
Zug & Ein (Spiel-)Zug ist die Interaktion des Spielers mit dem 3D-Modell des Knotens, um selbigen zu transformieren. Zug meint i. A. einen gültigen Zug und ist die Kurzversion für Knoten-Transformation.\\
\hline
Gültiger Zug & Eine Knoten-Transformation.\\
\hline
Ungültiger Zug & Züge, welche den Knoten zerstören könnten sind ungültig, nicht erlaubt und nicht durchführbar. Z.B. dürfen sich Kanten auf einer Geraden nicht berühren. Für weitere Informationen hierzu, siehe {\color{red}[TODO: VERWEIS]}\\
\hline
Kamera & Die Ansicht des Spielers während eines Spiels auf den Knoten.\\
\hline
Hauptmenü & Das erste nach dem Spielstart sichtbare Menü.\\
\hline
Modifikation & Beschreibt eine beliebige Änderung am Knoten. Umfasst damit Transformationen, Einfärben, ... alles was den Knoten ändert.\\
\hline
\end{longtable}


\end{document}