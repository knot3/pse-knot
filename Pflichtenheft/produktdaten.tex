\chapter{Produktdaten}

\renewcommand{\theenumi}{/PPD\_\arabic{enumi}0/}
\renewcommand{\labelenumi}{\theenumi}

\section{Persistente Daten}

\begin{enumerate}

\item Nutzerprofile speichern Informationen zum Spieler dauerhaft. 

  \begin{itemize}
     \item Nickname
  \end{itemize}

\item Eine Spielestatistik bietet dem aktuellen Spieler eine Übersicht über gebaute Knoten und absolvierte Challenges.

  \begin{itemize}
     \item Spielzeit
     \item Errungenschaften
     \item Bestandene Challenges
     \item Übersicht der Creatives
  \end{itemize}

\item Standard-Spracheinstellungen sind verfügbar

  \begin{itemize}
     \item Deutsche Sprache
     \item Englische Sprache
  \end{itemize}
  
%\item Neue Sprachpakete für die Internationalisierung sind integrierbar.

\item In der Offline-Bestenliste für Challenges wird der Spielername und die Zeit gespeichert.
\item 10-Challenges sind bei jedem Knot³-Spielpaket enthalten.

  \begin{itemize}
     \item Levelname
     \item Empfehlung (Anfänger oder Fortgeschrittene)
  \end{itemize}

\item Standard-Grafikeinstellungen werden beim ersten Spielstart gespeichert. Vom Spieler angepasste Grafikeinstellungen sind auch beim nächsten Start weiterhin aktiv.

\item Weitere Einstellungen

  \begin{itemize}
     \item Sprache
     \item Effekte
     \item (Hintergrund-)Musik
  \end{itemize}

\item Voreingestellte Standard-Steuerungseinstellungen.
\item {\color{red}Spielstände des 2. Modus (Spielstandname, Spieler, Spielzeit, ...) können aus einer eigenen Übersicht ausgewählt und geladen werden.}
\item Die Erfolge und lokale Bestenliste. Für jede Challenge ist eine Bestenliste anzulegen.
\item Die bei der  Entwicklung entstehenden Dokumentationen.
\item Grafiken welche Teil der Benutzeroberfläche sind.
%\item Die Dokumentation des Spiels für den Spielers ist im Hauptmenü abrufbar.
\item Die Online-Bestenliste.

  \begin{itemize}
     \item Spielername
     \item Datum
     \item Erreichte Punkte
     \item Spieldauer
  \end{itemize}
  
\item Die Knot³-Homepage.
\item Die Webseite der Online-Bestenliste.
\item Die Support-Webseite.
\item Informationen zur Erreichbarkeit Webseiten (URLs).
\item Der eine Knoten im Creative(-Mode) und die zwei Knoten im Challenge(-Mode).

  \begin{itemize}
     \item Knoten (im Austauschformat)
     \item Kantenfarben
     \item Texturierung
  \end{itemize}
  
\item Knoten bei Spielständen des 1. Modus werden in einem Format welches 3D-Drucker verstehen gespeichert.
\item Soundeffekt-Dateien, welche Geräusche von Effekten enthalten.
\item Dateien für die Musikstücke, welche als Hintergrundmusik abgespielt werden.



\end{enumerate}

~\\

\renewcommand{\theenumi}{/GPD\_\arabic{enumi}0/}
\renewcommand{\labelenumi}{\theenumi}

\section{Generierbare Daten}


\begin{enumerate}

\item Der Schwierigkeitsgrad muss nicht gespeichert werden, da er dynamisch aus der Knotenstruktur berechnet wird.
\item Die Knoten-Komplexität wird dynamisch berechnet und muss nicht gespeichert werden.

\end{enumerate}
