\chapter{Produktdaten}

\renewcommand{\theenumi}{/PD\_\arabic{enumi}0/}
\renewcommand{\labelenumi}{\theenumi}


\begin{enumerate}

\item Jeder Nutzer hat die Möglichkeit ein persönliches Spielerprofil anzulegen. Diese Daten unterstützen ihn auch bei späteren Aktionen, wie z.B. durch eine Autovervollständigung bei einem neuen Bestenlisteneintrag. 

  \begin{itemize}
     \item Nickname
     \item Der Schwierigkeitsgrad muss nicht gespeichert werden, da er dynamisch aus der Knotenstruktur berechnet wird.
  \end{itemize}

\item Eine Spielestatistik bietet dem aktuellen Spieler eine Übersicht über gebaute Knoten und absolvierte Challenges.

  \begin{itemize}
     \item Spielzeit
     \item Errungenschaften
     \item Bestandene Challenges
     \item Übersicht der Creatives
  \end{itemize}

\item Standard-Spracheinstellungen sind verfügbar

  \begin{itemize}
     \item Deutsche Sprache
     \item Englische Sprache
  \end{itemize}
  
\item Neue Sprachpakete können von der Support-Webseite gezogen werden.
\item In der Offline-Bestenliste für Challenges wird der Spielername und die Zeit gespeichert.
\item 10-Challenges sind bei jedem Knot³-Spielpaket enthalten.

  \begin{itemize}
     \item Levelname
     \item Empfehlung (Anfänger oder Fortgeschrittene)
  \end{itemize}

\item Standard-Grafikeinstellungen werden einmalig beim ersten Start des Spiels ermittelt.
\item Vom Spieler angepasste Grafikeinstellungen sind auch beim nächsten Spielstart weiterhin aktiv.
\item Weitere Einstellungen

  \begin{itemize}
     \item Sprache
     \item Effekte
     \item Musik
  \end{itemize}

\item Standard-Steuerungseinstellungen sind voreingestellt.
\item Spielstände des 2. Modus (Spielstandname, Spieler, Spielzeit, ...) können aus einer eigenen Übersicht ausgewählt und geladen werden.
\item Jeder Windows-Nutzer, für den eine lokale Kopie von Knot³ installiert wurde, kann seine Erfolge in einer lokale Bestenliste einsehen. Für jede Challenge ist eine Bestenliste anzulegen.
\item Bei der Entwicklung werden Visual Studio C-Sharp/XNA 4.0 Projekte gespeichert.
\item Bei der Entwicklung entstehen Dokumentationen der Quelltexte.
\item Grafiken welche bei der Benutzeroberfläche eingebunden werden.
\item Die Dokumentation des Spiels für den Spielers ist im Hauptmenü abrufbar.
\item Die Online-Bestenliste ist über das WWW aufrufbar.

  \begin{itemize}
     \item Spielername
     \item Datum
     \item Erreichte Punkte
     \item Spieldauer
  \end{itemize}
  
\item Eine Knot³-Homepage ist erste Anlaufstelle für Support, Downloads und die Bestenliste.
\item Es gibt eine Webseite von der die Online-Bestenliste abrufbar ist.
\item Es gibt eine Support-Webseite für Sprachpakete und neue Levels.
\item Die Adresse, an welche die Highscores zur Online-Veröffentlichung gesendet werden ist in den Einstellungen gespeichert.
\item Knoten bei Spielständen des 1. Modus werden in einem Standardformat gespeichert.

  \begin{itemize}
     \item Knoten
     \item Hinweis: Die Komplexität wird dynamisch berechnet und muss nicht gespeichert werden.
  \end{itemize}
  
\item Knoten bei Spielständen des 1. Modus werden in einem Format welches 3D-Drucker verstehen gespeichert.
\item Musik-Dateien, welche Geräusche von Effekten enthalten.
\item Musik-Dateien mit der Musik zum Spiel.



\end{enumerate}


