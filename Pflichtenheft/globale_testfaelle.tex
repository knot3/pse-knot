\chapter{Globale Testfälle}

\begin{tabular}{|p{0.125\textwidth}|p{0.875\textwidth}|}
\hline 
\textbackslash T10  \textbackslash & Veränderung der Grafikauflösung  im Einstellungsmenü \\ 
\hline 
\textbackslash T20  \textbackslash & Veränderung der Lautstärke der Musik und Toneffekte im Einstellungsmenü  \\ 
\hline 
\textbackslash T30  \textbackslash & Beenden des Spiels über das Hauptmenü\\ 
\hline
\textbackslash T40  \textbackslash & Rückkehr vom Pause-Menü zum Hauptmenü und beenden des aktuellen Spiels \\ 
\hline
\textbackslash T50  \textbackslash & Beenden des Spiels über das Pause-Menü\\ 
\hline
\textbackslash T60  \textbackslash & Transformieren des Knotens sowohl im Challengen-Modus als auch im Frei-Bau-Modus.\\ 
\hline
\textbackslash T70  \textbackslash & Kamerapostion verändern (bewegen, drehen und zoomen) im Challengen-Modus als auch im Frei-Bau-Modus.\\ 
\hline
\textbackslash T80  \textbackslash & Erfolgreiches Beenden einer Challenge und Speicherung der Bestenliste und der Challenge-Bewertung\\ 
\hline
\textbackslash T90  \textbackslash & Verformung eines Knoten im Frei-Bau-Modus sowie die Speicherung dieses Knotens\\ 
\hline
\textbackslash T100  \textbackslash & Exportieren und Importieren eines Knoten.\\ 
\hline
\textbackslash T110  \textbackslash & Undo beliebig vieler Knoten-Transformationen\\ 
\hline
\textbackslash T120 \textbackslash & Speichern eines Spielstands. \\
\hline
\textbackslash T130 \textbackslash & Laden eines Spielstands. \\
\hline
\textbackslash T140 \textbackslash & Löschen von Spielspeicherständen. \\
\hline
\textbackslash T150 \textbackslash & Installation des Spiels auf einem Windows System neuerer Generation (z.B. Windows 7, 8). \\
\hline
\textbackslash T160 \textbackslash & Restlose Deinstallation des Spiels von einem Windows System neuerer Generation (z.B. Windows 7, 8). \\
\hline
\end{tabular} 
