\chapter{Globale Testfälle}

\renewcommand{\theenumi}{/T\_\arabic{enumi}0/}
\renewcommand{\labelenumi}{\theenumi}

\begin{enumerate}
\item Die Grafikauflösung wird im Einstellungsmenü verändert.\\
	 	\textit{Erwartet:} Das Spiel verwendet die gewünschte Auflösung sofern diese vom System unterstützt wird. Falls nicht wird eine Fehlermeldung eingeblendet, die darauf hinweist, dass diese Einstellung nicht möglich ist. Die Auflösung wird in diesem Fall nicht geändert.
	 	
\item Die Lautstärke der Musik und Toneffekte wird im Einstellungsmenü angepasst. \\  
\textit{Erwartet:} Bei erhöhter Lautstärke wird die Musik oder die Toneffekte lauter abgespielt als bei niedrigeren Einstellungen. Die Soundeffekte oder Musik werden nicht abgespielt wenn die Lautstärke auf den Wert 0 gestellt wurde. Falls nur Musik auf 0 gestellt wird wird nur die Musik nicht abgespielt, aber die Toneffekte werden mit ihrer Lautstärke weiterhin ausgegeben.
\item Beenden des Spiels über das Hauptmenü\\
\textit{Erwartet:} Das Spiel schließt sich vollständig, d.h. alle laufenden Prozesse des Spieles werden beendet und der Speicher wird freigeben.
\item Verlassen eines aktiven Spiels über das Pause-Menü.\\
\textit{Erwartet:} Nach dem klicken auf den  über das Pause-Menü erscheint das Hauptmenü.
\item Transformieren des Knotens sowohl im Challenge-Modus als auch im Creative-Modus.\\
\textit{Erwartet:} Falls die Transformation gültig ist wird die Kante entsprechend transformiert. Dies funktioniert sowohl im Challenge-Modus als auch im Creative-Modus.
\item Kamerapostion verändern (bewegen, drehen und zoomen) im Challengen-Modus als auch im Creative-Modus.\\
\textit{Erwartet:} Die Kameraposition verändert sich wie gewünscht in die vorgegebene Richtung. Dies funktioniert sowohl im Challenge-Modus als auch im Creative-Modus.
\item Erfolgreiches Beenden einer Challenge. \\
\textit{Erwartet:} Die Zeit wird gestoppt und der Abschlussbildschirm wird eingeblendet. Falls die Zeit für die Bestenliste ausgereicht hat wird diese direkt eingetragen. Nun hat man die Möglichkeit die Challenge zu wiederholen oder zum Hauptmenü zu wechseln.
\item Speicherung eines Knotens den man im Creative-Modus erstellt hat und späteres Laden.\\
\textit{Erwartet:} Ein Knoten wird in einer Datei im Austauschformat gespeichert. Wenn diese Datei geladen wird erhält man den vorher abgespeicherten Knoten zurück.
\item Installation des Spiels auf Windows Zielsystemen
\item Restlose Deinstallation des Spiels von Windows Zielsystemen.

\end{enumerate}

\begin{enumerate}[resume]
\item Undo beliebig vieler Knoten-Transformationen 
\end{enumerate}
