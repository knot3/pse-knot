\chapter{Systemmodelle}


\section{Interaktionsverlauf}
    main menu - Das ist der erste Ansicht, die der Nutzer bekommt. Von hier aus erreicht er alle Berreiche des Programms.
    creative - Von hier aus startet der Nutzer ein neues Spiel, mit einem einfachem Standartknoten, läd einen Speicherstand oder startet das erstellen von neuen Herausforderungen (challenges).
    challenge - Eine Übersicht, der vorhandenen Herausforderungen. Der Nutzer kann nach verschieden Kriterien suchen und sortieren lassen und in einer Vorschau weitere Informationen betrachten.
    settings - Einstellungen an Grafik, Ton und Steuerung. Außerdem kann die persönliche Fabparlette angepasst werden.
    credits - Zeigt Infos über die Mitwirkenden an dem Programm und über das Programm selber.
	\begin{figure}[htbp]
	  \centering
	  \includesvg[svgpath=Systemmodelle/, width = \textwidth]{menu}
	  %\caption{...}
	\end{figure}
	Im Spiel kann der Nutzer auch die Einstellungen ereichen. Die Menüeinträge in den unterschiedlichen Spielmodi können variiren.
	settings - Genau wie aus dem Hauptmenü.
	save - Speichert den aktuellen Spielstand.
	quit - Beendet das laufende Spiel.
	render options - Bietet dem Nutzer verschiedene Möglichkeiten seinen Knoten zu rendern und zu exportieren.
    \begin{figure}[htbp]
	  \centering
	  \includesvg[svgpath=Systemmodelle/, width = \textwidth]{ingamemenu}
	  %\caption{...}
	\end{figure}


\section{Benutzerinteratktionsmodell}

	\begin{figure}[htbp]
	  \centering
	  \includesvg[svgpath=Systemmodelle/, width = \textwidth]{Spielverlauf}
	  %\caption{...}
	\end{figure}


\section{Benutzerschnittstelle}

	\begin{figure}[htbp]
	  \centering
	  \includesvg[svgpath=Systemmodelle/, width = \textwidth]{Hauptmenue_grob_entwurf}
	  %\caption{...}
	\end{figure}

	\begin{figure}[htbp]
	  \centering
	  \includesvg[svgpath=Systemmodelle/, width = \textwidth]{Auswahl}
	  %\caption{...}
	\end{figure}
