% !TeX encoding = UTF-8
%
% Produkt-Daten
%

% % % !!!!!!! siehe auch Pflichtenheft/Produktaten.txt (alt)


\chapter{Daten}
\label{DT}~\\

%...
%\\

\subsection{Persistente Daten}~\\

\section*{\underline{Pflicht-Daten:}}~\\

\begin{ids}{\gls{PD}}

	\id[10] Nutzerprofile speichern Informationen zum Spieler dauerhaft. 
	  \begin{itemize}
         \item Nickname
      \end{itemize}
      
     \id[20] Eine Spielestatistik bietet dem aktuellen Spieler eine Übersicht über gebaute Knoten und absolvierte Challenges.
%     \begin{itemize}
%       \item Spielzeit
%       \item Errungenschaften
%       \item Bestandene Challenges
%       \item Übersicht der Creatives
%     \end{itemize}
     
     \id[30] Standard-Spracheinstellungen sind verfügbar.
     \begin{itemize}
       \item Englische Sprache
       \item Deutsche Sprache (optional)
     \end{itemize}
     
	 \id[40] In der Offline-Bestenliste für Challenges wird der Spielername und die Zeit gespeichert.

	\id[45] 10-Challenges sind bei jedem Knot³-Spielpaket enthalten.
      \begin{itemize}
      \item Levelname
      \item Empfehlung (Anfänger oder      Fortgeschrittene)
      \end{itemize}

	 \id[50] Standard-Grafikeinstellungen werden beim ersten Spielstart gespeichert. Vom Spieler angepasste Grafikeinstellungen sind auch beim nächsten Start weiterhin aktiv.
	 
	 \id[60] Weitere Einstellungen
	 \begin{itemize}
     \item Sprache
     \item Effekte
     \item (Hintergrund-)Musik
     \end{itemize}
     
	 \id[70] Die Erfolge und lokale Bestenliste. Für jede Challenge ist eine Bestenliste anzulegen.
	 
	 \id[80] Die bei der  Entwicklung entstehenden Dokumentationen.
	 
	 \id[90] Grafiken welche Teil der Benutzeroberfläche sind.
	 
	 \id[100] Der eine Knoten im Creative(-Mode) und die zwei Knoten im Challenge(-Mode).
     \begin{itemize}
       \item Knoten (im Austauschformat)
       \item Kantenfarben
       \item Texturierung
     \end{itemize}
  
	 \id[110] Soundeffekt-Dateien, welche Geräusche von Effekten enthalten.
	 
	 \id[120] Dateien für die Musikstücke, welche als Hintergrundmusik abgespielt werden.

\end{ids}
~\\
\section*{\underline{Optionale Daten:}}~\\
\begin{ids}{\gls{OD}}

	\id[10] Die Online-Bestenliste.
	 \begin{itemize}
       \item Spielername
       \item Datum
       \item Erreichte Punkte
       \item Spieldauer
     \end{itemize}
     
     \id[20] Die Knot³-Homepage.
     
     \id[30] Die Webseite der Online-Bestenliste.
     
     \id[40] Die Support-Webseite.
     
     \id[50] Informationen zur Erreichbarkeit Webseiten (URLs).
     
     \id[60] Knoten bei Spielständen des 1. Modus werden in einem Format welches 3D-Drucker verstehen gespeichert.
     
\end{ids}
~\\


\subsection{Generierbare Daten}








%%
%% Im Inhaltsverzeichnis nach diesem "chapter" einen Seitenumbruch erzwingen.
%%

\addtocontents{toc}{\protect\newpage}

