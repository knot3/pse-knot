% !TeX encoding = UTF-8
%
% Einführung über das Projekt und das zu entwickelnde Produkt:
%


\chapter{Einf{"u}hrung}
\label{EF}


\section{Projekt}

Bei Knot³ handelt es sich um ein innovatives Spiel bei dem man Knoten im Dreidimensionalem Raum entweder frei modifizieren, oder nach Vorgabe auf Zeit ineinander überführen kann.
Die Idee und das Konzept zu diesem Spiel entstand zusammen mit der Hochschule für Gestaltung in Karlsruhe und wird im Rahmen der Praxis der Softwareentwicklung von Studenten des Karlsruher Instituts für Technologie umgesetzt.

\section{Konzepte}
\label{EF:Konzepte}

\subsection{Spiel}

Das Konzept des Spieles ist in die Kategorie der Sandbox-Spiele einzuordnen. Es wird nicht wie in klassischen Spielen ein Ziel vorgegeben und verschiedene Wege gegeben dies zu erreichen, sondern es wird dem Spieler überlassen, was er machen will. Dabei bietet man ihm viele Möglichkeiten schöpferisch tätig zu sein. Die Herausforderung und die Motivation entsteht dadurch, dass es kein 3D-Modellierer ist. Man ist ist gezwungen zu abstrahieren, sich Tricks auszudenken um bestimmte Wirkungen zu erzielen und sein selbst gestecktes Ziel zu erreichen. Dabei geht der kreative Prozess schon bei der Auswahl des Motivs los, manche lassen sich besser durch Kanten darstellen als andere.
Genauso gut kann man aber auch Herausforderungen für andere Erstellen. Komplizierte Transformationen, die sich nur schwer nachbauen lassen oder gewaltige Bauten die durch ihre schiere Größe beeindrucken. Und natürlich kann man auch die von anderen Nutzern erstellten Herausforderungen bestreiten und dem Ersteller zeigen, dass sein komplizierter Knoten nicht so schwer zu durchschauen ist wie er ursprünglich dachte.
Einzig die Vorstellungskraft des Spielers limitiert die Möglichkeiten der Anwendung.


\clearpage

\section{Vorstellungen}
\label{EF:Vorstellungen}
Wir erhoffen, dass die Spieler sich übers Internet austauschen und sich gegenseitig zu immer neuen Ideen anregen, Bilder ihrer schönsten Kreationen zeigen und ihre besten Herausforderungen verteilen. In Zukunft wird es viele Nachbauten von berühmten Gebäuden, wie z.B. dem Eiffelturm oder Brandenburger Tor geben, genauso wie abstrakte Gebilde, die die verschiedensten Wirkungen erzielen. Es wird wie bei anderen Sandbox-Spielen Fan-Seiten geben die die Bilder speichern Kategorisieren und auch die dazugehörigen Knotendateien in ihrem Austauschformat bereitstellen. Es wird Kunstprojekte geben, die dieses Programm nutzen um dreidimensionale Gebilde zu erstellen, da es erheblich leichter als ein 3D-Modellierer zu bedienen ist und durch sein offenes Austauschformat leicht in viele andere Formate umgewandelt werden kann, z.B. für den 3D-Druck. Genauso wird es aber auch Spieler geben, die das Programm dazu nutzen ihre räumlich Vorstellung zu stärken zur Entspannung oder als Konzentrationsübung.
Durch seine Freiheit in der Verwendung wird Knot³ darüber hinaus auch noch in Gebieten Verwendung finden die selbst wir uns noch nicht vorstellen können.
