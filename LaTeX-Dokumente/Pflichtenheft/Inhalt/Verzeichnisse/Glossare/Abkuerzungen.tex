% !TeX encoding = UTF-8
%
% Glossar für Projekt-spezifische Abkürzungen.
%


%%%
%%% Bezeichner:
%%%


%%
%% Für Umfangs-Kriterien.
%%
\newglossaryentry{PK}{
%
	name = PK,
	description = {Pflicht-Kriterium.},
	type = AK
%
}

\newglossaryentry{OK}{
%
	name = OK,
	description = {Optionales Kriterium.},
	type = AK
%
}

%%
%% Für Nutzergruppen.
%%
\newglossaryentry{PNG}{
%
	name = PNG,
	description = {Pflicht-Nutzergruppe},
	type = AK
%
}

\newglossaryentry{NG}{
%
	name = NG,
	description = {Nutzergruppe},
	type = AK
%
}

%%
%% Für Einsatzgebiete.
%%
\newglossaryentry{PEG}{
%
	name = PEG,
	description = {Pflicht-Einsatz-Gebiet},
	type = AK
%
}

\newglossaryentry{OEG}{
%
	name = OEG,
	description = {Optionales Einsatzgebiet},
	type = AK
%
}

\newglossaryentry{IG}{
%
	name = IG,
	description = {Interessenten-Gruppe},
	type = AK
%
}

%%
%% Für Betriebsbedingungen.
%%
\newglossaryentry{BB}{
%
	name = BB,
	description = {Betriebsbedingungen},
	type = AK
%
}

%%
%% Für Betriebsmittel-Veerbrauchs-Schätungen.
%%
\newglossaryentry{VS}{
%
	name = VS,
	description = {Betriebsmittel-Verbrauchs-Schätzung},
	type = AK
%
}

%%
%% Für Funktionale-Anforderungen.
%%
\newglossaryentry{PFA}{
%
	name = PFA,
	description = {Präfix einer funktionalen Anforderungs-Kategorie, die Pflicht ist.},
	type = AK
%
}

\newglossaryentry{OFA}{
%
	name = OFA,
	description = {Präfix einer funktionalen Anforderungs-Kategorie, die optional ist},
	type = AK
%
}

%%
%% Für Nicht Funktionale-Anforderungen.
%%
\newglossaryentry{PNFA}{
%
	name = PNFA,
	description = {Präfix einer nicht-funktionalen Anforderungs-Kategorie, die Pflicht ist},
	type = AK
%
}

\newglossaryentry{ONFA}{
%
	name = ONFA,
	description = {Präfix einer nicht-funktionalen Anforderungs-Kategorie, die optional ist},
	type = AK
%
}

%%
%% Testfälle.
%%
\newglossaryentry{PTF}{
%
	name = PTF,
	description = {Pflicht-Testfall},
	type = AK
%
}

\newglossaryentry{OTF}{
%
	name = OTF,
	description = {Optionaler Testfall},
	type = AK
%
}

%%
%% Anwendungsfälle.
%%
\newglossaryentry{PAF}{
%
	name = PAF,
	description = {Pflicht-Anwendungsfall},
	type = AK
%
}

\newglossaryentry{OAF}{
%
	name = OAF,
	description = {Optionaler Anwendungsfall},
	type = AK
%
}

%%
%% Abgrenzungskriterien.
%%
\newglossaryentry{AK}{
%
	name = AK,
	description = {Abgrenzungskriterien},
	type = AK
%
}

%%
%% Daten.
%%
\newglossaryentry{PD}{
%
	name = PD,
	description = {Pflicht-Daten},
	type = AK
%
}

\newglossaryentry{OD}{
%
	name = OD,
	description = {Optionale Daten},
	type = AK
%
}

%%
%% Grafische Oberflächen.
%%
\newglossaryentry{PGO}{
%
	name = PGO,
	description = {Pflicht-Grafische-Oberflächen},
	type = AK
%
}

\newglossaryentry{OGO}{
%
	name = OGO,
	description = {Optionale Grafische Oberflächen},
	type = AK
%
}

%%
%% Interaktionsmodelle.
%%
\newglossaryentry{PIM}{
%
	name = PIM,
	description = {Pflicht-Interaktionsmodell},
	type = AK
%
}

\newglossaryentry{OIM}{
%
	name = OIM,
	description = {Optionales Interaktionsmodell},
	type = AK
%
}

%%
%% Schnittstellen.
%%
\newglossaryentry{PSS}{
%
	name = PSS,
	description = {Pflicht-Schnittstellen},
	type = AK
%
}

\newglossaryentry{OSS}{
%
	name = OSS,
	description = {Optionales Schnittstellen},
	type = AK
%
}

%%
%% Präfixe von Pflichten und Optionalem.
%%
\newglossaryentry{P}{
%
	name = P,
	description = {Präfix von Pflichten},
	type = AK
%
}

\newglossaryentry{O}{
%
	name = O,
	description = {Präfix von Optionalem},
	type = AK
%
}
