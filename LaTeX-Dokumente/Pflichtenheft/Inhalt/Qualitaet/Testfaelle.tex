% !TeX encoding = UTF-8
%
% Produkt-Qualitätssicherung:
%


\vspace{1em}


\section{Testf{"a}lle}
\label{QS:TF}

%\paragraph*{\underline{Pflicht-Testfälle:}}


%
% Testen der Funktionen.
%
\subsection{Funktionstests}

\vspace{1em}

	\begin{ids}{\gls{PTF}}

		\id[ 10] Die Grafikauflösung wird im Einstellungsmenü verändert.\\
		
		\textit{Erwartet:} Das Spiel verwendet die gewünschte Auflösung, sofern diese vom System unterstützt wird. Falls nicht, wird eine Fehlermeldung eingeblendet, die darauf hinweist, dass diese Einstellung nicht möglich ist. Die Auflösung wird in diesem Fall nicht geändert.

		\id[ 20] Auswahl des \gls{fa:Creative}-Mode und Transformation des zur Verfügung gestellten Knoten.\\
		
		\textit{Erwartet:} Spiel initialisiert einen Knoten der verändert werden kann. Der Spieler kann alle gültigen Transformationen ohne Probleme ausführen.

		\id[ 30] Erstellen einer neuen Challenge.\\
		
		\textit{Erwartet:} Nach Auswahl des \gls{fa:Ausgangsknoten} und des \gls{fa:Referenzknoten} die zur Challenge gehören, soll die Challenge erstellt werden.
		
		\id[ 40] Die Lautstärke der Musik und Toneffekte  im Einstellungsmenü anpassen.\\ 
		
		\textit{Erwartet:} Bei erhöhter Lautstärke wird die Musik oder die Toneffekte lauter abgespielt, als bei niedrigeren Einstellungen. Die Soundeffekte oder Musik werden nicht abgespielt, wenn die Lautstärke auf den Wert 0 gestellt wurde. Falls nur die Musik auf dem Wert 0 steht, wird nur die Musik nicht abgespielt, aber die Toneffekte werden mit ihrer Lautstärke weiterhin ausgegeben.
		
		\id[ 50] Beenden des Spiels über das Hauptmenü.\\
		
		\textit{Erwartet:} Das Spiel schließt sich vollständig, d.h. alle laufenden Prozesse des Spieles werden beendet und der Speicher wird freigeben.

		\id[ 60] Verlassen eines aktiven Spiels über das Pause-Menü.\\
		
		\textit{Erwartet:} Nach dem Klicken auf den Beenden-Button, des \gls{fa:Pausemenu} erscheint das \gls{fa:Hauptmenu}.

		\id[70] Transformieren des Knotens, sowohl im Challenge-Mode als auch im Creative-Mode.\\
		
		\textit{Erwartet:} Falls die Transformation gültig ist, wird die Kante entsprechend transformiert. Dies funktioniert, sowohl im Challenge-Modus als auch im Creative-Modus.

		\id[ 80] Kamerapostion verändern (bewegen, drehen und zoomen), sowohl im Challenge-Modus als auch im Creative-Modus.\\
		
		\textit{Erwartet:} Die Kameraposition verändert sich wie gewünscht in die vorgegebene Richtung. Dies funktioniert, sowohl im Challenge-Modus als auch im Creative-Modus.

		\id[ 90] Erfolgreiches Beenden einer Challenge.\\
		
		\textit{Erwartet:} Die Zeit wird gestoppt und der Abschlussbildschirm wird eingeblendet. Falls die Zeit für die Bestenliste ausgereicht hat, wird diese direkt eingetragen mit dem eingetragenen Spielernamen.

		\id[100] Speicherung eines Knotens den man im Creative-Mode erstellt hat und späteres Laden.\\
		
		\textit{Erwartet:} Ein Knoten wird in einer Datei im Dateiaustauschformat gespeichert. Wenn diese Datei geladen wird erhält man den vorher abgespeicherten Knoten zurück.
		
		\id[120] Installation des Spiels auf Windows Zielsystemen.\\
		
		\textit{Erwartet:} Installation ohne Problem und anschließende Lauffähigkeit des Spiels.

		\id[130] Restlose Deinstallation des Spiels von Windows Zielsystemen.\\
		
		\textit{Erwartet:} Deinstallation des Spiels ohne verbliebene Dateien.
		
	

	\end{ids}


\clearpage


%
% Testen der Robustheit bei falschen Eingaben.
%
\subsection{Negativtests}

\vspace{1em}

\begin{ids}{\gls{PTF}}

	\id[500] Importieren einer Datei die keinen gültigen Knoten enthält.\\
	
	\textit{Erwartet:} Das Spiel bricht das Importieren ab und meldet, dass diese Datei keinen gültigen Knoten enthält.
	
	\id[510] Erstellen einer Challenge mit zwei identischen Knoten.\\
	
	\textit{Erwartet:} Das Spiel überprüft die beiden Knoten auf Gleichheit. Wenn Gleichheit besteht wird die Erstellung der Challenge nicht durchgeführt, da beide Knoten identisch sind und somit die Challenge automatisch gewonnen wäre nach dem Start.
	
	\id[520] Versuch eine nicht gültigen Transformation durchzuführen.\\
	
	\textit{Erwartet:} Wenn die Transformation an einer nicht gültigen Stelle beendet werden soll, so wird die Transformation nur bis zur letzten gültigen Stelle. Falls es keine gültige Stelle gibt wird keine Transformation durchgeführt.
	
	\id[530] Versuch, ein Standard-Level zu löschen.\\
	
	\textit{Erwartet:} Der Spieler gibt bei ausgewähltem Standard-Level in der Übersicht den Befehl zu dessen Löschung ein. Standard-Levels können vom Spieler nicht gelöscht werden, es passiert nichts.

\end{ids}


\clearpage


%
% Testen von Extremfällen.
%
\subsection{Extremtests}

\vspace{1em}

	\begin{ids}{\gls{PTF}}
	
		\id[1000] Importieren einer sehr großen gültigen Knoten-Datei (mit mehreren hunderten Kanten).\\
		
		\textit{Erwartet:} Das Spiel kann den Knoten importieren und korrekt überprüfen sofern die Ressourcen des Systems ausreichend sind.\\
		
		\id[1010] Erstellen einer Challenge mit zwei sehr Großen Knoten (jeweils mit mehreren Hundert Kanten).\\
		
		\textit{Erwartet:} Die Challenge wird erstellt sofern die beiden Knoten nicht identisch sind und das System genug Ressourcen besitzt um die Knoten zu laden.
		
		\id[1020] Wahllose Tasteneingaben im Creative-Mode.\\
		
		\textit{Erwartet:} Sofern gedrückte Taste keine Funktionalität erfüllt wird nicht passieren. Falls diese Taste eine Funktion erfüllt wird nur diese ausgeführt.
	
	\end{ids}


