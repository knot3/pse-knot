% !TeX encoding = UTF-8
%
% Projekt- und Produkt-Umfang:
%


\chapter{Umfang}
\label{UF}~\\


%
% Klare Beschreibung der Ziele, welche zu erreichen sind.
%
\section{Ziele}
\label{UF:Ziele}

Das Spiel versetzt einen einzelnen Spieler in die Lage Knoten im dreidimensionalen Raum zu erstellen und zu modifizieren. Zwischen den Kanten der Knoten besteht die M{"o}glichkeit Fl{"a}chen einzusetzen und diese zu texturieren. Zudem wird dem Spieler erlaubt sich in verschiedenen Herausforderungen mit anderen Spielern zu messen.\\

% 
%
\subsection*{\underline{Pflicht-Kriterien}}

\begin{ids}{\gls{PUK}}


		\id[10] Spielmodus 1: \gls{fa:Creative}.
		
		\id[20] Spielmodus 2: \gls{fa:Challenge}.
		
		\id[30] Knoten{"u}berg{"a}nge m{"u}ssen eindeutig erkennbar sein.
		
		\id[40] Darstellung mit passenden 3D-Modellen an {"U}berg{"a}ngen.
		
		\id[50] Selektion und \gls{fa:Modifikation} von Kantenz{"u}gen.
		
		\id[60] {"U}bergehen unm{"o}glicher Zust{"a}nde, wenn m{"o}glich.
		
		\id[70] \gls{fa:Bestenliste} der besten Zeiten für ein Level
		
		\id[80] einfaches \gls{fa:Datenaustauschformat} f{"u}r die Levels
		
		\id[80] mindestens zehn eindeutige \gls{fa:Level} mit steigendem \\Schwierigkeitsgrad.
		
		\id[90] Eine intuitive Steuerung 
		
		\id[100] Ein sinnvolles \gls{fa:Undo} unterstützt den Spieler
		
		\id[110] gute automatische Kameraf{"u}hrung
		\id[120] Standard Sprache ist Englisch
		
	
		\id[130] Einfaches Speicherformat das lokal Austauschbar ist
		
		\id[140] Windows als Plattform muss unterst{"u}tzt werden

\end{ids}

~\\


\subsection*{\underline{Optionale Kriterien}}

\begin{ids}{\gls{OUK}}


\id[10] Begleitender Sound erg{"a}nzt das Spielerlebnis

\id[20] Der Einsatz von Hintergrundmusik

\id[30] Eine ver{"a}nderbare Tastaturbelegung

\id[40] Einf{"a}rbung von Kanten nach Spieler Pr{"a}ferenz

\id[50] zus{"a}tzliche Lokalisierung in Deutsch

\id[60] \gls{fa:Redo} welches vorangegangene Undo r{"u}ckg{"a}ngig macht

\id[70] optionale Fl{"a}chenerstellung zwischen benachbarten \\Kanten
ine Wiederholung möglich.

\id[90] Spielerbewertungen f{"u}r Knoten

\id[100] Durchschnittszeit des Bestehens einer Challenge

\id[110] \gls{fa:Eastereggs} k{"o}nnen gefunden werden

\id[120] Unterst{"u}tzende Tutorials die den Einstieg erleichtern
	
\id[130] Der Einsatz eines oder mehrerer \gls{fa:Shadereffekte}

\id[140] Der Einsatz von besonderen \gls{fa:Rendereffekten}

\id[150] Online-Austausch der Leveldaten

\id[160] 3D-Drucker kompatible Ausgabe der Leveldaten

\id[170] Linux als Plattform wird unterst{"u}tzt

\end{ids}


%
% Produktgrenzen starten auf einer eigenen Seite.
%
\clearpage


\section{Grenzen}
\label{UF:Grenzen}

~\\

\begin{ids}{\gls{AK}}

	
	\id[10] Das Spiel ist keine 3D-Modellierungssoftware.
	\id[20] Versionen f{"u}r mobile Ger{"a}te sind nicht geplant.
	\id[20] Außer Maus und Tastatur ist keine Unterst{"u}tzung durch weiter Eingabeger{"a}te,wie z.B.  ber{"u}hrungsempfindliche Bildschirme, geplant.
	\id[40]F{"u}rs Spielen wird keine Internetverbindung ben{"o}tigt. 
	\id[50] Ein Spiel beansprucht je nach Schwierigkeit einiges an Zeit und ist deswegen nicht zum Spielen f{"u}r Zwischendurch geeignet.
	\id[60] Das Spiel ist f{"u}r einen Spieler konzipiert.
	
\end{ids}


