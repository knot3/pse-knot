% !TeX encoding = UTF-8
%
% Interaktionsmodelle:
%


%
% Interaktionsmodelle starten auf eigener Seite.
%
\clearpage


\section{Interaktionsmodelle}
\label{NU:Interaktion}


\paragraph*{\underline{Pflicht:}}

\begin{ids}{\gls{PIM}}

	\id[10] In Menüs /PGO\_00XX/ und /PGO\_1030/ können Knöpfe und Auswahlflächen mit der Maus angeklickt werden.
	
	\id[20] Im Editor /PGO\_1010/ kann eine Kanten mit der Maus ausgewählt werden.
	
	\id[30] Durch drücken einer Taste kann /PIM\_20/ mehrfach ausgeführt werden, sodass die Auswahl aus mehreren Kanten besteht. Efüllt /PFA\_130/.
	
	\id[40] Durch gedrückt halten der Maustaste beginnen auf den Pfeilen in den Raumrichtungen /PFA\_355/ kann die durch /PIM\_20/ enstandene Auswahl in die entsprechende Raumrichtung verschoben werden. Erfüllt /PFA\_140/.
	
	\id[50] /PIM\_40 löst beim loslassen der Maustaste einen Transformationsschritt aus, für den es eine Visuelle Vorschau gibt /PFA\_310/.
	
	\id[60] Im Editor /PGO\_1010/ kann durch Tastendruck oder Mausklick auf eine Schaltfläche das Pausenmenü /PGO\_1030/ öffnen.
	
	\id[70] Im Editor /PGO\_1010/ ist Lineares verschieben des Kamerafokus in Blickrichtung und Senkrecht dazu ist über die Tastatur möglich.
	
	\id[80] Im Editor /PGO\_1010/ kann man den Kamerabstand zum Fokus (Zoom) über das Mausrad oder durch Tastendruck vergrößern und verkleinern.
	
	\id[90] Im Editor /PGO\_1010/ kann durch halten der rechten Maustaste die Kamera bei gleichbleibendem Fokus Kugelförmig darum bewegt werden.
	
	\id[100] Ist durch /PIM\_20/ eine Auswahl getroffen worden bleibt beim drehen /PIM\_90/ nicht der Fokus, sondern der Mittelpunkt der Auswahl an der gleichen Stelle.
	
	\id[110] Eine durch /PIM\_20/ getroffene Auswahl kann durch klick in den leren Raum aufgehoben werden.
	\id[120] Ein Tranformationsschritt /PIM\_50/ kann duch eine Tastenkombinatation oder einen Klick auch den "undo" Knopf rückgängig gemacht werden. Erfüllt /PFA\_160/.
	
	\id[130] Die Automatische Kameraführung /PK\_110/ dreht die Kamera und verschiebt den Fokus, sodass stets alle bisherigen Kanten und alle neu erstellten sichtbar sind und Platz für einfache transformationen bleibt.

\end{ids}


%
% Optionale Interaktionsmodelle starten auf einer eigenen Seite.
% nicht
%\clearpage


\paragraph*{\underline{Optional:}}

\begin{ids}{\gls{OIM}}

	\id[10] durch eine Tastenkombination oder einem Klick auf den "redo" Knopf kann eine rückgängig gemachte transformation /PIM\_120/ wieder hergestellt werden. Erfüllt /OFA\_190/.
	\id[20] Bei /PIM\_20/ ausgewählte Kanten kann durch Tastendruck ein auswahlmenü für personalisierte Farben aufgerufen werden.
	\id[30] Ein Klick auf eine Farbe im Farbauswahlmenü /OIM\_20/ färbt die ausgewählten Kanten in der Entsprechenden Farbe ein. Ein Klick in den leeren Raum schließt das Menü.
	\id[40] Im Editor /PGO\_1010/ ruft ein Klick auf den Knopf "render/export" oder ein passendes Icon das Menü für die Exportfunktionen /OGO\_1020/ auf.
	\id[50] Bei passender Auswahl durch /PIM\_20/ kann die eingeschlossene Fläche gefüllt werden. /OFA\_210/ 

\end{ids}


\subsection*{\underline{Visualisierungen:}}

\begin{landscape}

	\begin{figure}[ht]
	% ssetpagelength
	  \centering
	  \includesvg[width = 1.2\textwidth]{inGame}
	  \caption{Interaktionen während eines Spiels (allgemein)}
	\end{figure}

\end{landscape}

\clearpage
