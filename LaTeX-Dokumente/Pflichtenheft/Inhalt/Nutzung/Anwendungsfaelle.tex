% !TeX encoding = UTF-8
%
% Anwendungsfälle:
%


\section{Anwendungsf{"a}lle}
\label{NU:AF}~\\



\subsection*{\underline{Szenarien:}}

\begin{itemize}	
\item Der Spieler startet das Spiel und gelangt zum Hauptmenü. Dort wählt er den "creative" Modus und im darauf folgendem Menü das Erstellen eines neuen Knotens. Er gelangt in den Editor und beginnt dort den Knoten zu transformieren. Nach einigen transformationen öffnet er das Menü und speichert den Knoten. Danach fährt er mit der transformation fort. Zwischendurch ist er mit einigen Transformationsschritten unzufrieden und macht sie mir \textquotedblleft{}undo\textquotedblright rückgängig. Nach einigen weiteren transformationen ruft er wieder das Menü auf, speichert und beendet daraufhin den Editor mit einem klick auf den Menüeintrag \textquotedblleft{}quit\textquotedblright{}. Daraufhin landet er wieder im Hauptmenü. Dort beendet er das Spiel.

\item Der Spieler startet das Spiel und gelangt zum Hauptmenü. Dort wählt er den \textquotedblleft{}creative\textquotedblright Modus und danch \textquotedblleft{}load\textquotedblright um einen alten Speicherstand zu laden. In der Auswahlliste wählt er den gewünschten Knoten aus und lädt diesen. Er landet im Editor. Dort betrachtet er Knoten ein Zeitlang von allen Seiten, indem er mit Tastatur und Maus die Kamera um den Knoten herum bewegt und beendet danach wieder den Editor.

\item  Der Spieler startet das Spiel und gelangt zum Hauptmenü. Dort wählt er den \textquotedblleft{}challenge\textquotedblright Modus. Er sieht eine Liste mit allen verfügbaren Herausforderungen und einige Informationen zu diesen, dazu gehören die aktuellen Bestzeiten. Er sucht sich eine Herausforderung aus und Startet sie. Er landet im Editor mit einer zusätzlichen Ansicht für den Zielknoten. Er betrachtet diesen ausgiebig und beginnt danach mit der Transformation des vorgegebenen Knotens. Zum Zeitpunkt der ersten Transformation beginnt die Zeit zu laufen. Nach einigen Transformation stimmen die Knoten überein und die Zeit stoppt automatisch. Der Spieler war schnell und darf seinen Namen in die Bestenliste eintragen. Danach hat er die Möglichkeit die Herausforderung zu bewerten. Er gibt der Herausforderung eine gute Bewertung und landet danach wieder im Hauptmenü. Dort beendet er das Spiel.

\item Der Spieler startet das Spiel und gelangt zum Hauptmenü. Dort wählt er den "{}creative\textquotedblright Modus. Er möchte eine neue Herausforderung erstellen und wählt daher \textquotedblleft{}new challenge\textquotedblright. Danach wählt er zwei Knoten aus seinen Speicherständen aus. Einen für den Startknoten, einen als Zielknoten. Weiter unten gibt er der Herausforderung einen Namen und Speichert sie ab. Danach gelangt er in den Editor um als erster eine Zeit vorzulegen und die Herausforderung zu bestreiten. Der Spieler beendet dei Herausforderung aber ohne sie abzuschließen. Er kann die Herausforderung noch bewerten und landet dann wieder im Hauptmenü. Dort beendet er das Spiel.
\end{itemize}


%\begin{ids}{\gls{PAF}}
%
%	\id[ 1] ... \hfill\\
%	
%	(Ein Anwendungsfall) ...
%	
%	\id[10] ... \hfill\\
%	
%	(Ein Anwendungsfall) ...
%
%\end{ids}


%
% Optionale Anwendungsfälle starten auf einer eigenen Seite.
%
%\clearpage

%\subsection*{\underline{Optionale Anwendungsfälle:}}

%\begin{ids}{\gls{OAF}}

%	\id[ 1] ...
%	\id[10] ...

%\end{ids}
