% !TeX encoding = UTF-8
%
% Funktionale Anforderungen.
%


\subsection{Datenverwaltung}

%
% Hier in {} ein Bezeichner-Kürzel (z.B.: K, -> PFAK_...) eingeben:
%
\renewcommand{\K}{}
%
% HINWEIS: Bezeichner im Glossar definieren (oder Referenz auf Glossar entfernen).
%

~\\
Grundlegende Inhalte des Spieles werden abgespeichert und verwaltet.
Diese Inhalte können auch zwischen verschiedenen Systemen ausgetauscht werden.

~\\

%
% !
%
\subsubsection*{\underline{Pflicht:}}~\\

\begin{ids}{\gls{PFA\K}}

	\id[ 390] Der \gls{fa:Spieler} kann den Knoten im \gls{fa:Creative}-Mode abspeichern.
	\id[ 400]Speicherung einer \gls{fa:Bestenliste} für jedes Level.
	\id[ 410] Zwei Knoten können als Challenge abgespeichert werden in dem einer als \gls{fa:Referenzknoten} und der andere als \gls{fa:Ausgangsknoten} deklariert wird.
	\id[420] Es können nicht zwei identische Knoten als Challenge abgespeichert werden.
	\id[ 430] Import und Export von Knoten und Challenges mit Hilfe eines \gls{fa:Datenaustauschformat}.
	\id[ 440] Der Spieler kann einen \gls{fa:Spielername}n eingeben, welcher gespeichert wird.
	\id[ 450] Das Spiel speichert die Platzierung des Spielers  für das \gls{fa:Level} in einer \gls{fa:Bestenliste} unter dessen \gls{fa:Spielername}n.
	\id[ 460] Das Importieren ungültiger Knoten ist nicht möglich.
	\id[ 470] Es existiert eine Installationsdatei, mit der der Spieler das Spiel installieren kann. Das Spiel ist nach erfolgreicher Installation spielbar auf dem System des Spielers.
	\id[ 475] Das Spiel kann restlos deinstalliert werden. 
 	
 	
	
\end{ids}

~\\


%
% ?
%
\subsubsection*{\underline{Optional:}}~\\


\begin{ids}{\gls{OFA\K}}

	\id[ 480] Wenn ein Knoten abgespeichert wird, kann der Spieler ein Bild von seinem Knoten erstellen, welches als Vorschaubild verwendet wird.
	\id[ 490] Beim Verlassen des \gls{fa:Creative}-Mode über das \gls{fa:Pausemenu}  kann der \gls{fa:Spieler} auswählen ob er den aktuellen Knoten speichern möchte oder ohne Speichern den Modus verlassen will.
	\id[ 500] Das Spiel speichert Spieler-Bewertungen des Levels.
	\id[ 510] Die Durchschnittszeit beim Bestehen einer Challenge wird automatisch mitgespeichert.
	\id[ 520] Unterschiedliche Sprachen können integriert werden.
 	
 	
	
\end{ids}

~\\