% !TeX encoding = UTF-8
%
% Funktionale Anforderungen.
%


\subsection{Datenverwaltung}

%
% Hier in {} ein Bezeichner-Kürzel (z.B.: K, -> PFAK_...) eingeben:
%
\renewcommand{\K}{}
%
% HINWEIS: Bezeichner im Glossar definieren (oder Referenz auf Glossar entfernen).
%

~\\
Grundlegende Inhalte des Spieles werden abgespeichert und verwaltet.
Diese Inhalte können auch zwischen verschiedenen Systemen ausgetauscht werden.

~\\

%
% !
%
\subsubsection*{\underline{Pflicht:}}~\\

\begin{ids}{\gls{PFA\K}}

	\id[ 390] Der Spieler kann den Knoten im Creative-Modus abspeichern.
	\id[ 400]Speicherung einer Bestenliste für jedes Level.
	\id[ 410] Import und Export von Knoten und Challenges mit Hilfe eines Austauschdatei-Formates.
	\id[ 420] Der Spieler kann einen Spielernamen eingeben, welcher gespeichert wird.
	\id[ 430] Das Spiel speichert die Platzierung des Spielers  für das Level in einer Bestenliste unter dessen Spielernamen.
	\id[ 440] Das Importieren ungültiger Knoten ist nicht möglich.
	
 	
 	
	
\end{ids}

~\\


%
% ?
%
\subsubsection*{\underline{Optional:}}~\\


\begin{ids}{\gls{OFA\K}}

	\id[ 450] Wenn ein Knoten abgespeichert wird, kann der Spieler ein Bild von seinem Knoten erstellen, welches als Vorschaubild verwendet wird.
	\id[ 460] Beim Verlassen des Creative-Modus über das Pause-Menü  kann der Spieler auswählen ob er den aktuellen Knoten speichern möchte oder ohne Speichern den Modus verlassen will.
	\id[ 470] Das Spiel speichert Spieler-Bewertungen des Levels.
	\id[ 480] Die Durchschnittszeit beim Bestehen einer Challenge wird automatisch mitgespeichert.
	
 	
 	
	
\end{ids}

~\\