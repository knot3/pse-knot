% !TeX encoding = UTF-8
%
% Funktionale Anforderungen.
%


\subsection{Spielfunktionen}

%
% Hier in {} ein Bezeichner-Kürzel (z.B.: K, -> PFAK_...) eingeben:
%
\renewcommand{\K}{}
%
% HINWEIS: Bezeichner im Glossar definieren (oder Referenz auf Glossar entfernen).
%

~\\
Der \gls{fa:Spieler} kann durch verschiedene Funktionen mit dem Spiel interagieren. Er kann zum Beispiel die Kamera drehen und den \gls{fa:Knoten} verformen. 
\\

%
% !
%
\paragraph*{\underline{Pflicht:}}~\\

\begin{ids}{\gls{PFA\K}}

	\id[ 90] Beim Starten des \gls{fa:Creative}-Mode wird dem \gls{fa:Spieler} ein einfacher \gls{fa:Knoten} zum Transfomieren bereitgestellt.
	
 	\id[100] Der Spieler kann im \gls{fa:Creative}-Mode aus zwei erstellten Knoten eine Level für den \gls{fa:Challenge}-Modus erstellen.
 	
 	\id[ 110] Beim Starten des \gls{fa:Creative}-Mode wird dem Spieler ein einfacher Knoten  zum Transfomieren bereitgestellt.
 	
 	\id[ 120] Der Spieler kann im \gls{fa:Creative}-Mode aus zwei erstellten Knoten ein Level für den \gls{fa:Challenge}-Modus erstellen.
 	
 	\id[ 130] Die Kanten des \gls{fa:Knoten}s können vom Spieler vollständig oder teilweise ausgewählt werden.
 	
 	\id[ 140] Ausgewählte Kanten kann der Spieler in die Richtung der Koordinatenachsen transformieren.
 	
 	\id[ 150] Das Programm überprüft, ob eine \gls{fa:Transformation} gültig ist, falls nicht wird diese nicht ausgeführt.
 	
 	\id[ 160] Wenn der \gls{fa:Spieler} auf den \gls{fa:Undo}-Button klickt wird seine letzte \gls{fa:Transformation} rückgängig gemacht (beliebig wiederholbar). 
 	
 	\id[ 170] Im \gls{fa:Challenge}-Mode prüft das Programm den transformierten \gls{fa:Ausgangsknoten} auf mit dem \gls{fa:Referenzknoten}. Falls Gleichheit besteht wird die Zeit angehalten und der \gls{fa:Abschlussbildschirm} wird eingeblendet.
 	
 	\id[ 180] Der \gls{fa:Spieler} kann das Spiel jederzeit beenden.
	
\end{ids}


%
% ?
%
\paragraph*{\underline{Optional:}}~\\


\begin{ids}{\gls{OFA\K}}

	\id[ 190] Wenn der Spieler die \gls{fa:Undo}-Funktion genutzt hat, kann er seine letzten \gls{fa:Undo}-Aktionen durch Klicks auf den \gls{fa:Redo}-Button schrittweise rückgängig machen. {\gls{fa:Redo}} funktioniert nur so lange der Spieler keine Veränderung am Knoten vorgenommen hat.
	
 	\id[ 200] Kanten können vom \gls{fa:Spieler} eingefärbt werden.
 	
 	\id[ 210] Der Spieler kann im \gls{fa:Creative}-Mode vier Kanten auswählen, zwischen denen eine Fläche erstellt wird, sofern diese Kanten ein Rechteck bilden.
 	
 	\id[ 220] Falls der \gls{fa:Spieler} nur drei Kanten für eine Fläche auswählt, wird  die fehlende Kante durch eine \textquotedblleft \gls{fa:virtuellerKnoten} \textquotedblright~ersetzt.
 	
	\id [230] Nach erfolgreichem Beenden einer \gls{fa:Challenge} kann der \gls{fa:Spieler} die \gls{fa:Challenge} neu starten.
	
 	\id[ 240] Von einem Knoten kann der \gls{fa:Spieler} ein Bild erzeugen und abspeichern.
 	
 	\id[ 250] Beim Erzeugen eines Bildes kann der \gls{fa:Spieler} verschiedene \gls{fa:Rendereffekte} auswählen.
 	
 	\id[ 260] Ein erstellter \gls{fa:Knoten} kann in ein Format für 3D-Drucker exportiert werden.
 	
 	\id[ 270] Der Spieler kann \gls{fa:Easteregg}s finden.
 	
 	\id[ 275] Der Spieler kann ein \gls{fa:Tutorial} auswählen. Dort wird die Bedienung erläutert.
	
\end{ids}


