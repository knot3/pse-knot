% !TeX encoding = UTF-8
%
% Funktionale Anforderungen.
%


\subsection{Konfiguration}

%
% Hier in {} ein Bezeichner-Kürzel (z.B.: K, -> PFAK_...) eingeben:
%
\renewcommand{\K}{}
%
% HINWEIS: Bezeichner im Glossar definieren (oder Referenz auf Glossar entfernen).
%

~\\
Der Spieler kann verschiedene Eigenschaften des Programms einsehen und an seine Vorlieben anpassen.
\\

%
% !
%
\subsubsection*{\underline{Pflicht:}}~\\

\begin{ids}{\gls{PFA\K}}

	\id[ 10] Der \gls{fa:Spieler} kann Einstellungen zur Grafik und dem Ton im Menüpunkt Einstellungen des \gls{fa:Hauptmenu} bzw. \gls{fa:Pausemenu} vornehmen.
 	\id[ 20] Standard Grafikeinstellungen werden vom Programm vorgegeben.
 	\id[ 30] Durch Tastendruck ist das \gls{fa:Pausemenu} während des laufenden Spiels erreichbar.
 	\id[ 40] Der \gls{fa:Spieler} kann seinen \gls{fa:Spielername} ändern.
 	
 	
	
\end{ids}

~\\


%
% ?
%
\subsubsection*{\underline{Optional:}}~\\


\begin{ids}{\gls{OFA\K}}

	\id[ 50] In den Einstellungen kann der \gls{fa:Spieler} die Tastaturbelegung einsehen und  ändern.
 	\id[60] Wechsel zwischen verschiedenen Kameraeinstellungen (Automatische oder manuelle Kamera).
 	\id[ 70] Die Farben zum Einfärben von \gls{fa:Knoten} kann der \gls{fa:Spieler} selbständig festlegen. Die Anzahl ist aber beschränkt.
 	\id[ 80] Der \gls{fa:Spieler} kann die Sprache der grafischen Oberfläche des Spiels einsehen und ändern.
 	
 	
	
\end{ids}

~\\