% !TeX encoding = UTF-8
%
% Funktionale Anforderungen.
%


\subsection{Spiel-Konfiguration}

%
% Hier in {} ein Bezeichner-Kürzel (z.B.: K, -> PFAK_...) eingeben:
%
\renewcommand{\K}{}
%
% HINWEIS: Bezeichner im Glossar definieren (oder Referenz auf Glossar entfernen).
%

~\\
Der Spieler kann verschiedene Eigenschaften des Programms einsehen und an seine Vorlieben anpassen.\\

%
% !
%
\paragraph*{\underline{Pflicht:}}~\\

\begin{ids}{\gls{PFA\K}}

 	\id[ 10] Standard Grafikeinstellungen werden vom Programm vorgegeben.
 	
 	\id[ 15] Beim ersten Spielstart wird der Spieler aufgefordert einen Namen, wie er auch später in die Bestenliste einzutragen ist, einzugeben.
 		
 	\id[ 20] Der \gls{fa:Spieler} kann seinen \gls{fa:Spielername}n ändern.
 	 	
 	\id[ 27] In der Levelübersicht ist ein Nicht-Standard-Level per Tastaturbefehl löschbar.
	
\end{ids}


%
% ?
%
\paragraph*{\underline{Optional:}}~\\


\begin{ids}{\gls{OFA\K}}

 	\id[ 25] Der Ton kann durch einen Tastendruck abgestellt werden.

	\id[ 30] Der \gls{fa:Spieler} kann Einstellungen zur Grafik und dem Ton im Menüpunkt Einstellungen des \gls{fa:Hauptmenu} bzw. \gls{fa:Pausemenu} vornehmen.
	
	\id[ 40] Durch Tastendruck ist das \gls{fa:Pausemenu} während des laufenden Spiels erreichbar.	
	
	\id[ 50] In den Einstellungen kann der \gls{fa:Spieler} die Tastaturbelegung einsehen und  ändern.
	
 	\id[60] Wechsel zwischen verschiedenen Kameraeinstellungen (Automatische oder manuelle Kamera).
 	
 	\id[ 70] Die \gls{fa:Knoten}farben kann der \gls{fa:Spieler} selbständig festlegen. Deren Anzahl ist jedoch beschränkt.
 	
 	\id[ 80] Der \gls{fa:Spieler} kann die Sprache der grafischen Oberfläche des Spiels einsehen und ändern.
 	
\end{ids}


