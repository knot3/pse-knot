% !TeX encoding = UTF-8
%
% Funktionale Anforderungen.
%


\subsection{Darstellung}

%
% Hier in {} ein Bezeichner-Kürzel (z.B.: K, -> PFAK_...) eingeben:
%
\renewcommand{\K}{}
%
% HINWEIS: Bezeichner im Glossar definieren (oder Referenz auf Glossar entfernen).
%

~\\
Alle wichtigen Informationen werden dem Spieler visuell oder akustisch dargestellt. Die Atmosphäre wird durch die musikalische Untermalung verbessert.

\\

%
% !
%
\subsubsection*{\underline{Pflicht:}}~\\

\begin{ids}{\gls{PFA\K}}

	\id[ 280] Knoten bestehen aus Kanten, welche durch schmale längliche Zylnder dargestellt werden.
	\id[ 290] Die Kanten eines Knotens werden im dreidimensionalen Raum an Rasterpunkten ausgerichtet.
	\id[ 300] Bei Kreuzungen im Knoten weichen die Kanten sich gegenseitig aus, sodass der Kantenverlauf eindeutig bleibt.
	\id[ 310] Während des Transformieren wird die neu entstehenden Kanten transparent an der nächsten gültigen Position angezeigt. Sobald der Vorgang beendet ist wird die Kante an dieser Position ohne Transparenz dargestellt.
	\id[ 320] Der Spieler kann sich eine Übersicht zu allen Knoten, welche er im Creative-Modus erstellt hat anzeigen lassen, um daraus einen zur weiteren Bearbeitung auszuwählen.
	\id[ 330] Nach der Auswahl des Challenge-Modus kann der Spieler in einer Übersicht nach verschiedenen Kriterien ein Level auswählen.
	\id[ 340] Nach dem Start eines Levels sieht der Spieler beide Knoten (Ausgangsknoten und Referenzknoten). Sobald er die erste Veränderung am Ausgangsknoten vornimmt startet die Zeitmessung.
	\id[ 350] Ausgewählte Kanten werden visuell hervorgehoben.
	
	
 	
 	
	
\end{ids}

~\\


%
% ?
%
\subsubsection*{\underline{Optional:}}~\\


\begin{ids}{\gls{OFA\K}}

	\id[ 360] Die vom Spieler ausgew{"a}hlte Musik wird im Hintergrund wiederholt abgespielt.
	\id[ 370] Die Levelliste kann der Spieler sortieren lassen.
	\id[ 380] Die Levelliste kann der Spieler filtern lassen.
	
 	
 	
	
\end{ids}

~\\