% !TeX encoding = UTF-8
%
% Produkt-Anforderungen:
%


%%
%% Kürzel:
%%

%
% Kategorie.
%
\newcommand{\K}{}




\chapter{Anforderungen}
\label{AF}

Die Funktionen (Funktionale Anforderungen)\\und nicht funktionalen Anforderungen an das Spiel Knot³.\\


\section{Funktionen}
\label{AF:FA}

%
% Hier Funktionen-Kategorien eintragen:
%
%  1: Spiel-Konfiguration.
%  2: Spielfunktionen.
%  3: Darstellung.
%  4: Datenverwaltung.
%  5: 
%  6: 
%  7: 
%  8: 
%  9:  
%
% !TeX encoding = UTF-8
%
% Funktionale Anforderungen.
%


\subsection{Konfiguration}

%
% Hier in {} ein Bezeichner-Kürzel (z.B.: K, -> PFAK_...) eingeben:
%
\renewcommand{\K}{}
%
% HINWEIS: Bezeichner im Glossar definieren (oder Referenz auf Glossar entfernen).
%

~\\
Der Spieler kann verschiedene Eigenschaften des Programms einsehen und an seine Vorlieben anpassen.
\\

%
% !
%
\subsubsection*{\underline{Pflicht:}}~\\

\begin{ids}{\gls{PFA\K}}

	\id[ 10] Der Spieler kann Einstellungen zur Grafik und dem Ton im Menüpunkt Einstellungen des Hauptmenüs bzw. Pause-Menü vornehmen.
 	\id[ 20] Standard Grafikeinstellungen werden vom Programm vorgegeben.
 	\id[ 30] Durch Tastendruck ist das Pause-Menü während des laufenden Spiels erreichbar.
 	\id[ 40] Der Spieler kann seinen Spielernamen ändern.
 	
 	
	
\end{ids}

~\\


%
% ?
%
\subsubsection*{\underline{Optional:}}~\\


\begin{ids}{\gls{OFA\K}}

	\id[ 50] In den Einstellungen kann der Spieler die Tastaturbelegung einsehen und  ändern.
 	\id[60] Wechsel zwischen verschiedenen Kameraeinstellungen (Geführte oder frei-bewegliche Kamera).
 	\id[ 70] Die Farben zum Einfärben von Knoten kann der Spieler selbständig festlegen. Die Anzahl ist aber beschränkt.
 	\id[ 80] Der Spieler kann die Sprache der grafischen Oberfläche des Spiels einsehen und ändern.
 	
 	
	
\end{ids}

~\\
% !TeX encoding = UTF-8
%
% Funktionale Anforderungen.
%


\subsection{Spielfunktionen}

%
% Hier in {} ein Bezeichner-Kürzel (z.B.: K, -> PFAK_...) eingeben:
%
\renewcommand{\K}{}
%
% HINWEIS: Bezeichner im Glossar definieren (oder Referenz auf Glossar entfernen).
%

~\\
Der \gls{fa:Spieler} kann durch verschiedene Funktionen mit dem Spiel interagieren. Er kann zum Beispiel die Kamera drehen und den \gls{fa:Knoten} verformen. 
\\

%
% !
%
\paragraph*{\underline{Pflicht:}}~\\

\begin{ids}{\gls{PFA\K}}
	 	
 	\id[ 110] Beim Starten des \gls{fa:Creative}-Mode wird dem \gls{fa:Spieler} ein einfacher \gls{fa:Knoten} zum Transfomieren bereitgestellt.
 	
 	\id[ 120] Der Spieler kann im \gls{fa:Creative}-Mode aus zwei erstellten Knoten ein Level für den \gls{fa:Challenge}-Modus erstellen.
 	
 	\id[ 130] Die Kanten des \gls{fa:Knoten}s können vom Spieler vollständig oder teilweise ausgewählt werden.
 	
 	\id[ 140] Ausgewählte Kanten kann der Spieler in die Richtung der Koordinatenachsen transformieren.
 	
 	\id[ 150] Das Programm überprüft, ob eine \gls{fa:Transformation} gültig ist, falls nicht wird diese nicht ausgeführt.
 	
 	\id[ 160] Wenn der \gls{fa:Spieler} die \gls{fa:Undo}-Aktion auslöst wird seine letzte \gls{fa:Transformation} rückgängig gemacht (beliebig wiederholbar). 
 	
 	\id[ 170] Im \gls{fa:Challenge}-Mode prüft das Programm den transformierten \gls{fa:Ausgangsknoten} auf mit dem \gls{fa:Referenzknoten}. Falls Gleichheit besteht wird die Zeit angehalten und der \gls{fa:Abschlussbildschirm} wird eingeblendet.
 	
 	\id[ 180] Der \gls{fa:Spieler} kann das Spiel jederzeit beenden.
	
\end{ids}


%
% ?
%
\paragraph*{\underline{Optional:}}~\\


\begin{ids}{\gls{OFA\K}}

	\id[ 190] Wenn der Spieler die \gls{fa:Undo}-Funktion genutzt hat, kann er seine letzten \gls{fa:Undo}-Aktionen durch nutzen der \gls{fa:Redo}-Funktion schrittweise rückgängig machen. {\gls{fa:Redo}} funktioniert nur so lange der Spieler keine Veränderung am Knoten vorgenommen hat.
	
 	\id[ 200] Kanten können vom \gls{fa:Spieler} eingefärbt werden.
 	
 	\id[ 210] Der Spieler kann im \gls{fa:Creative}-Mode vier Kanten auswählen, zwischen denen eine Fläche erstellt wird, sofern diese Kanten ein Rechteck bilden.
 	
 	\id[ 220] Falls der \gls{fa:Spieler} nur drei Kanten für eine Fläche auswählt, wird  die fehlende Kante durch eine \textquotedblleft \gls{fa:virtuellerKnoten} \textquotedblright~ersetzt.
 	
	\id [230] Nach erfolgreichem Beenden einer \gls{fa:Challenge} kann der \gls{fa:Spieler} die \gls{fa:Challenge} neu starten.
	
 	\id[ 240] Von einem Knoten kann der \gls{fa:Spieler} ein Bild erzeugen und abspeichern.
 	
 	\id[ 250] Beim Erzeugen eines Bildes kann der \gls{fa:Spieler} verschiedene \gls{fa:Rendereffekte} auswählen.
 	
 	\id[ 260] Ein erstellter \gls{fa:Knoten} kann in ein Format für 3D-Drucker exportiert werden.
 	
 	\id[ 270] Der Spieler kann \gls{fa:Easteregg}s finden.
 	
 	\id[ 275] Der Spieler kann ein \gls{fa:Tutorial} auswählen. Dort wird die Bedienung erläutert.
	
\end{ids}



% !TeX encoding = UTF-8
%
% Funktionale Anforderungen.
%


\subsection{Darstellung}

%
% Hier in {} ein Bezeichner-Kürzel (z.B.: K, -> PFAK_...) eingeben:
%
\renewcommand{\K}{}
%
% HINWEIS: Bezeichner im Glossar definieren (oder Referenz auf Glossar entfernen).
%

~\\
Alle wichtigen Informationen werden dem \gls{fa:Spieler} visuell oder akustisch dargestellt. Die Atmosphäre wird durch die musikalische Untermalung verbessert.

~\\

%
% !
%
\subsubsection*{\underline{Pflicht:}}~\\

\begin{ids}{\gls{PFA\K}}

	\id[ 280] {\gls{fa:Knoten}} bestehen aus Kanten, welche durch schmale längliche Zylnder dargestellt werden.
	\id[ 290] Die Kanten eines \gls{fa:Knoten}s werden im dreidimensionalen Raum an \gls{fa:Rasterpunkt} ausgerichtet.
	\id[ 300] Bei Kreuzungen im \gls{fa:Knoten} weichen die Kanten sich gegenseitig aus, sodass der Kantenverlauf eindeutig bleibt.
	\id[ 310] Während des Transformieren wird die neu entstehenden Kanten transparent an der nächsten gültigen Position angezeigt. Sobald der Vorgang beendet ist wird die Kante an dieser Position ohne Transparenz dargestellt.
	\id[ 320] Der \gls{fa:Spieler} kann sich eine Übersicht zu allen Knoten, welche er im \gls{fa:Creative}-Mode erstellt hat anzeigen lassen, um daraus einen zur weiteren Bearbeitung auszuwählen.
	\id[ 330] Nach der Auswahl des \gls{fa:Challenge}-Mode kann der \gls{fa:Spieler} in einer Übersicht nach verschiedenen Kriterien ein Level auswählen.
	\id[ 340] Nach dem Start eines \gls{fa:Level}s sieht der Spieler beide \gls{fa:Knoten} (\gls{fa:Ausgangsknoten} und \gls{fa:Referenzknoten}). Sobald er die erste Veränderung am \gls{fa:Ausgangsknoten} vornimmt startet die Zeitmessung.
	\id[ 350] Ausgewählte Kanten werden visuell hervorgehoben.
	\id[ 355] An ausgew{"a}hlten Kanten werden Pfeile parallel zu der Richtung der Koordinatenachsen angezeigt, in welche eine gültige \gls{fa:Transformation} m{"o}glich w{"a}re.
	
	
 	
 	
	
\end{ids}

~\\


%
% ?
%
\subsubsection*{\underline{Optional:}}~\\


\begin{ids}{\gls{OFA\K}}

	\id[ 360] Die vom Spieler ausgew{"a}hlte Musik wird im Hintergrund wiederholt abgespielt.
	\id[ 370] Die Levelliste kann der Spieler sortieren lassen.
	\id[ 380] Die Levelliste kann der Spieler filtern lassen.
	
 	
 	
	
\end{ids}

~\\
% !TeX encoding = UTF-8
%
% Funktionale Anforderungen.
%


\subsection{Datenverwaltung}

%
% Hier in {} ein Bezeichner-Kürzel (z.B.: K, -> PFAK_...) eingeben:
%
\renewcommand{\K}{}
%
% HINWEIS: Bezeichner im Glossar definieren (oder Referenz auf Glossar entfernen).
%

~\\
Grundlegende Inhalte des Spieles werden abgespeichert und verwaltet.
Diese Inhalte können auch zwischen verschiedenen Systemen ausgetauscht werden.

~\\

%
% !
%
\subsubsection*{\underline{Pflicht:}}~\\

\begin{ids}{\gls{PFA\K}}

	\id[ 390] Der \gls{fa:Spieler} kann den Knoten im \gls{fa:Creative}-Mode abspeichern.
	\id[ 400]Speicherung einer \gls{fa:Bestenliste} für jedes Level.
	\id[ 410] Zwei Knoten können als Challenge abgespeichert werden in dem einer als \gls{fa:Referenzknoten} und der andere als \gls{fa:Ausgangsknoten} deklariert wird.
	\id[420] Es können nicht zwei identische Knoten als Challenge abgespeichert werden.
	\id[ 430] Import und Export von Knoten und Challenges mit Hilfe eines \gls{fa:Datenaustauschformat}.
	\id[ 440] Der Spieler kann einen \gls{fa:Spielername}n eingeben, welcher gespeichert wird.
	\id[ 450] Das Spiel speichert die Platzierung des Spielers  für das \gls{fa:Level} in einer \gls{fa:Bestenliste} unter dessen \gls{fa:Spielername}n.
	\id[ 460] Das Importieren ungültiger Knoten ist nicht möglich.
	\id[ 470] Es existiert eine Installationsdatei, mit der der Spieler das Spiel installieren kann. Das Spiel ist nach erfolgreicher Installation spielbar auf dem System des Spielers.
	\id[ 475] Das Spiel kann restlos deinstalliert werden. 
 	
 	
	
\end{ids}

~\\


%
% ?
%
\subsubsection*{\underline{Optional:}}~\\


\begin{ids}{\gls{OFA\K}}

	\id[ 480] Wenn ein Knoten abgespeichert wird, kann der Spieler ein Bild von seinem Knoten erstellen, welches als Vorschaubild verwendet wird.
	\id[ 490] Beim Verlassen des \gls{fa:Creative}-Mode über das \gls{fa:Pausemenu}  kann der \gls{fa:Spieler} auswählen ob er den aktuellen Knoten speichern möchte oder ohne Speichern den Modus verlassen will.
	\id[ 500] Das Spiel speichert Spieler-Bewertungen des Levels.
	\id[ 510] Die Durchschnittszeit beim Bestehen einer Challenge wird automatisch mitgespeichert.
	\id[ 520] Unterschiedliche Sprachen können integriert werden.
 	
 	
	
\end{ids}

~\\
%% !TeX encoding = UTF-8
%
% Funktionale Anforderungen.
%


\subsection{(Name der Funktionen-Kategorie) ...}

~\\
(Beschreibung der Funktionen-Kategorie) ...
\\

%
% !
%
\subsubsection*{\underline{Pflicht:}}~\\

\begin{reqs}{PF}

	\req[ 1] ...
 	\req[10] ...
	
\end{reqs}

~\\


%
% ?
%
\subsubsection*{\underline{Optional:}}~\\


\begin{reqs}{OF}

	\req[ 11] ...
 	\req[100] ...
	
\end{reqs}

~\\
%% !TeX encoding = UTF-8
%
% Funktionale Anforderungen.
%


\subsection{(Name der Funktionen-Kategorie) ...}

~\\
(Beschreibung der Funktionen-Kategorie) ...
\\

%
% !
%
\subsubsection*{\underline{Pflicht:}}~\\

\begin{reqs}{PF}

	\req[ 1] ...
 	\req[10] ...
	
\end{reqs}

~\\


%
% ?
%
\subsubsection*{\underline{Optional:}}~\\


\begin{reqs}{OF}

	\req[ 11] ...
 	\req[100] ...
	
\end{reqs}

~\\
%% !TeX encoding = UTF-8
%
% Funktionale Anforderungen.
%


\subsection{(Name der Funktionen-Kategorie) ...}

~\\
(Beschreibung der Funktionen-Kategorie) ...
\\

%
% !
%
\subsubsection*{\underline{Pflicht:}}~\\

\begin{reqs}{PF}

	\req[ 1] ...
 	\req[10] ...
	
\end{reqs}

~\\


%
% ?
%
\subsubsection*{\underline{Optional:}}~\\


\begin{reqs}{OF}

	\req[ 11] ...
 	\req[100] ...
	
\end{reqs}

~\\
%% !TeX encoding = UTF-8
%
% Funktionale Anforderungen.
%


\subsection{(Name der Funktionen-Kategorie) ...}

~\\
(Beschreibung der Funktionen-Kategorie) ...
\\

%
% !
%
\subsubsection*{\underline{Pflicht:}}~\\

\begin{reqs}{PF}

	\req[ 1] ...
 	\req[10] ...
	
\end{reqs}

~\\


%
% ?
%
\subsubsection*{\underline{Optional:}}~\\


\begin{reqs}{OF}

	\req[ 11] ...
 	\req[100] ...
	
\end{reqs}

~\\
%% !TeX encoding = UTF-8
%
% Funktionale Anforderungen.
%


\subsection{(Name der Funktionen-Kategorie) ...}

~\\
(Beschreibung der Funktionen-Kategorie) ...
\\

%
% !
%
\subsubsection*{\underline{Pflicht:}}~\\

\begin{reqs}{PF}

	\req[ 1] ...
 	\req[10] ...
	
\end{reqs}

~\\


%
% ?
%
\subsubsection*{\underline{Optional:}}~\\


\begin{reqs}{OF}

	\req[ 11] ...
 	\req[100] ...
	
\end{reqs}

~\\


%
% Nicht-funktionale Anforderungen starten auf einer eigenen Seite.
%
\clearpage


\section{Nicht-Funktionale Anforderungen}
\label{AF:NFA}~\\

%
% Hier Nicht-Funktionale Kategorien eintragen:
%
%  1: 
%  2: 
%  3: 
%  4: 
%  5: 
%  6: 
%  7: 
%  8: 
%  9:  
%
% !TeX encoding = UTF-8
%
% Nicht-Funktionale Anforderungen.
%

\subsection{(Name der Nicht-Funktionalen Kategorie) ...}

~\\
(Beschreibung der Nicht-Funktionalen Kategorie) ...
\\

%
% Pflicht.
%
\subsubsection*{\underline{Pflicht:}}~\\

\begin{reqs}{PNF} 

	\req[ 1] ...
 	\req[10] ...
	
\end{reqs}

~\\


%
% Optional.
%
\subsubsection*{\underline{Optional:}}~\\


\begin{reqs}{ONF}

	\req[ 11] ...
 	\req[100] ...
	
\end{reqs}

~\\
% !TeX encoding = UTF-8
%
% Nicht-Funktionale Anforderungen.
%

\subsection{(Name der Nicht-Funktionalen Kategorie) ...}

~\\
(Beschreibung der Nicht-Funktionalen Kategorie) ...
\\

%
% Pflicht.
%
\subsubsection*{\underline{Pflicht:}}~\\

\begin{reqs}{PF} 

	\req[ 1] ...
 	\req[10] ...
	
\end{reqs}

~\\


%
% Optional.
%
\subsubsection*{\underline{Optional:}}~\\


\begin{reqs}{OF}

	\req[ 11] ...
 	\req[100] ...
	
\end{reqs}

~\\
% !TeX encoding = UTF-8
%
% Nicht-Funktionale Anforderungen.
%

\subsection{(Name der Nicht-Funktionalen Kategorie) ...}

~\\
(Beschreibung der Nicht-Funktionalen Kategorie) ...
\\

%
% Pflicht.
%
\subsubsection*{\underline{Pflicht:}}~\\

\begin{reqs}{PF} 

	\req[ 1] ...
 	\req[10] ...
	
\end{reqs}

~\\


%
% Optional.
%
\subsubsection*{\underline{Optional:}}~\\


\begin{reqs}{OF}

	\req[ 11] ...
 	\req[100] ...
	
\end{reqs}

~\\
%% !TeX encoding = UTF-8
%
% Nicht-Funktionale Anforderungen.
%

\subsection{(Name der Nicht-Funktionalen Kategorie) ...}

~\\
(Beschreibung der Nicht-Funktionalen Kategorie) ...
\\

%
% Pflicht.
%
\subsubsection*{\underline{Pflicht:}}~\\

\begin{reqs}{PF} 

	\req[ 1] ...
 	\req[10] ...
	
\end{reqs}

~\\


%
% Optional.
%
\subsubsection*{\underline{Optional:}}~\\


\begin{reqs}{OF}

	\req[ 11] ...
 	\req[100] ...
	
\end{reqs}

~\\
%% !TeX encoding = UTF-8
%
% Nicht-Funktionale Anforderungen.
%

\subsection{(Name der Nicht-Funktionalen Kategorie) ...}

~\\
(Beschreibung der Nicht-Funktionalen Kategorie) ...
\\

%
% Pflicht.
%
\subsubsection*{\underline{Pflicht:}}~\\

\begin{reqs}{PF} 

	\req[ 1] ...
 	\req[10] ...
	
\end{reqs}

~\\


%
% Optional.
%
\subsubsection*{\underline{Optional:}}~\\


\begin{reqs}{OF}

	\req[ 11] ...
 	\req[100] ...
	
\end{reqs}

~\\
%% !TeX encoding = UTF-8
%
% Nicht-Funktionale Anforderungen.
%

\subsection{(Name der Nicht-Funktionalen Kategorie) ...}

~\\
(Beschreibung der Nicht-Funktionalen Kategorie) ...
\\

%
% Pflicht.
%
\subsubsection*{\underline{Pflicht:}}~\\

\begin{reqs}{PF} 

	\req[ 1] ...
 	\req[10] ...
	
\end{reqs}

~\\


%
% Optional.
%
\subsubsection*{\underline{Optional:}}~\\


\begin{reqs}{OF}

	\req[ 11] ...
 	\req[100] ...
	
\end{reqs}

~\\
%% !TeX encoding = UTF-8
%
% Nicht-Funktionale Anforderungen.
%

\subsection{(Name der Nicht-Funktionalen Kategorie) ...}

~\\
(Beschreibung der Nicht-Funktionalen Kategorie) ...
\\

%
% Pflicht.
%
\subsubsection*{\underline{Pflicht:}}~\\

\begin{reqs}{PF} 

	\req[ 1] ...
 	\req[10] ...
	
\end{reqs}

~\\


%
% Optional.
%
\subsubsection*{\underline{Optional:}}~\\


\begin{reqs}{OF}

	\req[ 11] ...
 	\req[100] ...
	
\end{reqs}

~\\
%% !TeX encoding = UTF-8
%
% Nicht-Funktionale Anforderungen.
%

\subsection{(Name der Nicht-Funktionalen Kategorie) ...}

~\\
(Beschreibung der Nicht-Funktionalen Kategorie) ...
\\

%
% Pflicht.
%
\subsubsection*{\underline{Pflicht:}}~\\

\begin{reqs}{PF} 

	\req[ 1] ...
 	\req[10] ...
	
\end{reqs}

~\\


%
% Optional.
%
\subsubsection*{\underline{Optional:}}~\\


\begin{reqs}{OF}

	\req[ 11] ...
 	\req[100] ...
	
\end{reqs}

~\\
%% !TeX encoding = UTF-8
%
% Nicht-Funktionale Anforderungen.
%

\subsection{(Name der Nicht-Funktionalen Kategorie) ...}

~\\
(Beschreibung der Nicht-Funktionalen Kategorie) ...
\\

%
% Pflicht.
%
\subsubsection*{\underline{Pflicht:}}~\\

\begin{reqs}{PF} 

	\req[ 1] ...
 	\req[10] ...
	
\end{reqs}

~\\


%
% Optional.
%
\subsubsection*{\underline{Optional:}}~\\


\begin{reqs}{OF}

	\req[ 11] ...
 	\req[100] ...
	
\end{reqs}

~\\


