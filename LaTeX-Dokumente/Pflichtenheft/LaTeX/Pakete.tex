% !TeX encoding = UTF-8
%
% LaTeX-Pakete.
%

%\usepackage{makeidx}


%
% Neue Deutsche Rechtschreibung.
%
\usepackage[ngerman]{babel}

%
% Automatische Kodierung vieler Sonderzeichen.
%
\usepackage[utf8x]{inputenc}

%
% Z.B. Symbol: Schwarzer Stern.
%
\usepackage{amssymb}

%
%
%
%\usepackage{tikz}

%
%
%
\usepackage[T1]{fontenc}

%
% Schriftpaket für "Latin Modern"
%
\usepackage{lmodern}

%
% Schriftfarbe.
%
\usepackage{xcolor}

%
% Tabellen über mehrere Seiten.
%
\usepackage{longtable}

%
% Abstand links, rechts, vor, nach Abschnitten (engl. "section")
%
\usepackage{titlesec}
\titlespacing{\section}{0pt}{5.5ex plus 1ex minus .2ex}{4.3ex plus .2ex}
\titlespacing*{\subsection}{0pt}{5.5ex plus 1ex minus .2ex}{4.3ex plus .2ex}

%
% Abstände.
%
\usepackage{vmargin} % ToDo: durch geometry ersetzen
\setpapersize{A4} % besser oben?
\setmarginsrb{3cm}{1.5cm}{3cm}{1cm}{6mm}{6mm}{5mm}{15mm}

%
%
%
\usepackage{graphicx}

%
% Skalierbare Vektorgrafiken (SVG'en) einbinden.
%
% HINWEIS: Inkscape, pdflatex --shell-escape nötig.
%
\usepackage{svg}

%
% Verlinkungen verfügbar und "klickbar" machen.
%
\usepackage[hidelinks]{hyperref}

%
% Glossare.
%
% HINWEIS: perl nötig.
%
\usepackage[nomain]{glossaries}
\renewcommand*{\glspostdescription}{} % Kein automatischer Setzpunkt nach einem Eintrag.
\newglossary{FA}{fai}{fao}{Fachausdrücke}
\newglossary{AK}{aki}{ako}{Abkürzungen}
\makeglossaries
%% Glossar-Einträge im Voraus bekannt machen,
%% damit sie im Folgenden verwendbar sind.
\loadglsentries[FA]{Inhalt/Verzeichnisse/Glossare/Fachausdruecke}
\loadglsentries[AK]{Inhalt/Verzeichnisse/Glossare/Abkuerzungen}
%
% HINWEIS:
%
% Begriffe, die im Glossar auftauchen müssen auch im
% Text auftauchen und durch gls{Begriff} markiert worden
% sein, damit sie im Verzeichnis gelistet werden!
%

