\chapter{Aufbau}

\newcommand{\class}[1]{\textcolor{BlueViolet}{\sloppy{#1}}}
\newcommand{\interface}[1]{\textcolor{OliveGreen}{\mbox{\sloppy{#1}}}}
\newcommand{\xna}[1]{\textcolor{Gray}{\sloppy{#1}}}



\section{Architektur}

Die grundlegende Architektur des Spiels basiert auf der Spielkomponenten-Infrastruktur des XNA-Framework, die mit Spielzuständen kombiniert wird. Die abstrakten Klassen \class{GameStateComponent} und \class{DrawableGameStateComponent} erben von den von XNA bereitgestellten Klassen \xna{GameComponent} und \xna{DrawableGameComponent} implementieren zusätzlich die Schnittstelle \interface{IGameStateComponent}. Sie unterscheiden sich von den XNA-Basisklassen dadurch, dass sie immer eine Referenz auf einen bestimmten Spielzustand halten und nur in Kombination mit diesem zu verwenden sind.
\\\\
Die Spielzustände erben von der abstrakten Basisklasse \class{GameScreen} und halten eine Liste von \interface{IGameStateComponent}-Objekten. Wird ein Spielzustand aktiviert, indem von einem anderen Spielzustand aus zu ihm gewechselt wird oder indem er der Startzustand ist, dann weist er seine Liste von \interface{IGameStateComponent}-Objekten dem \xna{Components}-Attribut der \class{Game}-Klasse zu, die von der vom XNA-Framework bereitgestellten abstrakten Klasse \xna{Game} erbt. So ist zu jedem Zeitpunkt während der Laufzeit des Spiels ein Spielzustand aktiv, der die aktuelle Liste von Spielkomponenten verwaltet.

\section{Klassendiagramm}

\section{Verwendete Entwurfsmuster}