\chapter{Einleitung}

% Hinweis: Die Konstanten \Count ... werden automatisch aktualisiert und sollten nach einem Aufruf von CSharpUML.exe
% auf dem neuesten Stand sein!

Das Knobel- und Konstruktionsspiel Knot$^3$ wurde vom IBDS Dachsbacher in Auftrag gegeben. Die Idee und das Konzept entstanden am Institut für Computergrafik in Zusammenarbeit mit der Hochschule für Gestaltung (HfG) in Karlsruhe und werden im Rahmen des Softwareprojekts PSE von Studenten des Karlsruher Instituts für Technologie umgesetzt. Knot$^3$ wird wie im Pflichtenheft spezifiziert ausgearbeitet.
\\\\
Als Entwurfsumgebung wurde Microsoft Visual Studio 2013 verwendet. Die Entwicklung soll in der Sprache C-Sharp, aufbauend auf dem Microsoft XNA-Framework, erfolgen.
\\\\
Der Entwurf besteht aus insgesamt \CountClasses~Klassen, \CountInterfaces~Schnittstellen und \CountEnums~Enumerationen. Im UML-Diagramm befinden sich etwa \CountAll~Elemente.