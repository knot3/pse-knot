\chapter{Abläufe}

\section{Sequenzdiagramme}

\subsection{Challengeende}

Knot ruft bei Kantenänderungen eine Methode auf, die ihm beim Hinzufügen zum \class{ChallengeModeScreen} zugewiesen wurde. Diese prüft die beiden Knoten auf Gleichheit, indem sie die Knoten sich vergleichen lässt. Sind beide \class{Knot}en gleich, wird die benötigte Zeit ermittelt und ein neuer \class{Dialog} erstellt, der die Zeit anzeigt und nach einem Benutzernamen fragt.
\newline
Dazu wird aus Options.Default der Benutzernamen abgefragt und als Standardwert eingetragen. Wird ein neuer Benutzername eingegeben, wird dieser auch in Options.Default geändert. Mit Enter bestätigt wird eine zugewiesene Methode aufgerufen, die Benutzername und Zeit \class{Challenge} mit AddToHighscore(name, time) übergibt, was die Werte richtig einordnet und gegebenenfalls (wenn die Zeit gut genug ist) speichert.
\newline
Daraufhin öffnet diese Methode einen neuen \class{Dialog}, der die besten 10, soweit vorhanden, anzeigt und 2 Buttons 
von denen einer die \class{Challenge} neu startet und ein anderer den StartScreen aufruft.

\subsection{Screenwechsel vom CreativeMainScreen in den CreativeLoadScreen}

Wenn man im \class{CreativeMainScreen} auf den Button "Load" klickt, wird der Wechsel zu \class{CreativeLoadScreen} eingeleitet.
Beim nächsten Update()-Call von XNA wird dann der PostProcessingEffect von \class{CreativeLoadScreen} für ein Überblenden der Menüs z.B. auf \class{FadeEffect} gesetzt und der \class{CreativeMainScreen} deaktiviert, d.h. alle \interface{IGameScreenComponent}s von \class{CreativeMainScreen} werden aus der Liste aller \interface{IGameScreenComponent} gelöscht. Dann wird \class{CreativeLoadScreen} aktiviert und dessen \interface{IGameScreenComponent} in die XNA \xna{IGameComponent}-Liste eingetragen.
\newline
Daraufhin durchsucht \class{CreativeLoadScreen} das Speicherstand-Verzeichnis und lädt die Metadaten der Speicherstände aller zwischengespeicherten Knoten.
\newline
Noch nicht zwischengespeicherte Knoten werden auf Gültigkeit überprüft und wenn für Gültig befunden in den Zwischenspeicher geschrieben als Hash.
\newline
Anschließend werden die gefundenen gültigen Speicherstände in ein Menü eingetragen, welches dann beim nächsten Draw()-Call von XNA auf den Bildschirm gezeichnet wird.
\newline
Dieses Menü ist ein \class{VerticalMenu}, welches einen Button für jeden gültigen gefundenen Spielstand enthält. Durch einen Klick auf den jeweiligen Button startet man mit dem Spielstand im \class{CreativeModeScreen}.
