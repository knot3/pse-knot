\documentclass{report}
\sloppy
\usepackage[usenames,dvipsnames]{color}
\usepackage[top=1.5in, bottom=1in, left=1in, right=1in]{geometry}
\usepackage[utf8]{inputenc}
\usepackage[ngerman]{babel}
\usepackage{microtype}
\usepackage{enumitem}
\usepackage{graphicx}
\usepackage{wrapfig}
\usepackage{listings}
\usepackage{hyperref}


%
% Glossare.
%
% HINWEIS: perl nötig.
%
%\usepackage[nomain,toc,section, sanitize={description=false}]{glossaries}
\usepackage[nomain]{glossaries}
\renewcommand*{\glspostdescription}{} % Kein automatischer Satzpunkt nach einem Eintrag.
\newglossary{FA}{fai}{fao}{Fachausdrücke}
\newglossary{AK}{aki}{ako}{Abkürzungen}
\makeglossaries
%% Glossar-Einträge im Voraus bekannt machen,
%% damit sie im Folgenden verwendbar sind.
\loadglsentries[FA]{Fachausdruecke}
\loadglsentries[AK]{Abkuerzungen}

%
% HINWEIS:
%
% Begriffe, die im Glossar auftauchen müssen auch im
% Text auftauchen und durch gls{Begriff} markiert worden
% sein, damit sie im Verzeichnis gelistet werden!
%







\begin{document}

\title{\textbf{\Huge{ENTWURFSDOKUMENT}}\\\Large{(V. 1.0)}~\\~\\~\\
		\textbf{\Large{KNOT$^3$}}\\
		PSE WS 2013/14}
\author{\Large{Auftraggeber:}\\
        \Large{Karlsruher Institut für Technologie}\\
        \Large{Institut für Betriebs- und Dialogsysteme}\\
        \Large{Prof. Dr.-Ing. C. Dachsbacher}\\~\\
        \Large{Betreuer:}\\
        \Large{Dipl.-Inf. Thorsten Schmidt}\\
        \Large{Dipl.-Inf. M. Retzlaff}\\~\\
        \Large{Auftragnehmer:}\\
        \Large{Tobias Schulz, Maximilian Reuter, Pascal                    Knodel,}\\
	 	\Large{Gerd Augsburg, Christina Erler, Daniel Warzel}~\\~\\}
\date{\today}

\maketitle

\tableofcontents

\chapter{Einleitung}

Das Knobel- und Konstruktionsspiel Knot$^3$, welches im Auftrag des IBDS Dachsbacher ausgearbeitet und wie im Pflichtenheft spezifiziert angefertigt wird.
\chapter{Aufbau}

\newcommand{\class}[1]{\textcolor{BlueViolet}{\sloppy{#1}}}
\newcommand{\interface}[1]{\textcolor{OliveGreen}{\mbox{\sloppy{#1}}}}
\newcommand{\xna}[1]{\textcolor{Gray}{\sloppy{#1}}}



\section{Architektur}

Die grundlegende Architektur des Spiels basiert auf der Spielkomponenten-Infrastruktur des XNA-Framework, die mit Spielzuständen kombiniert wird. Die abstrakten Klassen \class{GameStateComponent} und \class{DrawableGameStateComponent} erben von den von XNA bereitgestellten Klassen \xna{GameComponent} und \xna{DrawableGameComponent} implementieren zusätzlich die Schnittstelle \interface{IGameStateComponent}. Sie unterscheiden sich von den XNA-Basisklassen dadurch, dass sie immer eine Referenz auf einen bestimmten Spielzustand halten und nur in Kombination mit diesem zu verwenden sind.
\\\\
Die Spielzustände erben von der abstrakten Basisklasse \class{GameScreen} und halten eine Liste von \interface{IGameStateComponent}-Objekten. Wird ein Spielzustand aktiviert, indem von einem anderen Spielzustand aus zu ihm gewechselt wird oder indem er der Startzustand ist, dann weist er seine Liste von \interface{IGameStateComponent}-Objekten dem \xna{Components}-Attribut der \class{Game}-Klasse zu, die von der vom XNA-Framework bereitgestellten abstrakten Klasse \xna{Game} erbt. So ist zu jedem Zeitpunkt während der Laufzeit des Spiels ein Spielzustand aktiv, der die aktuelle Liste von Spielkomponenten verwaltet.
\\\\
Die Spielkomponenten, die nicht gezeichnet werden und nur auf Eingaben reagieren, haben nur eine Update()-Methode und erben von \class{GameStateComponent}. Dies sind vor allem verschiedene Input-Handler, welche Tastatur- und Mauseingaben verarbeiten und beispielsweise die Kameraposition und das Kameratarget ändern oder Spielobjekte bewegen.
\\\\
Spielkomponenten, die neben der Update()-Methode auch eine Draw()-Methode besitzen, erben von \class{DrawableGameStateComponent}. Dies sind vor allem die Elemente, aus denen die grafische Benutzeroberfläche zusammengesetzt ist, deren abstrakte Basisklasse \class{Widget} darstellt. [weitere Erklärungen zu Widgets...]
\\\\
Alle Spielobjekte implementieren die Schnittstelle \interface{IGameObject}. Die abstrakte Klasse \class{GameModel} repräsentiert dabei ein Spielobjekt, das aus einem 3D-Modell besteht, und hält zu diesem Zweck eine Referenz auf ein Objekt der Klasse \xna{Model} aus dem XNA-Framework sowie weitere Eigenschaften wie Position, Drehung und Skalierung.

Spielobjekte sind keine Komponenten, sondern werden in einer Spielwelt zusammenfasst, die durch die Klasse \class{World} repräsentiert wird. Die Spielwelt ist ein \class{DrawableGameStateComponent} und ruft in ihrer Update()- und Draw()-Methoden jeweils die dazugehörigen Methoden aller in ihr enthaltenen Spielobjekte auf.
\\\\
Shadereffekte werden durch die abstrakte Klasse \class{RenderEffect} und die von ihr abgeleiteten Klassen gekapselt. Ein \class{RenderEffect} enthält ein Rendertarget vom Typ \xna{RenderTarget2D} als Attribut und implementiert jeweils eine Begin()- und eine End-Methode. In der Methode Begin() wird das aktuell von XNA genutzte Rendertarget auf einem Stack gesichert und das Rendertarget des Effekts wird als aktuelles Rendertarget gesetzt.

Nach dem Aufruf von Begin() werden alle Draw()-Calls von XNA auf dem gesetzten Rendertarget ausgeführt. Es wird also in eine im \xna{RenderTarget2D}-Objekt enthaltene Bitmap gezeichnet. Dabei wird von den Draw()-Methoden der \class{GameModel}s die DrawModel(GameModel)-Methode des \class{RenderEffect}s aufgerufen, der die Modelle mit bestimmten Shadereffekten in die Bitmap zeichnet.

In der End()-Methode wird schließlich das auf dem Stack gesicherte vorher genutzte Rendertarget wiederhergestellt und das Rendertarget des \class{RenderEffect}s wird, unter Umständen verändert durch Post-Processing-Effekte, auf dieses übergeordnete Rendertarget gezeichnet.



\section{Klassendiagramm}

\section{Verwendete Entwurfsmuster}


\chapter{Klassenübersicht}

\newcommand{\property}[1]{\texttt{#1}}

\newcommand{\method}[1]{\texttt{#1}}

\newcommand{\keyword}[1]{\textcolor{BlueViolet}{#1}}

\newcommand{\ptype}[1]{\textcolor{OliveGreen}{#1}}

\newcommand{\varname}[1]{\textcolor{Black}{#1}}

\newcommand{\param}[1]{\textit{\textcolor{Gray}{#1}}}


\chapter{Klassenübersicht}

\newcommand{\property}[1]{\texttt{#1}}

\newcommand{\method}[1]{\texttt{#1}}

\newcommand{\keyword}[1]{\textcolor{BlueViolet}{#1}}

\newcommand{\ptype}[1]{\textcolor{OliveGreen}{#1}}

\newcommand{\varname}[1]{\textcolor{Black}{#1}}

\newcommand{\param}[1]{\textit{\textcolor{Gray}{#1}}}


\chapter{Klassenübersicht}

\newcommand{\property}[1]{\texttt{#1}}

\newcommand{\method}[1]{\texttt{#1}}

\newcommand{\keyword}[1]{\textcolor{BlueViolet}{#1}}

\newcommand{\ptype}[1]{\textcolor{OliveGreen}{#1}}

\newcommand{\varname}[1]{\textcolor{Black}{#1}}

\newcommand{\param}[1]{\textit{\textcolor{Gray}{#1}}}


\input{Klassenuebersicht.gentex}



\chapter{Abläufe}

\section{Sequenzdiagramme}
\subsection{Challengeende}
Knot ruft bei Kantenänderungen eine Methode auf, die ihm beim Hinzufügen zum ChallengeModeScreen zugewiesen wurde.
Diese prüft die beiden Knoten auf Gleichheit, indem sie die Knoten sich vergleichen lässt.
Sind beide Knoten gleich, wird die benötigte Zeit ermittelt und ein neuer Dialog erstellt, der die Zeit anzeigt und nach einem Benutzernamen fragt.
\newline
Dazu wird aus Options.Default der Benutzernamen abgefragt und als Standardwert eingetragen.
Wird ein neuer Benutzername eingegeben, wird dieser auch in Options.Default geändert.
Mit Enter bestätigt, wird eine zugewiesene Methode aufgerufen, die Benutzername und Zeit Challenge mit AddToHighscore(name, time) 
\newline
übergibt, was die Werte richtig einordnet und gegebenenfalls (wenn es gut genug ist) speichert.
Daraufhin öffnet diese Methode einen neuen Dialog, der die besten 10, soweit vorhanden, anzeigt und 2 Buttons 
von denen einer die Challenge neu startet und ein anderer den StartScreen aufruft.
\newline
\newline
\subsection{Screenwechsel CreativeMainScreen -> CreativeLoadScreen}
\newline
\newline
Wenn man Im CreativeMainScreen auf den Button "Load" klickt, wird der Wechsel zu CreativeLoadScreen eingeleitet.
Beim nächsten Update()-Call von XNA wird dann der PostProcessingEffect von CreativeLoad für ein Überblenden der Menüs z.B. auf FadeEffect gesetzt und der CreativeMainScreen deaktiviert, d.h. alle IGameComponent von CreativeMainScreen 
werden aus der Liste aller IGameComponent gelöscht. Dann wird CreativeLoadScreen aktiviert und dessen IGameComponent in die XNA IGameComponent-Liste eingetragen.
\newline
Daraufhin durchsucht CreativeLoadScreen das Speicherstand-Verzeichnis und lädt die Metadaten aller Speicherstände aller zwischengespeicherten Knoten. 
\newline
Noch nicht zwischengespeicherte Knoten werden auf Gültigkeit überprüft und wenn für Gültig befunden in den Zwischenspeicher geschrieben als Hash.
\newline
Anschließend werden die gefundenen gültigen Speicherstände in ein Menü eingetragen, welches dann beim nächsten Draw()-Call von XNA auf den Bildschirm gezeichnet wird.
\newline
Dieses Menü ist ein VerticalMenu, welches einen Button für jeden gültigen gefundenen Spielstand enthält. Durch einen Klick auf den jeweiligen Button 
startet man mit dem Spielstand im CreativeModeScreen.

\chapter{Klassenindex}

\section{Klassen}

\chapter{Klassenindex}

\section{Klassen}

\chapter{Klassenindex}

\section{Klassen}

\input{Klassenindex.gentex}
\input{Anmerkungen.tex}
\chapter{Glossar}


% F?r Abk?rzungen: s. LaTeX: \footnote, Zeigt am Ende der Seite z.B. f?r XYZ?, ? XYZ ist ... an.


\begin{longtable}{p{0.35\textwidth} p{0.5\textwidth}}



Bounding-Frustum & Ein Frustum, das z.B. aus der View- und Projection-Matrix berechnet werden kann und dabei hilft zu bestimmen, ob ein 3D-Modell in dem f"ur den Spieler sichtbaren Bereich des 3D-Raums liegt. \footnote{Genauere Definition: \url{http://msdn.microsoft.com/de-de/library/microsoft.xna.framework.boundingfrustum(v=xnagamestudio.40).aspx}} \\



Einstellungsdatei & Eine Datei im INI-Format, die die Einstellungen des Spiels speichert. Die zentrale Einstellungsdatei von Knot$^3$ liegt unter Windows in einem Unterordner des Dokumente-Ordners des aktuellen Windows-Benutzers, beispielsweise in \textit{C:\textbackslash Users\textbackslash Name\textbackslash Documents\textbackslash Knot3\textbackslash config.ini}. Wird Knot$^3$ unter Linux in Verbindung mit der offenen XNA-Implementierung Monogame gespielt, wird eine Datei aus dem Unterverzeichnis ".knot3" des Homeverzeichnisses ausgelesen, etwa \textit{/home/name/.knot3/config.ini}. \\



Inputhandler & Ein Inputhandler ist eine M"oglichkeit um Benutzereingaben zu verarbeiten. Der Imputhandler stellt beispielsweise Informationen "uber die Position der Maus bereit.  \\



Rendereffekt & Ein Effekt, der auf 3D-Modelle oder Sprites angewandt wird und durch Shader realisiert wird. \\



Rendertarget & Ein Rendertarget ist ein Pixel-Buffer. \\



Spielkomponentenliste & Die Sammlung von GameComponent-Elementen, die im Besitz des Spiels sind. \newline Quelle: \url{http://msdn.microsoft.com/de-de/library/microsoft.xna.framework.game.components(v=xnagamestudio.40).aspx} \\



Sprite & Ein Sprite (engl. unter anderem f"ur ein Geistwesen, Kobold) ist ein Grafikobjekt, das von der Grafikhardware "uber das Hintergrundbild bzw. den restlichen Inhalt der Bildschirmanzeige eingeblendet wird. Die Positionierung wird dabei komplett von der Grafikhardware erledigt. \newline Quelle: \url{http://de.wikipedia.org/wiki/Sprite_(Computergrafik)} \\



Spritestapel & Erm"oglicht das Zeichnen einer Gruppe von Sprites mithilfe derselben Einstellungen. \newline Quelle: \url{http://msdn.microsoft.com/de-de/library/microsoft.xna.framework.graphics.spritebatch(v=xnagamestudio.40).aspx}\\



Widget & Ein Widget ist eine Komponente eines grafischen Fenstersystems. Das Widget besteht zum einen aus dem Fenster, einem sichtbaren Bereich, der Maus- und/oder Tastaturereignisse empf"angt, und zum anderen aus dem nicht sichtbaren Objekt, das den Zustand der Komponente speichert und "uber bestimmte Zeichenoperationen den sichtbaren Bereich ver"andern kann. \newline Quelle: \url{http://de.wikipedia.org/wiki/Widget} \\



XNA-Content-Pipeline & Die XNA Game Studio-Inhalts-Pipeline ist eine Gruppe von Prozessen, die beim Build eines Spiels mit Grafikassets angewendet werden. Der Prozess beginnt mit einem Grafikasset, das urspr"unglich als Datei vorliegt. Dieses Grafikasset wird dann in Daten transformiert, die "uber die XNA Framework-Klassenbibliothek abgerufen und in einem XNA Game Studio-Spiel verwendet werden k"onnen. \newline Quelle: \url{http://msdn.microsoft.com/de-de/library/bb447745.aspx#fbid=cm5w_zXCQc2} \\



\end{longtable}

\end{document}
