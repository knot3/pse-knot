\documentclass{report}
\sloppy
\usepackage[usenames,dvipsnames]{color}
\usepackage[top=1.5in, bottom=1in, left=1in, right=1in]{geometry}
\usepackage[utf8]{inputenc}
\usepackage[ngerman]{babel}
\usepackage{microtype}
\usepackage{enumitem}
\usepackage{graphicx}
\usepackage{wrapfig}
\usepackage{listings}
\usepackage{hyperref}


%
% Glossare.
%
% HINWEIS: perl nötig.
%
%\usepackage[nomain,toc,section, sanitize={description=false}]{glossaries}
\usepackage[nomain]{glossaries}
\renewcommand*{\glspostdescription}{} % Kein automatischer Satzpunkt nach einem Eintrag.
\newglossary{FA}{fai}{fao}{Fachausdrücke}
\newglossary{AK}{aki}{ako}{Abkürzungen}
\makeglossaries
%% Glossar-Einträge im Voraus bekannt machen,
%% damit sie im Folgenden verwendbar sind.
\loadglsentries[FA]{Fachausdruecke}
\loadglsentries[AK]{Abkuerzungen}

%
% HINWEIS:
%
% Begriffe, die im Glossar auftauchen müssen auch im
% Text auftauchen und durch gls{Begriff} markiert worden
% sein, damit sie im Verzeichnis gelistet werden!
%







\begin{document}

\title{\textbf{\Huge{ENTWURFSDOKUMENT}}\\\Large{(V. 1.0)}~\\~\\~\\
		\textbf{\Large{KNOT$^3$}}\\
		PSE WS 2013/14}
\author{\Large{Auftraggeber:}\\
        \Large{Karlsruher Institut für Technologie}\\
        \Large{Institut für Betriebs- und Dialogsysteme}\\
        \Large{Prof. Dr.-Ing. C. Dachsbacher}\\~\\
        \Large{Betreuer:}\\
        \Large{Dipl.-Inf. Thorsten Schmidt}\\
        \Large{Dipl.-Inf. M. Retzlaff}\\~\\
        \Large{Auftragnehmer:}\\
        \Large{Tobias Schulz, Maximilian Reuter, Pascal                    Knodel,}\\
	 	\Large{Gerd Augsburg, Christina Erler, Daniel Warzel}~\\~\\}
\date{\today}

\maketitle

\tableofcontents

\chapter{Einleitung}
Nach Pflichtenheft und Entwurf stand nun die Implementierung auf dem Plan.
Auf den folgenden Seiten beschreiben wir, wie das vonstattenging, welche Änderungen wir vornehmen mussten und auf welche Probleme wir gestoßen sind.
Durch die vielfältigen Möglichkeiten, die ein frei veränderbarer Knoten bietet, entstand eine hoch komplexe Interaktionsvielfalt,
von der manche Bereiche im Entwurf nicht ausreichend Beachtung fanden, bzw. finden konnte.
Dadurch sind einige Änderungen zum ursprünglichen Entwurf notwendig geworden, die sich aber vor allem auf Erweiterungen bezieht.
Dennoch hat sich unser Entwurf als gut erwiesen, da alle Änderungen ohne Umbau der grundlegenden Strukturen integriert werden konnten.

\chapter{Aufbau}

\newcommand{\class}[1]{\textcolor{BlueViolet}{\sloppy{#1}}}
\newcommand{\interface}[1]{\textcolor{OliveGreen}{\mbox{\sloppy{#1}}}}
\newcommand{\xna}[1]{\textcolor{Gray}{\sloppy{#1}}}



\section{Architektur}

Die grundlegende Architektur des Spiels basiert auf der Spielkomponenten-Infrastruktur des XNA-Framework, die mit Spielzuständen kombiniert wird. Die abstrakten Klassen \class{GameStateComponent} und \class{DrawableGameStateComponent} erben von den von XNA bereitgestellten Klassen \xna{GameComponent} und \xna{DrawableGameComponent} implementieren zusätzlich die Schnittstelle \interface{IGameStateComponent}. Sie unterscheiden sich von den XNA-Basisklassen dadurch, dass sie immer eine Referenz auf einen bestimmten Spielzustand halten und nur in Kombination mit diesem zu verwenden sind.
\\\\
Die Spielzustände erben von der abstrakten Basisklasse \class{GameScreen} und halten eine Liste von \interface{IGameStateComponent}-Objekten. Wird ein Spielzustand aktiviert, indem von einem anderen Spielzustand aus zu ihm gewechselt wird oder indem er der Startzustand ist, dann weist er seine Liste von \interface{IGameStateComponent}-Objekten dem \xna{Components}-Attribut der \class{Game}-Klasse zu, die von der vom XNA-Framework bereitgestellten abstrakten Klasse \xna{Game} erbt. So ist zu jedem Zeitpunkt während der Laufzeit des Spiels ein Spielzustand aktiv, der die aktuelle Liste von Spielkomponenten verwaltet.
\\\\
Die Spielkomponenten, die nicht gezeichnet werden und nur auf Eingaben reagieren, haben nur eine Update()-Methode und erben von \class{GameStateComponent}. Dies sind vor allem verschiedene Input-Handler, die Tastatur und Mauseingaben verarbeiten und die beispielsweise die Kameraposition und das Kameratarget ändern oder Spielobjekte bewegen.
\\\\
Spielkomponenten, die neben der Update()-Methode auch eine Draw()-Methode besitzen, erben von \class{DrawableGameStateComponent}. Dies sind vor allem die Elemente, aus denen die grafischen Benutzeroberfläche zusammengesetzt ist, deren abstrakte Basisklasse \class{Widget} darstellt. [weitere Erklärungen zu Widgets...]
\\\\
Alle Spielobjekte implementieren die Schnittstelle \interface{IGameObject}. Die abstrakte Klasse \class{GameModel} repräsentiert dabei ein Spielobjekt, das aus einem 3D-Modell besteht, und hält zu diesem Zweck eine Referenz auf ein Objekt der Klasse \xna{Model} aus dem XNA-Framework sowie weitere Eigenschaften wie Position, Drehung und Skalierung.

Spielobjekte sind keine Komponenten, sondern werden in einer Spielwelt zusammenfasst, die durch die Klasse \class{World} repräsentiert wird. Die Spielwelt ist ein \class{DrawableGameStateComponent} und ruft in ihrer Update()- und Draw()-Methoden jeweils die dazugehörigen Methoden aller in ihr enthaltenen Spielobjekte auf.
\\\\
Shadereffekte werden durch die abstrakte Klasse \class{RenderEffect} und die von ihr abgeleiteten Klassen gekapselt. Ein \class{RenderEffect} enthält ein Rendertarget vom Typ \xna{RenderTarget2D} als Attribut und implementiert jeweils eine Begin()- und eine End-Methode. In der Methode Begin() wird das aktuell von XNA genutzte Rendertarget auf einem Stack gesichert und das das Rendertarget des Effekts wird als aktuelles Rendertarget gesetzt.

Nach dem Aufruf von Begin() werden alle Draw()-Calls von XNA auf dem gesetzten Rendertarget ausgeführt. Es wird also in eine im \xna{RenderTarget2D}-Objekt enthaltene Bitmap gezeichnet. Dabei wird von den Draw()-Methoden der \class{GameModel}s die DrawModel(GameModel)-Methode des \class{RenderEffect}s aufgerufen, der die Modelle mit bestimmten Shadereffekten in die Bitmap zeichnet.

In der End()-Methode wird schließlich das auf dem Stack gesicherte vorher genutzte Rendertarget wiederhergestellt und das Rendertarget des \class{RenderEffect}s wird, unter Umständen verändert durch Post-Processing-Effekte, auf dieses übergeordnete Rendertarget gezeichnet.



\section{Klassendiagramm}

\section{Verwendete Entwurfsmuster}


\chapter{Klassenübersicht}
\chapter{Abläufe}

\section{Sequenzdiagramme}

\subsection{Challengeende}

Knot ruft bei Kantenänderungen eine Methode auf, die ihm beim Hinzufügen zum \class{ChallengeModeScreen} zugewiesen wurde. Diese prüft die beiden Knoten auf Gleichheit, indem sie die Knoten sich vergleichen lässt. Sind beide \class{Knot}en gleich, wird die benötigte Zeit ermittelt und ein neuer \class{Dialog} erstellt, der die Zeit anzeigt und nach einem Benutzernamen fragt.
\newline
\newline
Dazu wird aus Options.Default der Benutzernamen abgefragt und als Standardwert eingetragen. Wird ein neuer Benutzername eingegeben, wird dieser auch in Options.Default geändert. Mit Enter bestätigt wird eine zugewiesene Methode aufgerufen, die Benutzername und Zeit dem \class{Challenge}-Objekt mit AddToHighscore(name, time) übergibt, was die Werte richtig einordnet und gegebenenfalls (wenn die Zeit gut genug ist) speichert.
\newline
\newline
Daraufhin öffnet diese Methode einen neuen \class{Dialog}, der, soweit vorhanden, die besten zehn Einträge sowie zwei Buttons anzeigt, von denen einer die \class{Challenge} neu startet und ein anderer den \class{StartScreen} aufruft. Obwohl nur die besten zehn Highscore-Einträge angezeigt werden, sind trotzdem alle bisher hinzugefügten Einträge sowohl in dem Challenge-Objekt als auch in dem Speicherformat der Challenge enthalten, da sie zur Berechnung der durchschnittlichen Zeit benötigt werden, die u.A. im \class{ChallengeStartScreen} zu jeder Challenge angezeigt werden kann.

\subsection{Screenwechsel vom CreativeMainScreen in den CreativeLoadScreen}

Wenn man in dem Menü, das durch den \class{CreativeMainScreen} dargestellt wird, auf den Button "Load" klickt, wird der Wechsel zu \class{CreativeLoadScreen} eingeleitet. Dabei wird ein neues \class{CreativeLoadScreen} erstellt und oben auf den Stack in \class{Knot3Game} gelegt, in dem die Spielstände abgespeichert werden.
\newline
\newline
Beim nächsten Update()-Call des XNA-Frameworks wird der PostProcessingEffect von \class{CreativeLoadScreen} für ein Überblenden der Rendertargets auf \class{FadeEffect} gesetzt und der \class{CreativeMainScreen} wird deaktiviert, d.h. alle \interface{IGameScreenComponent}s von \class{CreativeMainScreen} werden aus der Liste der aktuell in XNA registrierten \interface{IGameScreenComponent} gelöscht. Außerdem wird die BeforeExit()-Methode von \class{CreativeMainScreen} aufgerufen, in der eventuell anfallende Aufräumarbeiten ausgeführt werden können.
\newline
\newline
Dann wird \class{CreativeLoadScreen} aktiviert und dessen Entered()-Methode aufgerufen, die dessen \interface{IGameScreenComponent}s mittels der in \class{GameScreen} implementierten AddGameComponents()-Methode in die \xna{IGameComponent}-Liste des XNA-Frameworks einträgt. Diese Operation ist rekursiv, d.h. die über die SubComponents()-Methode jedes \interface{IGameScreenComponent}s ermittelten Unterkomponenten werden von AddGameComponents() ebenfalls erfasst.
\newline
\newline
In diesem Beispiel wäre eine direkt von \class{CreativeLoadScreen} hinzugefügte Komponente das Spielstand-Menü, das von der Klasse \class{VerticalMenu} repräsentiert wird. Dieses gibt über die SubComponents()-Methode alle in dem Menü enthaltenen \class{MenuItem}s zurück, damit diese von AddGameComponents() ebenfalls als \interface{IGameScreenComponent} registriert und dargestellt werden können.
\newline
\newline
Daraufhin durchsucht \class{CreativeLoadScreen} das Spielstand-Verzeichnis und lädt die Metadaten der Speicherstände aller zwischengespeicherten Knoten.
Noch nicht zwischengespeicherte Knoten werden auf Gültigkeit überprüft und wenn für Gültig befunden in den Zwischenspeicher geschrieben als Hash.
\newline
\newline
Anschließend werden die gefundenen gültigen Spielstände in ein Menü eingetragen, welches dann beim nächsten Draw()-Call von XNA auf den Bildschirm gezeichnet wird.
Dieses Menü ist ein \class{VerticalMenu}, welches einen \class{MenuButton} für jeden gültigen gefundenen Spielstand enthält. Durch einen Klick auf den jeweiligen Button startet man mit dem Spielstand im \class{CreativeModeScreen}.

\chapter{Klassenindex}

\printindex
\chapter{Anmerkungen}
\chapter{Glossar}


% F�r Abk�rzungen: s. LaTeX: \footnote, Zeigt am Ende der Seite z.B. f�r XYZ�, � XYZ ist ... an.


\begin{longtable}{|p{0.35\textwidth}|p{0.5\textwidth}|}

\hline

Cel-Shader-Effekt & ...\\

\hline

Inputhandler & ...\\

\hline

Rendertarget & ...\\

\hline

Sprite & ...\\

\hline

Spritestapel & ...\\

\hline

\end{longtable}

\end{document}
